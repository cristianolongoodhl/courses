\documentclass[8pt]{beamer}
\usepackage[nobglogo]{beamerthemedmi-owled}
\usepackage[utf8x]{inputenc}
\usepackage{default}
\usepackage{url}
\usepackage{verbatim}
\usepackage{graphicx}
\usepackage{mathrsfs}
\usepackage[official]{eurosym}

%\usepackage{listings}


\mode<presentation>
{
  \usetheme{dmi-owled}
  %\usetheme{Warsaw}
  % or ...

  \setbeamercovered{transparent}
  % or whatever (possibly just delete it)
}

\title{Introduzione agli Open Data\\
Stumenti per dati a 2 e 3 stelle}

\author{Cristiano Longo\\ 
{\small{longo@dmi.unict.it}}}



\date{Universit\`a di Catania}

\begin{document}
\maketitle
\setcounter{tocdepth}{1}

\begin{frame}
\frametitle{Riepilogo - 5 stelle}
Classificazione a 5 stelle:\footnote{vedi anche \url{http://5stardata.info/}}
\begin{enumerate}
 \item dati disponibili rilasciati con una licenza aperta;
 \item dati in un formato leggibile da un agente automatico;
 \item dati in un formato aperto;
 \item dati resi disponibili con le tecnologie del web semantico;
 \item dati collegati ad altri dataset.
\end{enumerate}

\uncover<2->{
Vedremo alcuni strumenti per trattare dati a 2 e 3 stele.
}
\end{frame}

\begin{frame}
\frametitle{Dati ad una e due stelle}

Esempi di dati ad una e due stelle: files pdf.
\vspace{\baselineskip}

\uncover<2->{
Le scansioni pdf non sono trattabili. Vedi ad esempio

\begin{center}
\begin{small}
\url{http://www.comune.messina.it/informazioni/trasparenza-valutazione-e-merito/allegato/pubblicazione-art-1-c-735-lf-2007.pdf} .
\end{small} 
\end{center}
}

\uncover<3->{
Le tabelle dentro i PDF \emph{selezionabili} (PDFa) possono invece essere estratte. Vedi ad esempio
\begin{center}
\begin{small}
\url{http://www.comune.messina.it/informazioni/trasparenza-valutazione-e-merito/dati-relativi-a-incarichi-e-consulenze/allegati/report-dipendenti-2010.pdf} .
\end{small} 
\end{center}
}

\end{frame}

\begin{frame}
\frametitle{Da due a tre stelle: Tabula}
\emph{Tabula}\footnote{\url{http://tabula.technology/}} \`e uno strumento open source che permette di 
estrarre le tabelle nei files pdf e convertirle in formato CSV.
\vspace{\baselineskip}

Sul sito principale sono presenti i pacchetti per mac e windows. Per gli altri sistemi operativi
\`e necessario scaricare \texttt{tabula-jar.zip}. 

\uncover<2->{
Per avviare l'applicazione, scompattare il pacchetto ed eseguire
il seguente comando

\begin{quote}
java -Dfile.encoding=utf-8 -Xms256M -Xmx1024M -jar tabula.jar 
\end{quote}
}

\uncover<3->{
Nella form disponibile al seguente indirizzo, importare il file PDF da convertire:
\begin{center}
\begin{small}
\url{http://127.0.0.1:8080/} .
\end{small} 
\end{center}
}

\uncover<4->{
Nell'anteprima, selezionare le aree contenenti le tabelle da convertire.
Se l'anteprima dell'import non mostra problemi, scaricare il CSV. Nel caso in
cui la tabella sia su pi\`u pagine, \`e possibile selezionare pi\`u di un'area.
Quando tutte le aree di interesse sono state selezionate, premere su \emph{Download all data}.
}
\end{frame}


\begin{frame}
 \frametitle{Trattamento di Dati Tabellari - Datawrapper}
 \emph{Datawrapper}\footnote{\url{https://datawrapper.de/}} \`e uno
 strumento open-source per la creazione di grafici a partire da dati 
 tabellari in formato CSV. I grafici creati con il servizio online possono essere
 visualizzati nelle proprie pagine web.
 \vspace{\baselineskip} 
 
 \uncover<2->{
 Per creare un nuovo grafico \`e necessario innanzitutto caricare un 
 file CSV che contiene i dati da mostrare. \`E necessario specificare
 la fonte, che poi verrà riportata nel grafico finale come meta-dato.
 \vspace{\baselineskip} 
 }
 
 \uncover<3->{
  Il secondo passo \`e speficare i tipi delle colonne, nel caso in cui
  il detect automatico non abbia funzionato. \`E necessario che sia presente
  almeno un campo con tipo number.
 \vspace{\baselineskip} 
 }
 
 \uncover<4->{
 Infine si sceglie il tipo di grafico e si esporta in PDF o altro formato.
 }
 
\end{frame}

\begin{frame}
 \frametitle{Trattamento di Dati Tabellari - Datawrapper - incidenti stradali}
 
 Vediamo come usare Datawrapper per visualizzare i dati sugli incidenti stradali
 nel comune di catania. In particolare evidenziamo il numero di incidenti per mese,
 tralasciando invece il numero di morti e feriti.
 \vspace{\baselineskip} 
 
 \uncover<2->{
 \emph{Passo 1} - scaricare i dati in CSV dal seguente indirizzo
 \begin{center}
   \begin{small}
      \url{http://opendata.comune.catania.gov.it/dataset/incidenti-2012} .
   \end{small}
 \end{center}
 }
 
 \uncover<3->{
 \emph{Passo 2} - creare un nuovo grafico su Datawrapper, caricando il dataset 
 e indicando la fonte. 
 \vspace{\baselineskip} 
 }

 \uncover<4->{
 \emph{Passo 3} - accertarsi che le colonne \emph{INCIDENTI CON FERITI} e \emph{INCIDENTI MORTALI} siano
 rilevate come colonne numeriche. Eliminare le due colonne \emph{FERITI} e \emph{MORTI} dalla visualizzazione.
 \vspace{\baselineskip} 
 }

 \uncover<5->{
 \emph{Passo 4} - Inserire il titolo (scheda \emph{Annotate}) e selezionare il tipo di grafico 
 (scelta consigliata: istogramma raggruppato). 
 Per confrontare il numero di incidenti mortali e con feriti per mese \`e conveniente trasporre
 il grafico.
 \vspace{\baselineskip} 
 }

  \uncover<6->{
 \emph{Passo 5} - Infine pubblicare o esportare il grafico.
 \vspace{\baselineskip} 
 }
\end{frame}

\begin{frame}
 \frametitle{Trattamento di Dati Tabellari - Datawrapper - Risultati Elettorali}
 
 Vediamo come usare Datawrapper per visualizzare i risultati elettorali delle
 elezioni amministrative del 2013 del comune di Catania.
 \vspace{\baselineskip} 
 
 \uncover<2->{
 I dati sono disponibili in CSV al seguente indirizzo
 \begin{center}
   \begin{small}
      \url{http://opendata.comune.catania.gov.it/dataset/elezioni-amministrative-2013-voti-liste-consiglio-comunale} .
   \end{small}
 \end{center}
 }
 
 \uncover<3->{
 Si crea il grafico con Data Wrapper come visto prima. La prima colonna deve essere nascosta e di dati trasposti.  
 \vspace{\baselineskip} 
 }

 \uncover<4->{
 \emph{Passo 3} - accertarsi che le colonne \emph{INCIDENTI CON FERITI} e \emph{INCIDENTI MORTALI} siano
 rilevate come colonne numeriche. Eliminare le due colonne \emph{FERITI} e \emph{MORTI} dalla visualizzazione.
 \vspace{\baselineskip} 
 }

 \uncover<5->{
 \emph{Passo 4} - Inserire il titolo (scheda \emph{Annotate}) e selezionare il tipo di grafico 
 (scelta consigliata: istogramma raggruppato). 
 Per confrontare il numero di incidenti mortali e con feriti per mese \`e conveniente trasporre
 il grafico.
 \vspace{\baselineskip} 
 }

  \uncover<6->{
 \emph{Passo 5} - Infine pubblicare o esportare il grafico.
 \vspace{\baselineskip} 
 }
\end{frame}

\begin{frame}
 \frametitle{Trattamento di Dati Tabellari - Pulizia}
 A volte, anche se forniti in formati tabellari aperti, i dati non sono adatti ad essere 
 processati automaticamente. Ad esempio, lo stesso nome pu\`o essere indicato tutto in
 maiuscolo e, nella stessa tabella, solo con la prima lettera maiuscola. 
 \vspace{\baselineskip}
 
 Nell'esempio visto prima riguardante gli incarichi del comune di messina, il simbolo 
 dell'euro e l'utilizzo della virgola come separatore della parte decimale per i compensi
 rende impossibile il trattamento di questi dati con Datawrapper.
\end{frame}

\begin{frame}
 \frametitle{Trattamento di Dati Tabellari - Pulizia - Open Refine (1/3)}
 \emph{Open Refine}\footnote{\url{http://openrefine.org}} \`e un progetto open-source
 che, assieme ad altre funzionalit\`a, fornisce alcuni strumenti per la pulizia e 
 la trasformazione dei dati.
 \vspace{\baselineskip}
 
 \uncover<2>{
 Per utilizzare open-refine \`e sufficiente scompattare il pacchetto fornito sul sito
 e avviare l'applicatovo \texttt{refine}. All'avvio viene comunicato l'indirizzo 
 attraverso il quale accedere all'applicativo.\footnote{Solitamente \url{http://127.0.0.1:3333/} .} 
 }
\end{frame}

\begin{frame}
 \frametitle{Trattamento di Dati Tabellari - Pulizia - Open Refine (2/3)}
 Su Open Refine \`e possibile caricare i dati da diverse tipologie di fonti (web, file locale, \ldots).
 Carichiamo il file sugli incarichi al comune di Messina precedentemente creato con Tabula. In questa fase \`e
 necessario specificare che la prima riga contiene l'header e che il file \`e in formato CSV con i campi
 separati da virgole.
 \vspace{\baselineskip}
 
 \uncover<2->{
 Quando l'anteprima non rileva errori o imprecisioni, scegliere un nome e creare il progetto.
 \vspace{\baselineskip}
 }
 
\uncover<3->{
  Effettuiamo le operazioni di pulizia. Nel nostro caso dobbiamo rimuovere il simbolo dell'euro e sostituire
  la virgola con il punto nella colonna \texttt{Importo Erogato}. Per fare questo selezionare 
  \texttt{Edit Cells - Transform} nel men\`u a tendina relativo alla colonna. Si apre a questo punto una
  form nella quale specificare le modifiche da effettuare in linguaggio GREL.\footnote{Per il linguaggio gRel vedi 
  \url{https://github.com/OpenRefine/OpenRefine/wiki/Understanding-Expressions} e \url{https://github.com/OpenRefine/OpenRefine/wiki/GREL-Functions} .}
  Nel nostro caso inserire la seguente espressione:
  \begin{center}
   \begin{small}
   value.replace("\euro{}","").replace(".","").replace(",",".").trim()    
   \end{small}
  \end{center}
 \vspace{\baselineskip}
 }

 \uncover<4->{
 Infine, si esporta il progetto in formato CSV.
 }
\end{frame}

\begin{frame}
 \frametitle{Trattamento di Dati Tabellari - Pulizia - Open Refine (3/3)}
 
 Altre caratteristiche interessanti di Open Refine sono:
 
 \begin{itemize}[<+->]
  \item \emph{esportazione dei template}  che permette di riapplicare le stesse trasformazioni ad un file
  di formato simile in termini di colonne;
  \item \emph{suddivisione di celle} ad esempio per separare nome e cognome, o via e numero civico;
  \item \emph{conversione in maiuscolo}.
 \end{itemize}
\end{frame}

\begin{frame}
 \frametitle{Trattamento di Dati Tabellari - Pulizia - Clustering}
 
 Una delle funzionalit\`a pi\`u utilizzate di Open Refine \`e il clustering.
 \vspace{\baselineskip}
 
 Il \emph{clustering} permette di accorpare, usando delle euristiche predefinite, 
 campi con valori differenti ma che, secondo le informazioni ottenute utilizzando
 le suddette euristiche, rappresentano lo stesso oggetto.
 \vspace{\baselineskip}
 
 Ad esempio, le stringhe ``\texttt{Cristiano Longo}'', ``\texttt{Longo, Cristiano}'',
 ``\texttt{cristiano longo}'' sarano identificate come appartententi allo stesso cluster,
 e sar\`a possibile sostituirle con un'unica stringa (ad esempio, ``\texttt{Cristiano Longo}''). 
\end{frame}

\begin{frame}
 \frametitle{Trattamento di Dati Tabellari - Pulizia - Esempi di Clustering: Basi Dati delle PA}

 La tecnica del clustering \`e stata utilizzata per raffinare
 le basi dati della Pubblica Amministrazione comunicati all'Agenzia 
 dell'Italia Digitale.
  \begin{center}
    \begin{small}
      \url{http://basidati.agid.gov.it/catalogo/download.html} 
    \end{small} 
  \end{center}

\uncover<2->{
  Tra tutti i file disponibili, consideriamo quello relativo alle
  \emph{Basi di Dati}. Cliccando sulla colonna amministrazione
  e selezionando Edit cells - Cluster and edit si apre la piattaforma
  di clustering.  
 \vspace{\baselineskip}
}  

\uncover<3->{
  Tra i vari cluster riconosciuti, il primo contiene i seguenti valori:
  \texttt{ISTITUTO COMPRENSIVO}, \texttt{ISTITUTO COMPRENSIVO - ISTITUTO COMPRENSIVO},
  \texttt{Istituto Comprensivo}. \`E possibile decidere di sostituire tutti
  questi valori con un'unica stringa, specificata nell'ultima colonna, 
  selezionando il flag \texttt{Merge}.
 \vspace{\baselineskip}
}

\uncover<4->{
  Il processo di clustering pu\`o essere applicato iterativamente.
}
\end{frame}

\section{Dati Geografici}

\begin{frame}
 \frametitle{Dati Geografici}
 
 I dati geospaziali e geolocalizzati rappresentano una grossa fetta degli open data disponibili.
 Vengono forniti in diversi formati e modalit\`a. Vedi come esempio l'elenco delle farmacie
 pubblicato dal comune di catania.\footnote{\url{http://opendata.comune.catania.gov.it/dataset/test-geo}}
 \vspace{\baselineskip}
 
 Esamineremo alcuni tool per l'utilizzo di dati geo-referenziati.
 \vspace{\baselineskip}
 
 Spesso la posizione di alcuni oggetti di interesse (scuole, asili, discariche, ...) viene fornita
 senza le coordinate geospaziali. Vedremo alcuni servizi di \emph{geocoding} per ottenere le coordinate
 dagli indirizzi.
\end{frame}

\begin{frame}
 \frametitle{Dati Geografici - CartoDB - Importare un Dataset}
 Sono disponibili alcuni strumenti online per visualizzare dati geografici.
 \emph{CartoDB}\footnote{\url{https://cartodb.com/}} \`e un servizio web che permette di realizare mappe da esporre sul
 proprio sito realizzate a partire da dataset contenenti informazioni georeferenziate.
 \vspace{\baselineskip}

 \uncover<2->{
 Una volta entrati nel proprio account, CartoDB fornisce due viste: una per le mappe 
 ed una per i dataset. Un dataset pu\`o essere caricato da file o importato da diverse
 sorgenti (Dropbox, Google sheet, ...), recuperato online oppure caricato da un file
 locale.
 }
 \vspace{\baselineskip}
 
 \uncover<3->{
 I dataset cos\`i ottenuti vengono salvati su un database relazionale interno a
 CartoDB. \`E possibile selezionare gli oggetti da mostrare attraverso dei filtri
 in linguaggio SQL.
 }
\end{frame}

\begin{frame}
 \frametitle{Trattamento di Dati Geografici - CartoDB - Creare Mappe}
 La vista \emph{Maps} permette invece di realizzare delle mappe. \`E possibile
 collegare una mappa ad un dataset quando la mappa viene creata, oppure 
 specificare il dataset in un momento successivo.
 \vspace{\baselineskip}
 

 \uncover<2->{
 Per creare una mappa, selezionare innanzitutto la \emph{cartografia} da utilizzare 
 (\emph{change base-map}). Inoltre, specificare i contenuti degli info-box.
 } 
 \vspace{\baselineskip}
 
  \uncover<3->{
  Infine, si pu\`o pubblicare la mappa. Fatto questo sar\`a possibile includerla
  nel proprio sito.
  } 
\end{frame}

\begin{frame}
 \frametitle{Trattamento di Dati Geografici - uMap}

 \emph{uMap}\footnote{\url{https://umap.openstreetmap.fr}} \`e un servizio
 libero e open source per la creazione e fruizione di mappe basate su
 \emph{Open Street Map}.\footnote{\url{http://www.openstreetmap.org/}}
 \vspace{\baselineskip}

 Open Street Map \`e un database collaborativo realizzato sul modello di
 wikipedia (tutti gli utenti possono aggiungere contenuti). Tutti i dati 
 presenti su Open Street Map sono rilasciati con licenza aperta.
\end{frame}
 
\begin{frame}
 \frametitle{Trattamento di Dati Geografici - uMap - Creare Mappe}

 Con uMap \`e possibile creare mappe personalizzate da conservare nel
 proprio account.
 \vspace{\baselineskip}
 
\uncover<2->{
 Oltre ad aggiungere punti di interesse (POI) manualmente, \`e possibile
 importare punti caricando file in diversi formati o importandone alcuni
 disponibili sul web.
}
 \vspace{\baselineskip}

\uncover<3->{
 I marker relativi ai punti di interesse sono personalizzabili in 
 termini di forma, colori ed eventuali simboli. ATTENZIONE: \`e possibile
 inserire una immagine solo per marker di tipo \emph{derivato}.
}
 \vspace{\baselineskip}

\uncover<4->{
 Infine, anche la mappa utilizzata come sfondo pu\`o essere 
 selezionata tra quelle disponibili.
}
\end{frame}

\begin{frame}
 \frametitle{Trattamento di Dati Geografici - Geocoding}
 
 Nei dataset geografici a volte non sono indicate le coordinate,
 ma solo gli indirizzi. Un esempio \`e il dataset 
 \emph{ANAGRAFE DEGLI EDIFICI PUBBLICI}\footnote{\url{http://www.comune.palermo.it/opendata_dld.php?id=319}}
 del comune di Palermo.
 \vspace{\baselineskip}

 Col termine \emph{geocoding} si intende un processo che permetta
 di ottenere da un indirizzo (stato, citt\`a, via, civico) le
 corrispondenti coordinate.
 \vspace{\baselineskip}
 
\end{frame}

\begin{frame}[fragile]
 \frametitle{Trattamento di Dati Geografici - Geocoding - Esempio: Beni Confiscati Palermo}

 I beni confiscati in gestione al comune di Palermo sono indicati nel dataset
 \emph{ANAGRAFE DEGLI EDIFICI PUBBLICI}\footnote{\url{http://www.comune.palermo.it/opendata_dld.php?id=319}}
 del comune di Palermo. Vediamo come visualizzarli usando CartoDB.
 \vspace{\baselineskip}

 Innanzitutto si importi su CartoDB il dataset visto prima. Successivamente
 selezioniamo tra quelli importati solo i beni confiscati. Questi sono quelli
 che nella tabella rilasciata dal comune contengono riferimenti alla legge
 575/65. La query da effettuare \`e la seguente:
 \begin{verbatim}
 SELECT * FROM _4 WHERE destinazione LIKE '%575/65%'
 \end{verbatim}
 \vspace{\baselineskip}

 Alla creazione della mappa viene richiesto su quali colonne effettuare
 il geocoding. Selezionare l'opzione \texttt{Street Addresses} e indicare
 \texttt{indirizzo} come Street Address e \texttt{no.civico} come componente
 aggiuntiva dell'indirizzo (tasto ``+''). L'applicazione del geocoding aggiunger\`a
 dei campi con le indicazioni geografiche nel dataset.
\end{frame}

\end{document}
