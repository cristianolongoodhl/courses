\documentclass[8pt]{beamer}
\usepackage[nobglogo]{beamerthemedmi-owled}
\usepackage[utf8x]{inputenc}
\usepackage{default}
\usepackage{url}
\usepackage{verbatim}
\usepackage{graphicx}
\usepackage{mathrsfs}
\usepackage{dl}
\usepackage{mls}

\mode<presentation>
{
  \usetheme{dmi-owled}
  %\usetheme{Warsaw}
  % or ...

  \setbeamercovered{transparent}
  % or whatever (possibly just delete it)
}

\title{Web Reasoning 2015/2016\\
Lezione 2}

\author{Cristiano Longo\\ 
{\small{longo@dmi.unict.it}}}



\date{Universit\`a di Catania}
\newcommand{\urlsingle}[1]{{\small {\center {\url{#1}}}}}
\begin{document}
\maketitle
\setcounter{tocdepth}{1}

\section{URI/IRI}

\begin{frame}
	\frametitle{The Semantic Web Stack}
	L'insieme delle tecnologie impiegate nel Web Semantico
	costituisce il \emph{Semantic Web Stack}.

	\phantom{Tratteremo la gran parte di queste.}

	\begin{figure}
	    \includegraphics[width=180px]{imgs/Semantic_Web_Stack0.png}
	    \caption{The Semantic Web Stack}
	\end{figure}
\end{frame}

\begin{frame}
	\frametitle{The Semantic Web Stack}
	L'insieme delle tecnologie impiegate nel Web Semantico 
	costituisce il \emph{Semantic Web Stack}.
	
	Tratteremo la gran parte di queste.
	
	\begin{figure}
	    \includegraphics[width=180px]{imgs/Semantic_Web_Stack0all.png}
	    \caption{The Semantic Web Stack} 
	\end{figure}
\end{frame}

\begin{frame}
	\frametitle{URI/IRI}
	
	\begin{itemize}
	  \item \emph{Uniform Resource Identifiers} (URI) - RFC3986
	  \item \emph{Internationalized Resource Identifiers (IRIs)} (IRI) - RFC3987
	\end{itemize}	
	
	Nel Web Semantico le IRI servono ad \emph{indicare} oggetti di qualsiasi
	natura, concreti o astratti.
	
	\begin{figure}
	    \includegraphics[width=180px]{imgs/Semantic_Web_Stack_uri.png}
	    \caption{The Semantic Web Stack} 
	\end{figure}
\end{frame}

\begin{frame}[fragile]
	\frametitle{Uniform Resource Identifier (URI)}
	La specifica \emph{Uniform Resource Identifier (URI)} ingloba
	le nozioni di Uniform Resource Name e Uniform Resource Locator
	ed \`e definita nell'RFC 3986.
	\vspace{\baselineskip}
	
	La \emph{grammatica} delle URI \`e espressa usando la notazione
	\emph{Augmented BNF for Syntax Specifications (ABNF)}, definita
	nell'RFC 2234. 
	\vspace{\baselineskip}

	I simboli terminali \tt{ALPHA} e \tt{DIGIT} indicano rispettivamente
	lettere e numeri disponibili nell'encoding \tt{US-Ascii}.
	\vspace{\baselineskip}
	
	\`E previsto un meccanismo di \emph{percent-encoding} per utilizzare
	caratteri non in \tt{US-Ascii}.
	
	\begin{verbatim}
  pct-encoded = "%" HEXDIG HEXDIG	
	\end{verbatim}	
	
\end{frame}

\begin{frame}[fragile]
	\frametitle{URI - caratteri riservati}
	
	I caratteri riservati solitamente ricoprono la funzione di 
	separatori per delimitare le varie parti della URI.
	
	\begin{verbatim}
 reserved    = gen-delims / sub-delims
 
 gen-delims  = ":" / "/" / "?" / "#" / "[" / "]" / "@"

 sub-delims  = "!" / "$" / "&" / "'" / "(" / ")"
                  / "*" / "+" / "," / ";" / "=" 
                  

                  
	\end{verbatim}
	\vspace{\baselineskip}
	
\uncover<2->{
	Nel caso in cui si abbia la necessit\`a di usare caratteri riservati
	nel corpo di alcune parti della URI e non come separatori, \`e possibile
	farlo applicando il percent-encodind.
	\vspace{\baselineskip}
}

\phantom{
 	I caratteri che possono comparire all'interno delle parti di una URI
 	sono detti \emph{non riservati}. In alre parole, i caratteri non riservati
 	sono tutti i caratteri permessi in una URI ad esclusione di quelli riservati.
}	
\end{frame}

\begin{frame}[fragile]
	\frametitle{URI - caratteri riservati}
	
	I caratteri riservati solitamente ricoprono la funzione di 
	separatori per delimitare le varie parti della URI.
	
	\begin{verbatim}
 reserved    = gen-delims / sub-delims
 
 gen-delims  = ":" / "/" / "?" / "#" / "[" / "]" / "@"

 sub-delims  = "!" / "$" / "&" / "'" / "(" / ")"
                  / "*" / "+" / "," / ";" / "=" 
                  
 unreserved  = ALPHA / DIGIT / "-" / "." / "_" / "~"
                  
	\end{verbatim}
	\vspace{\baselineskip}
	
	Nel caso in cui si abbia la necessit\`a di usare caratteri riservati
	nel corpo di alcune parti della URI e non come separatori, \'e possibile
	farlo applicando il percent-encodind.
	\vspace{\baselineskip}

 	I caratteri che possono comparire all'interno delle parti di una URI
 	sono detti \emph{non riservati}. In alre parole, i caratteri non riservati
 	sono tutti i caratteri permessi in una URI ad esclusione di quelli riservati.	
\end{frame}

\begin{frame}[fragile]
	\frametitle{URI - componenti di una URI}
	
	La sintassi delle URI \`e definita come segue:
	
	\begin{verbatim}
  URI = scheme ":" hier-part [ "?" query ] [ "#" fragment ]

  hier-part   = "//" authority path-abempty
              / path-absolute
              / path-rootless
              / path-empty                  
	\end{verbatim}
	\vspace{\baselineskip}
	
	Riportiamo due esempi di uri.
	\vspace{\baselineskip}

	\begin{figure}
	    \includegraphics[width=250px]{imgs/uri-pieces.png}
	    \caption{Esempi di URI}
	\end{figure}
	
\end{frame}

\begin{frame}[fragile]
	\frametitle{URI - componenti - schema}
    Ogni URI inizia con uno \emph{schema}. Ogni schema identificana
    ulteriori restrizioni sintattiche nelle URI che vi ricadono.
    Alcuni schemi noti sono \tt{http}, \tt{mail}, \tt{tel}.\footnote{Un elenco
    esaustivo di URI scheme \`e disponibile alla pagina
    \url{http://www.iana.org/assignments/uri-schemes/uri-schemes.xhtml}.}
    \vspace{\baselineskip}
    
    Gli schemi hanno la seguente sintassi
    \begin{verbatim}
  scheme = ALPHA *( ALPHA / DIGIT / "+" / "-" / "." )
    \end{verbatim}    
\end{frame}

\begin{frame}[fragile]
	\frametitle{URI - componenti - authority}
	
	Molti URI scheme prevedono l'indicazione di una parte
	gerarchica denominata \emph{authority}. L'authority indica
	generalmente l'organizzazione responsabile delle risorse
	indicate dalla URI con l'authority in questione.
	Un esempio di authority (dello schema \tt{http}) \`e \tt{www.google.it}.
    \vspace{\baselineskip}
    
    La sintassi dell'authority \`e la seguente:
    \begin{verbatim}
  authority   = [ userinfo "@" ] host [ ":" port ]
    \end{verbatim}    

	\tt{userinfo} contiene il nome utente ed eventuali altre informazioni 
	necessarie per l'accesso alla risorsa indicata dalla URI (ad esempio la
	password).
    \begin{verbatim}
  userinfo    = *( unreserved / pct-encoded / sub-delims / ":" ) .
    \end{verbatim}    
    
    \tt{host} \`e fondamentalmente un indirizzo IP o un nome di
    dominio.\footnote{per le definizioni di \tt{IP-literal}, \tt{IPv4address} e
    \tt{reg-name} consultare lo RFC3986.}
    \begin{verbatim}
  host        = IP-literal / IPv4address / reg-name
    \end{verbatim}    
    
    Infine, \tt{port} \`e una indicazione sull'eventual porta di rete (vedi
    stack TCP/IP) per accedere al servizio.
    \begin{verbatim}
  port        = *DIGIT
    \end{verbatim}  
\end{frame}

\begin{frame}[fragile]
	\frametitle{URI - componenti - path}
	
	Il \emph{path} permette di specificare ulteriormente un percorso,
	usualmente gerarchico, per indicare una risorsa nell'amito dello
	schema e della eventuale authority. Un path \`e costituito da una
	sequenza di  \emph{segmenti} separati dal carattere \tt{/}.
	
	\begin{verbatim}
  URI = scheme ":" hier-part [ "?" query ] [ "#" fragment ]

  hier-part   = "//" authority path-abempty
              / path-absolute
              / path-rootless
              / path-empty                  

  path-abempty  = *( "/" segment ) ; begins with "/" or is empty
  path-absolute = "/" [ segment-nz *( "/" segment ) ] 
              ; begins with "/" but not "//"
  path-rootless = segment-nz *( "/" segment ) ; begins with a segment
  path-empty    = 0<pchar> ; zero characters  

  pchar = unreserved / pct-encoded / sub-delims / ":" / "@"  
	\end{verbatim}
	
	\begin{figure}
	    \includegraphics[width=250px]{imgs/uri-pieces.png}
	    \caption{Esempi di URI}
	\end{figure}
	
\end{frame}

\begin{frame}[fragile]
	\frametitle{URI - componenti - query}
	
	La \emph{query} \`e un componente opzionale non gerarchico che serve
	a specificare ulteriormente i parametri di indirizzamento della
	risorsa. Usualmente viene utilizzata per il passaggio dei parametri.
	
	\begin{verbatim}
  URI = scheme ":" hier-part [ "?" query ] [ "#" fragment ]

  query = *( pchar / "/" / "?" )
	\end{verbatim}
	
	\vspace{\baselineskip}
	
	\begin{figure}
	    \includegraphics[width=250px]{imgs/uri-pieces.png}
	    \caption{Esempi di URI}
	\end{figure}
	
\end{frame}

\begin{frame}[fragile]
	\frametitle{URI - componenti - fragment}
	
	Il \emph{fragment} permette infine di indirizzare una risorsa
	\emph{secondaria} in riferimento ad una \emph{primaria}, ad esempio un
	frammento di una pagina web o un dato istante in un video.
	
	\begin{verbatim}
  URI = scheme ":" hier-part [ "?" query ] [ "#" fragment ]

  fragment    = *( pchar / "/" / "?" )
	\end{verbatim}
	
	\vspace{\baselineskip}
	
	\begin{figure}
	    \includegraphics[width=250px]{imgs/uri-pieces.png}
	    \caption{Esempi di URI}
	\end{figure}
	
\end{frame}

\begin{frame}[fragile]
	\frametitle{Internationalized Resource Identifiers (IRIs)}

	La specifica delle \emph{Internationalized Resource Identifier (IRI)}
	estende quella delle URI con delle funzionali\`a per l'internazionalizzazione.
	\vspace{\baselineskip}
	 
	Una IRI \`e una sequenza di caratteri nell'\emph{Universal Character Set
	(Unicode/ISO 10646)}.
	
	La grammatica delle IRI pu\`o essere definita a partire da quella delle URI
	estendendo l'insieme dei caratteri non riservati per comprendere i caratteri
	nell'UCS.
	\begin{verbatim}
  iunreserved = ALPHA / DIGIT / "-" / "." / "_" / "~" / ucschar
	\end{verbatim}
	\vspace{\baselineskip}
	
	Per maggiori approfondimenti si veda l'RFC 3987.
\end{frame}

\end{document}
