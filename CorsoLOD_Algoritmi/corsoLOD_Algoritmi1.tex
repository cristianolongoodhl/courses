\documentclass[8pt]{beamer}
\usepackage[nobglogo]{beamerthemedmi-owled}
\usepackage[utf8x]{inputenc}
\usepackage{default}
\usepackage{url}
\usepackage{verbatim}
\usepackage{graphicx}
\usepackage{mathrsfs}
\usepackage{dl}
\usepackage{mls}
%\usepackage{listings}


\mode<presentation>
{
  \usetheme{dmi-owled}
  %\usetheme{Warsaw}
  % or ...

  \setbeamercovered{transparent}
  % or whatever (possibly just delete it)
}

\title{Linked Open Data e Semantic Web:\\
Fondamenti e Linguaggi di Interrogazione\\
Parte Prima}

\author{Cristiano Longo\\ 
{\small{longo@dmi.unict.it}}}



\date{Universit\`a di Catania, 2014-2015}
\newcommand{\urlsingle}[1]{{\small {\center {\url{#1}}}}}
\begin{document}
\maketitle
\setcounter{tocdepth}{1}

\section{Introduzione}

\begin{frame}
\frametitle{Argomenti}
Questa presentazione tratter\`a i seguenti argomenti:
\begin{itemize}
 \item Motivazioni del Web Semantico
 \item Definizione formale di Ontologie
 \item Interrogazioni sulle Ontologie
 \item Vocabolari
\end{itemize}
\end{frame}

\begin{frame}
\frametitle{World Wide Web Consortium}
Le tecnologie del \emph{Web Semantico} rispondono ad un insieme di
standard e protocolli promossi e mantenuti dal \emph{World Wide Web Consortium} 
(in breve \emph{W3C}, vedi \url{http://www.w3.org}).
\vspace{\baselineskip}

Il W3C \`e un consorzio di standardizzazione per il Web che
conta 403 membri tra aziende e organizzazioni governative: CNR,
Microsoft Corporation, Apple Inc., Intel Corporation, Facebook, Google Inc., \ldots
\vspace{\baselineskip}

Altri standard sviluppati in seno al W3C sono: URL, HTTP, XML, HTML, CSS, SOAP, 
WSDL, Javascript. 
\end{frame}

\begin{frame}
\frametitle{Limiti del World Wide Web (1/4)}
\begin{quote}
The Web was designed as an information space, with the goal that it should be 
useful not only for human-human communication, but also that machines would be
able to participate and help. One of the major obstacles to this has been the
fact \textbf{that most information on the Web is designed for human consumption}, and 
even if it was derived from a database with well defined meanings (in at least 
some terms) for its columns, that \textbf{the structure of the data is not evident to 
a robot browsing the web.}  
\end{quote}
\vspace{\baselineskip}
\emph{Semantic Web Roadmap}, Tim Berners-Lee, 1998.
\end{frame}

\begin{frame}
\frametitle{Limiti del World Wide Web (2/4)}
Alcuni problemi nell'interpretazione di testi derivano da:
\begin{description}
 \item[Lingue Differenti] e.g. $Parigi$ e $Paris$ possono indicare la stessa citt\`a.
 \item[Omonimie] e.g. esistono svariate citt\`a chiamate 
 \emph{Paris} nel mondo (Arkansas, Idaho, Illinois, Kentucky,
 Maine, Michigan, Missouri, New York, \ldots);
\end{description}
\end{frame}

\begin{frame}
\frametitle{Limiti del World Wide Web (3/4)}

La situazione si complica in presenza di contenuti multimediali.

\begin{figure}
    \includegraphics[width=250px]{unrecognizable.jpg} 
\end{figure}
\end{frame}

\begin{frame}
\frametitle{Limiti del World Wide Web (4/4)}
Come conseguenza, spesso \`e impossibile eseguire su web ricerce 
\emph{complesse} ottenendo risultati accurati. Ad esempio, cercando sul web
\emph{``Federico II places''} non si ottengono risultati in prima pagina su 
Federico II, ma solo sull'omonima universit\`a:
  
  \begin{small}
    \begin{enumerate}
   \item Universit\`a degli Studi di Napoli "Federico II" | OPEN Places
   \item AOU - Policlinico "Federico II" - Napoli, Italy - Hospital | Facebook
   \item Federico II Ingegneria Via Claudio - College and University | Facebook
   \item MARIA CATERINA FONTE - www.docenti.unina.it
  \end{enumerate}
  \end{small}
\end{frame}

\begin{frame}
\frametitle{Il Web Semantico (1/2)}
\begin{quote}
[\ldots] the Semantic Web approach instead develops languages for expressing
information in a machine processable form. 
\end{quote}
\emph{Semantic Web Roadmap}, Tim Berners-Lee, 1998.

\begin{figure}
    \includegraphics[width=250px]{federicoII_dbpedia.png} 
    \caption{Federico II su dbpedia.org}
\end{figure}
\end{frame}

\begin{frame}
\frametitle{Il Web Semantico (1/2)}
I \emph{linguaggi di rappresentazione} usati nel Web semantico 
hanno una \emph{sintassi rigorosa} e sono dotati di una \emph{semantica formale}.
\vspace{\baselineskip}

Questo rende possibile effettuare interrogazioni complesse sui dataset, 
ottenendo dei risultati precisi anche se a volte parziali:

\begin{center}
Q = “Luoghi di nascita di Federico II e dei suoi parenti stretti” . 
\end{center}
\end{frame}

\begin{frame}
\frametitle{Linked Open Data Cloud (1/2)}
\begin{quote}
The Semantic Web is a web of data, in some ways like a global database.
\end{quote}
\small{\emph{Semantic Web Roadmap}, Tim Berners-Lee, 1998.}
\begin{figure}
    \includegraphics[width=250px]{lod-cloud_colored_1000px.png} 
    \caption{Linked Open Data Cloud}
\end{figure}
\end{frame}

\begin{frame}
\frametitle{Linked Open Data Cloud (2/2)}
Nel \emph{Linked Open Data Cloud} sono presenti 365 dataset (fonte \url{http://stats.lod2.eu/}).
\vspace{\baselineskip}

Alcuni dataset:
\begin{itemize}
 \item \emph{DBPedia} (\url{dbpedia.org}) corrispondente a \url{wikipedia.org};
 \item \emph{Linked Movie Database} (\url{http://linkedmdb.org/}) controparte sul Web Semantico di \emph{Internet Movie Database} 
 (\url{http://www.imdb.com/});
 \item \emph{Linked GeoData} (\url{http://linkedgeodata.org}) 
  contiene i dati di \emph{OpenStreetMap} (\url{http://www.openstreetmap.org/});
  \item \emph{AGROVOC} (\url{http://aims.fao.org/agrovoc}) \`e il dataset della FAO (\url{http://fao.org});
  \item \emph{Europeana} (\url{http://pro.europeana.eu/linked-open-data}) contiene dati su beni culturali e tradizioni Europee. 
\end{itemize}
\end{frame}

\begin{frame}
\frametitle{Linked Data (1/2)}

I dataset nel Web Semantico possono essere \emph{collegati} tra loro. Ad esempio, una stessa risorsa pu\`o essere descritta 
sotto diversi aspetti in dataset differenti.
\vspace{\baselineskip}

Ad esempio, la citt\`a di Catania \`e presente:
\begin{itemize}
 \item come pubblica amministrazione nel dataset del sistema pubblico di connettivit\`a e cooperazione\footnote{\url{http://spcdata.digitpa.gov.it/}}
 con la url \url{http://spcdata.digitpa.gov.it/Amministrazione/c_c351};
 \item come divisione amministrativa nel dataset \url{http://www.geonames.org};
 \item come area territoriale nel dataset dell'ISTAT  \url{http://linkedstat.spaziodati.eu/}.
\end{itemize}
\end{frame}

\begin{frame}
\frametitle{Linked Data (2/2)}
\`E possibile effettuare interrogazioni che coinvolgano diversi
dataset (anche eterogenei).
\vspace{\baselineskip}

Ad esempio, la seguente query pu\`o essere eseguita
interrogando un data set contenente dati storici ed uno 
sulle strutture ricettive:


\begin{center}
Q = “Strutture ricettive nei luoghi di nascita di Federico II e dei suoi parenti stretti.”
\end{center}
\end{frame}

\section{Ontologie}
\begin{frame}
\frametitle{Ontologie}
I dataset del Web Semantico vengono spesso definiti \emph{ontologie}.
\vspace{\baselineskip}

Una \emph{ontologia} \`e una descrizione \emph{parziale} del mondo:
\begin{itemize}
 \item descrive una porzione del mondo, spesso \`e limitata ad un'unico \emph{dominio di conoscenza};
 \item non si assume che i fatti non esplicitamente presenti nell'ontologia siano falsi (\emph{Open World Assumption}).
\end{itemize}
\vspace{\baselineskip}

Essa \`e costituita da un insieme finito di \emph{affermazioni}. Ad esempio:
\begin{itemize}
 \item Tutti gli esseri umani sono mortali;
 \item Socrate \`e mortale;
 \item Alice \`e la madre di Roberto.
\end{itemize}
\end{frame}

\begin{frame}
\frametitle{Ontologie - Affermazioni}
Le affermazioni contenute in una ontologia sono di tre tipi:
\vspace{\baselineskip}

\emph{Constraints:} impongono dei vincoli \emph{semantici} sul dominio di conoscenza 
che si va a rappresentare. La notazione richiama quella insiemistica;
\vspace{\baselineskip}
\[
 HumanBeing \Issub Mortal 
\]
\emph{Property Assertions}: impongono una relazione tra due elementi del dominio;
\[
 Alice\,motherOf\,Bob 
\]
\emph{Class Assertions}: indicano l'appartenenza di un elemento ad un insieme.
\[
 HumanBeing(Socrate) 
\]
\end{frame}

\begin{frame}
\frametitle{Ontologie - Reasoning}

Con il termine \emph{reasoning} si intende l'attivit\`a di estrazione di 
conoscenza \emph{implicita} in una ontologia.
\[
\begin{array}{ccc}
\left \{ \begin{array}{l}
HumanBeing \Issub Mortal , \\
HumanBeing(Socrate)  
\end{array} \right \}&\Longrightarrow &  Mortal(Socrate)\\
\end{array}
\]
Le attivit\`a di reasoning sono rese possibili dalle \emph{semantiche formali}
associate ai linguaggi di rappresentazione utilizzati.
\end{frame}


\newcommand{\CNames}{N_C}
\newcommand{\PNames}{N_P}
\newcommand{\INames}{N_I}
\newcommand{\VNames}{V}

\begin{frame}
\frametitle{Ontologie - Definizione}

Siano $\CNames$, $\PNames$, $\INames$ tre insiemi infiniti, numerabili e 
a due a due disgiunti di nomi di \emph{classe}, \emph{propriet\`a} e \emph{individuo},
rispettivamente.
\vspace{\baselineskip}

Una \emph{ontologia} \`e un insieme finito di asserzioni dei seguenti tipi:
\[
 \begin{array}{l|l|l}
  & Sintassi & Semantica \\
  \hline
  &&\\
  \mbox{Constraints} & C \Issub D & (\forall x)(x \in C \rightarrow x \in D) \\
  & R \Issub S & (\forall x, y)([x, y] \in R \rightarrow [x,y] \in S) \\
  & \dom(R) \Issub C & (\forall x, y)([x, y] \in R \rightarrow x \in C) \\
  & \range(R) \Issub C & (\forall x, y)([x, y] \in R \rightarrow y \in C) \\
  &&\\
  \hline
  &&\\
  \mbox{Class Assertions} & C(a) & a \in C\\
  &&\\
  \hline
  &&\\
  \mbox{Property Assertions} & a\,P\,b\;(\mbox{equivalente }P(a,b)) & [a,b] \in P \\
  &&\\
  \hline  
 \end{array}
\]
dove $C, D \in \CNames$, $R, S \in \PNames$ e $a, b \in \INames$.
\end{frame}

\newcommand{\Ont}{\mathcal{O}}
\newcommand{\Ontp}{\mathcal{O'}}

\begin{frame}
\frametitle{Ontologie - Esempio}
Riportiamo un esempio di ontologia. Siano $HumanBeing, Mortal \in \CNames$,
$teacherOf \in \PNames$, $Socrate, Platone \in \INames$.
\vspace{\baselineskip}

\phantom{Mediante \emph{reasoning} \`e possibile esplicitare ulteriori 
affermazioni.}
\[
 \begin{array}{clccl}
  \Ont  =  &\{HumanBeing \Issub Mortal, & &\phantom{\Ontp =} & \phantom{\{Mortal(Socrate), }\\
  &\phantom{\{}\range(teacherOf) \Issub HumanBeing, & \phantom{\Longrightarrow} && \phantom{\{HumanBeing(Platone)}\\
  &\phantom{\{}HumanBeing(Socrate), &&&\phantom{\{Mortal(Platone) \}}\\
  &\phantom{\{}Socrate\,teacherOf\,Platone \}\\
 \end{array}
\]
\end{frame}

\begin{frame}
\frametitle{Ontologie - Esempio}
Riportiamo un esempio di ontologia. Siano $HumanBeing, Mortal \in \CNames$,
$teacherOf \in \PNames$, $Socrate, Platone \in \INames$.
\vspace{\baselineskip}

Mediante \emph{reasoning} \`e possibile esplicitare ulteriori 
affermazioni.
\[
 \begin{array}{clccl}
  \Ont  =  &\{\mathbf{HumanBeing \Issub Mortal}, & &\Ontp = & \{\mathbf{Mortal(Socrate)}, \\
  &\phantom{\{}\range(teacherOf) \Issub HumanBeing, & \Longrightarrow&& \phantom{\{HumanBeing(Platone)}\\
  &\phantom{\{}\mathbf{HumanBeing(Socrate)}, &&&\phantom{\{Mortal(Platone)}\}\\
  &\phantom{\{}Socrate\,teacherOf\,Platone \}\\
 \end{array}
\]
\end{frame}

\begin{frame}
\frametitle{Ontologie - Esempio}
Riportiamo un esempio di ontologia. Siano $HumanBeing, Mortal \in \CNames$,
$teacherOf \in \PNames$, $Socrate, Platone \in \INames$.
\vspace{\baselineskip}

Mediante \emph{reasoning} \`e possibile esplicitare ulteriori 
affermazioni.
\[
 \begin{array}{clccl}
  \Ont  =  &\{HumanBeing \Issub Mortal, & &\Ontp = & \{Mortal(Socrate), \\
  &\phantom{\{}\mathbf{\range(teacherOf) \Issub HumanBeing}, & \Longrightarrow&& \phantom{\}}\mathbf{HumanBeing(Platone)},\\
  &\phantom{\{}HumanBeing(Socrate), &&&\phantom{\{Mortal(Platone)}\}\\
  &\phantom{\{}\mathbf{Socrate\,teacherOf\,Platone} \}\\
 \end{array}
\]
\end{frame}

\begin{frame}
\frametitle{Ontologie - Esempio}
Riportiamo un esempio di ontologia. Siano $HumanBeing, Mortal \in \CNames$,
$teacherOf \in \PNames$, $Socrate, Platone \in \INames$.
\vspace{\baselineskip}

Mediante \emph{reasoning} \`e possibile esplicitare ulteriori 
affermazioni.
\[
 \begin{array}{clccl}
  \Ont  =  &\{\mathbf{HumanBeing \Issub Mortal} & &\Ontp = & \{Mortal(Socrate), \\
  &\phantom{\{}\range(teacherOf) \Issub HumanBeing, & \Longrightarrow && \phantom{\{}\mathbf{HumanBeing(Platone)}\\
  &\phantom{\{}HumanBeing(Socrate), &&&\phantom{\{}\mathbf{Mortal(Platone)} \}\\
  &\phantom{\{}Socrate\,teacherOf\,Platone \}\\
 \end{array}
\]
\end{frame}

\section{Interrogazioni}

\begin{frame}
\frametitle{Interrogazioni}

Il metodo pi\`u immediato per ottenere informazioni da una ontologia \`e il 
\emph{Conjunctive Query Answering}. Consideriamo ad esempio la seguente ontologia:

\begin{tabular}{cc}
\hline
$\begin{array}{cl}
  \Ont  =  &  \{Female(Elise), Female(Alice), Male(Bob), \\
  &\phantom{\{}Male(Charlie), Male(Daniel), \\
  &\phantom{\{}Alice\,childOf\,Elise, Charlie\,childOf\,Elise, \\
  &\phantom{\{}Daniel\,childOf\,Alice, Daniel\,childOf\,Bob, \\
  &\phantom{\{}Francis\,childOf\,Charlie \}
 \end{array}$ & \includegraphics[width=120px]{family.png} \\
\hline
\end{tabular}

Alcune interrogazioni che \`e possibile effettuare con il conjunctive query 
answering sono:
\begin{itemize}
 \item ``Trova tutti gli individui maschi.''
 \item ``Chi sono gli individui con almeno un figlio maschio?''
 \item ``Chi sono i figli di $Alice$?''
 \item ``Chi sono gli individui con almeno un figlio maschio ed una femmina?''
 \item ``Chi sono gli individui maschi con almeno un figlio maschio?''
\end{itemize}
\end{frame}

\begin{frame}
\frametitle{Formule Atomiche}

Per definire in maniera rigorosa le query congiuntive \`e necessario
definire preliminarmente l'insieme delle \emph{formule atomiche}. 
\vspace{\baselineskip}

Sia $\VNames = \{x, y, z, ... \}$ l'insieme infinito, numerabile 
e disgiunto da $\CNames$, $\PNames$ e $\INames$ delle \emph{variabili}.
Le \emph{formule atomiche} sono espressioni dei due seguenti tipi:
\[
 C(x), \quad P(x, y)
\]
con $x, y \in \INames \cup \VNames$, $C \in \CNames$ e $P \in \PNames$.
\vspace{\baselineskip}

Esempi di formule atomiche sono:
\begin{itemize}
 \item $HumanBeing(x)$,
 \item $x\,childOf\,Alice$,
 \item $Bob\,childOf\,x$,
 \item $x\,childOf\,y$,
 \item $Mortal(Socrate)$,
 \item $Alice\,childOf\,Elise$
\end{itemize}
con $HumanBeing, Mortal \in \CNames$, $childOf \in \PNames$, 
$Alice, Bob, Elise \in \INames$ e $x, y \in \VNames$.
\end{frame}

\begin{frame}
\frametitle{Formule Atomiche Chiuse}
Una formula atomica nella quale non compaiano variabili si dice \emph{chiusa}.
\vspace{\baselineskip}

Negli esempi che seguono sono evidenziate le formule atomiche chiuse:
\begin{itemize}
 \item $HumanBeing(x)$,
 \item $x\,childOf\,Alice$,
 \item $Bob\,childOf\,x$,
 \item $x\,childOf\,y$,
 \item $Mortal(Socrate)$,
 \item $Alice\,childOf\,Elise$
\end{itemize}
con $HumanBeing, Mortal \in \CNames$, $childOf \in \PNames$, 
$Alice, Bob, Elise \in \INames$ e $x, y \in \VNames$.
\vspace{\baselineskip}

\phantom{Le asserzioni presenti nelle ontologie sono formule atomiche chiuse.}
\end{frame}

\begin{frame}
\frametitle{Formule Atomiche Chiuse}
Una formula atomica nella quale non compaiano variabili si dice \emph{chiusa}.
\vspace{\baselineskip}

Negli esempi che seguono sono evidenziate le formule atomiche chiuse:
\begin{itemize}
 \item $HumanBeing(x)$,
 \item $x\,childOf\,Alice$,
 \item $Bob\,childOf\,x$,
 \item $x\,childOf\,y$,
 \item $\mathbf{Mortal(Socrate)}$,
 \item $\mathbf{Alice\,childOf\,Elise}$
\end{itemize}
con $HumanBeing, Mortal \in \CNames$, $childOf \in \PNames$, 
$Alice, Bob, Elise \in \INames$ e $x, y \in \VNames$.
\vspace{\baselineskip}

Le asserzioni presenti nelle ontologie sono formule atomiche chiuse.
\end{frame}

\begin{frame}
\frametitle{Query Congiuntive}
Una \emph{query congiuntiva} \`e una congiunzione finita di formule atomiche $T_1 \wedge \ldots \wedge T_n$.
\vspace{\baselineskip}

Alcuni esempi di query congiuntive:
\begin{itemize}
 \item ``Trova tutti gli individui maschi.''
 \[
  Male(x)
 \]
 \item ``Chi sono gli individui con almeno un figlio maschio?''
\[
 y\,childOf\,x\,\wedge\,Male(y)  
\]
 \item ``Chi sono i figli di $Alice$?''
\[
 x\,childOf\,Alice
\]
%  \item ``Chi sono gli individui con almeno un figlio maschio ed una femmina?''
% \[
%  y\,childOf\,x\,\wedge\,Male(y)\,\wedge\,z\,childOf\,x\,\wedge\,Female(z)   
% \]
%  \item ``Chi sono gli individui maschi con almeno un figlio maschio?''
% \[
%  Male(x)\,\wedge\,y\,childOf\,x\,\wedge\,Male(y)
% \]
con $x, y \in \VNames$, $Male, Female \in \CNames$, $childOf \in \PNames$ e $Alice \in \INames$.
 \end{itemize}
\end{frame}

\begin{frame}
\frametitle{Sostituzioni (1/2)}

Per definire le \emph{soluzioni} (risposte) delle query congiuntive introduciamo la
nozione di \emph{sostituzione}.
\vspace{\baselineskip}

Una \emph{sostituzione} $\sigma=[x_1 \rightarrow a_1, \ldots, x_n \rightarrow a_n]$
($x_1, \ldots, x_n \in \VNames$, $a_1, \ldots, a_n \in \INames$)
\`e una mappa finita che associa nomi di individui a variabili.
\vspace{\baselineskip}

Sia $T$ una formula atomica e $\sigma=[x_1 \rightarrow a_1, \ldots, x_n \rightarrow a_n]$
una sostituzione. L'\emph{applicazione} $T\sigma$ di $\sigma$ a $T$ \`e la formula atomica 
che si ottiene sostituendo in $T$ ad ogni occorrenza della variabile $x_i$ il
corrispondente nome di individuo $a_i$, per ogni $1\leq i\leq n$.
\vspace{\baselineskip}

Alcuni esempi:
\[
 \begin{array}{lcl}
  Male(x)[x \rightarrow Bob] & = & Male(Bob) \\
  Male(x)[y \rightarrow Bob] & = & \phantom{Male(x)} \\
  (x\,childOf\,y)[x \rightarrow Alice] & = & \phantom{Alice\,childOf\,y} \\
  (x\,childOf\,y)[x \rightarrow Alice, y \rightarrow Elise] & = & \phantom{Alice\,childOf\,Elise}
 \end{array}
\]
con $x,y \in \VNames$, $Male \in \CNames$, $childOf \in \PNames$ e $Alice, Bob, Elise \in \INames$.
\end{frame}

\begin{frame}
\frametitle{Sostituzioni (1/2)}

Per definire le \emph{soluzioni} (risposte) delle query congiuntive introduciamo la
nozione di \emph{sostituzione}.
\vspace{\baselineskip}

Una \emph{sostituzione} $\sigma=[x_1 \rightarrow a_1, \ldots, x_n \rightarrow a_n]$
($x_1, \ldots, x_n \in \VNames$, $a_1, \ldots, a_n \in \INames$)
\`e una mappa finita che associa nomi di individui a variabili.
\vspace{\baselineskip}

Sia $T$ una formula atomica e $\sigma=[x_1 \rightarrow a_1, \ldots, x_n \rightarrow a_n]$
una sostituzione. L'\emph{applicazione} $T\sigma$ di $\sigma$ a $T$ \`e la formula atomica 
che si ottiene sostituendo in $T$ ad ogni occorrenza della variabile $x_i$ il
corrispondente nome di individuo $a_i$, per ogni $1\leq i\leq n$.
\vspace{\baselineskip}

Alcuni esempi:
\[
 \begin{array}{lcl}
  Male(x)[x \rightarrow Bob] & = & Male(Bob) \\
  Male(x)[y \rightarrow Bob] & = & Male(x) \\
  (x\,childOf\,y)[x \rightarrow Alice] & = & \phantom{Alice\,childOf\,y} \\
  (x\,childOf\,y)[x \rightarrow Alice, y \rightarrow Elise] & = & \phantom{Alice\,childOf\,Elise}
 \end{array}
\]
con $x,y \in \VNames$, $Male \in \CNames$, $childOf \in \PNames$ e $Alice, Bob, Elise \in \INames$.
\end{frame}

\begin{frame}
\frametitle{Sostituzioni (1/2)}

Per definire le \emph{soluzioni} (risposte) delle query congiuntive introduciamo la
nozione di \emph{sostituzione}.
\vspace{\baselineskip}

Una \emph{sostituzione} $\sigma=[x_1 \rightarrow a_1, \ldots, x_n \rightarrow a_n]$
($x_1, \ldots, x_n \in \VNames$, $a_1, \ldots, a_n \in \INames$)
\`e una mappa finita che associa nomi di individui a variabili.
\vspace{\baselineskip}

Sia $T$ una formula atomica e $\sigma=[x_1 \rightarrow a_1, \ldots, x_n \rightarrow a_n]$
una sostituzione. L'\emph{applicazione} $T\sigma$ di $\sigma$ a $T$ \`e la formula atomica
che si ottiene sostituendo in $T$ ad ogni occorrenza della variabile $x_i$ il
corrispondente nome di individuo $a_i$, per ogni $1\leq i\leq n$.
\vspace{\baselineskip}

Alcuni esempi:
\[
 \begin{array}{lcl}
  Male(x)[x \rightarrow Bob] & = & Male(Bob) \\
  Male(x)[y \rightarrow Bob] & = & Male(x) \\
  (x\,childOf\,y)[x \rightarrow Alice] & = & Alice\,childOf\,y \\
  (x\,childOf\,y)[x \rightarrow Alice, y \rightarrow Elise] & = & \phantom{Alice\,childOf\,Elise}
 \end{array}
\]
con $x,y \in \VNames$, $Male \in \CNames$, $childOf \in \PNames$ e $Alice, Bob, Elise \in \INames$.
\end{frame}

\begin{frame}
\frametitle{Sostituzioni (1/2)}

Per definire le \emph{soluzioni} (risposte) delle query congiuntive introduciamo la
nozione di \emph{sostituzione}.
\vspace{\baselineskip}

Una \emph{sostituzione} $\sigma=[x_1 \rightarrow a_1, \ldots, x_n \rightarrow a_n]$
($x_1, \ldots, x_n \in \VNames$, $a_1, \ldots, a_n \in \INames$)
\`e una mappa finita che associa nomi di individui a variabili.
\vspace{\baselineskip}

Sia $T$ una formula atomica e $\sigma=[x_1 \rightarrow a_1, \ldots, x_n \rightarrow a_n]$
una sostituzione. L'\emph{applicazione} $T\sigma$ di $\sigma$ a $T$ \`e la formula atomica
che si ottiene sostituendo in $T$ ad ogni occorrenza della variabile $x_i$ il
corrispondente nome di individuo $a_i$, per ogni $1\leq i\leq n$.
\vspace{\baselineskip}

Alcuni esempi:
\[
 \begin{array}{lcl}
  Male(x)[x \rightarrow Bob] & = & Male(Bob) \\
  Male(x)[y \rightarrow Bob] & = & Male(x) \\
  (x\,childOf\,y)[x \rightarrow Alice] & = & Alice\,childOf\,y \\
  (x\,childOf\,y)[x \rightarrow Alice, y \rightarrow Elise] & = & Alice\,childOf\,Elise
 \end{array}
\]
con $x,y \in \VNames$, $Male \in \CNames$, $childOf \in \PNames$ e $Alice, Bob, Elise \in \INames$.
\end{frame}

\begin{frame}
\frametitle{Sostituzioni (2/2)}

L'applicazione di sostituzioni a query congiuntive si definisce come segue.
\vspace{\baselineskip}

Sia $\sigma=[x_1 \rightarrow a_1, \ldots, x_n \rightarrow a_n]$ una sostituzione
e siano $T_1, \ldots, T_m$ formule atomiche. Allora
\[
 (T_1 \wedge \ldots \wedge T_m)\sigma \defAs T_1\sigma \wedge \ldots \wedge T_m\sigma . 
\]
\vspace{\baselineskip}

Alcuni esempi:
\[
 \begin{array}{lcl}
 (y\,childOf\,x\,\wedge\,Male(y))[x \rightarrow Alice] & \defAs & \phantom{y\,childOf\,Alice\,\wedge\,Male(y)}  \\
 (y\,childOf\,x\,\wedge\,Male(y))[x \rightarrow Alice, z \rightarrow Bob] & \defAs & \phantom{y\,childOf\,Alice\,\wedge\,Male(y)}  \\
 (y\,childOf\,x\,\wedge\,Male(y))[x \rightarrow Alice, y \rightarrow Bob] & \defAs & \phantom{Bob\,childOf\,Alice\,\wedge\,Male(Bob)}  \\
 \end{array}
\]
\end{frame}

\begin{frame}
\frametitle{Sostituzioni (2/2)}

L'applicazione di sostituzioni a query congiuntive si definisce come segue.
\vspace{\baselineskip}

Sia $\sigma=[x_1 \rightarrow a_1, \ldots, x_n \rightarrow a_n]$ una sostituzione
e siano $T_1, \ldots, T_m$ formule atomiche. Allora
\[
 (T_1 \wedge \ldots \wedge T_m)\sigma \defAs T_1\sigma \wedge \ldots \wedge T_m\sigma . 
\]
\vspace{\baselineskip}

Alcuni esempi:
\[
 \begin{array}{lcl}
 (y\,childOf\,x\,\wedge\,Male(y))[x \rightarrow Alice] & \defAs & y\,childOf\,Alice\,\wedge\,Male(y)  \\
 (y\,childOf\,x\,\wedge\,Male(y))[x \rightarrow Alice, z \rightarrow Bob] & \defAs & \phantom{y\,childOf\,Alice\,\wedge\,Male(y)}  \\
 (y\,childOf\,x\,\wedge\,Male(y))[x \rightarrow Alice, y \rightarrow Bob] & \defAs & \phantom{Bob\,childOf\,Alice\,\wedge\,Male(Bob)}  \\
 \end{array}
\]
\end{frame}

\begin{frame}
\frametitle{Sostituzioni (2/2)}

L'applicazione di sostituzioni a query congiuntive si definisce come segue.
\vspace{\baselineskip}

Sia $\sigma=[x_1 \rightarrow a_1, \ldots, x_n \rightarrow a_n]$ una sostituzione
e siano $T_1, \ldots, T_m$ formule atomiche. Allora
\[
 (T_1 \wedge \ldots \wedge T_m)\sigma \defAs T_1\sigma \wedge \ldots \wedge T_m\sigma . 
\]
\vspace{\baselineskip}

Alcuni esempi:
\[
 \begin{array}{lcl}
 (y\,childOf\,x\,\wedge\,Male(y))[x \rightarrow Alice] & \defAs & y\,childOf\,Alice\,\wedge\,Male(y)  \\
 (y\,childOf\,x\,\wedge\,Male(y))[x \rightarrow Alice, z \rightarrow Bob] & \defAs & y\,childOf\,Alice\,\wedge\,Male(y)  \\
 (y\,childOf\,x\,\wedge\,Male(y))[x \rightarrow Alice, y \rightarrow Bob] & \defAs & \phantom{Bob\,childOf\,Alice\,\wedge\,Male(Bob)}  \\
 \end{array}
\]
\end{frame}

\begin{frame}
\frametitle{Sostituzioni (2/2)}

L'applicazione di sostituzioni a query congiuntive si definisce come segue.
\vspace{\baselineskip}

Sia $\sigma=[x_1 \rightarrow a_1, \ldots, x_n \rightarrow a_n]$ una sostituzione
e siano $T_1, \ldots, T_m$ formule atomiche. Allora
\[
 (T_1 \wedge \ldots \wedge T_m)\sigma \defAs T_1\sigma \wedge \ldots \wedge T_m\sigma . 
\]
\vspace{\baselineskip}

Alcuni esempi:
\[
 \begin{array}{lcl}
 (y\,childOf\,x\,\wedge\,Male(y))[x \rightarrow Alice] & \defAs & y\,childOf\,Alice\,\wedge\,Male(y)  \\
 (y\,childOf\,x\,\wedge\,Male(y))[x \rightarrow Alice, z \rightarrow Bob] & \defAs & y\,childOf\,Alice\,\wedge\,Male(y)  \\
 (y\,childOf\,x\,\wedge\,Male(y))[x \rightarrow Alice, y \rightarrow Bob] & \defAs & Bob\,childOf\,Alice\,\wedge\,Male(Bob)  \\
 \end{array}
\]
\end{frame}

\begin{frame}
\frametitle{Soluzioni per una Query}
Siano $\sigma=[x_1 \rightarrow a_1, \ldots, x_n \rightarrow a_n]$ una sostituzione,
$Q=T_1 \wedge \ldots \wedge T_m$ una query congiuntiva e $\Ont$ una ontologia.
\vspace{\baselineskip}

$\sigma$ \`e detta essere una \emph{soluzione} per $Q$ rispetto ad $\Ont$ se
e solo se $T_1\sigma, \ldots, T_2\sigma$ compaiono in $\Ont$. 
\vspace{\baselineskip}

Consideriamo l'ontologia $\Ont$ e la query $Q$ definite come segue:
\[
\begin{array}{cl}
  \Ont  =  &  \{Female(Elise), Female(Alice), Male(Bob), \\
  &\phantom{\{}Male(Charlie), Male(Daniel), \\
  &\phantom{\{}Alice\,childOf\,Elise, Charlie\,childOf\,Elise, \\
  &\phantom{\{}Daniel\,childOf\,Alice, Daniel\,childOf\,Bob, \\
  &\phantom{\{}Francis\,childOf\,Charlie \}\\
  &\\
  Q = & y\,childOf\,x\,\wedge\,Male(y) .
 \end{array}
\]
Sia $\sigma_1=[x \rightarrow Alice, y \rightarrow Daniel]$. $\sigma_1$ \`e una soluzione per $Q$
rispetto ad $\Ont$?
\end{frame}

\begin{frame}
\frametitle{Soluzioni per una Query - Esempio 1}
Siano $\sigma=[x_1 \rightarrow a_1, \ldots, x_n \rightarrow a_n]$ una sostituzione,
$Q=T_1 \wedge \ldots \wedge T_m$ una query congiuntiva e $\Ont$ una ontologia.
\vspace{\baselineskip}

$\sigma$ \`e detta essere una \emph{soluzione} per $Q$ rispetto ad $\Ont$ se
e solo se $T_1\sigma, \ldots, T_2\sigma$ compaiono in $\Ont$. 
\vspace{\baselineskip}

Consideriamo l'ontologia $\Ont$ e la query $Q$ (``Chi sono gli individui con almeno un figlio maschio?'') definite come segue:
\[
\begin{array}{cl}
  \Ont  =  &  \{Female(Elise), Female(Alice), Male(Bob), \\
  &\phantom{\{}Male(Charlie), \textbf{Male(Daniel)}, \\
  &\phantom{\{}Alice\,childOf\,Elise, Charlie\,childOf\,Elise, \\
  &\phantom{\{}\textbf{Daniel\,childOf\,Alice}, Daniel\,childOf\,Bob, \\
  &\phantom{\{}Francis\,childOf\,Charlie \}\\
  &\\
  Q = & y\,childOf\,x\,\wedge\,Male(y) .
 \end{array}
\]
Sia $\sigma_1=[x \rightarrow Alice, y \rightarrow Daniel]$. $\sigma_1$ \`e una soluzione per $Q$
rispetto ad $\Ont$? \textbf{SI}.
\[
 Q\sigma_1 = Daniel\,childOf\,Alice\,\wedge\,Male(Daniel) .
\]
\end{frame}

\begin{frame}
\frametitle{Soluzioni per una Query - Esempio 2}
Siano $\sigma=[x_1 \rightarrow a_1, \ldots, x_n \rightarrow a_n]$ una sostituzione,
$Q=T_1 \wedge \ldots \wedge T_m$ una query congiuntiva e $\Ont$ una ontologia.
\vspace{\baselineskip}

$\sigma$ \`e detta essere una \emph{soluzione} per $Q$ rispetto ad $\Ont$ se
e solo se $T_1\sigma, \ldots, T_2\sigma$ compaiono in $\Ont$. 
\vspace{\baselineskip}

Consideriamo l'ontologia $\Ont$ e la query $Q$ definite come segue:
\[
\begin{array}{cl}
  \Ont  =  &  \{Female(Elise), Female(Alice), Male(Bob), \\
  &\phantom{\{}Male(Charlie), Male(Daniel), \\
  &\phantom{\{}Alice\,childOf\,Elise, Charlie\,childOf\,Elise, \\
  &\phantom{\{}Daniel\,childOf\,Alice, Daniel\,childOf\,Bob, \\
  &\phantom{\{}Francis\,childOf\,Charlie \}\\
  &\\
  Q = & y\,childOf\,x\,\wedge\,Male(y) .
 \end{array}
\]
Sia $\sigma_2=[x \rightarrow Alice, y \rightarrow Bob]$. $\sigma_2$ \`e una soluzione per $Q$
rispetto ad $\Ont$?
\end{frame}

\begin{frame}
\frametitle{Soluzioni per una Query - Esempio 2}
Siano $\sigma=[x_1 \rightarrow a_1, \ldots, x_n \rightarrow a_n]$ una sostituzione,
$Q=T_1 \wedge \ldots \wedge T_m$ una query congiuntiva e $\Ont$ una ontologia.
\vspace{\baselineskip}

$\sigma$ \`e detta essere una \emph{soluzione} per $Q$ rispetto ad $\Ont$ se
e solo se $T_1\sigma, \ldots, T_2\sigma$ compaiono in $\Ont$. 
\vspace{\baselineskip}

Consideriamo l'ontologia $\Ont$ e la query $Q$ (``Chi sono gli individui con almeno un figlio maschio?'') definite come segue:
\[
\begin{array}{cl}
  \Ont  =  &  \{Female(Elise), Female(Alice), Male(Bob), \\
  &\phantom{\{}Male(Charlie), Male(Daniel), \\
  &\phantom{\{}Alice\,childOf\,Elise, Charlie\,childOf\,Elise, \\
  &\phantom{\{}Daniel\,childOf\,Alice, Daniel\,childOf\,Bob, \\
  &\phantom{\{}Francis\,childOf\,Charlie \}\\
  &\\
  Q = & y\,childOf\,x\,\wedge\,Male(y) .
 \end{array}
\]
Sia $\sigma_2=[x \rightarrow Alice, y \rightarrow Bob]$. $\sigma_2$ \`e una soluzione per $Q$
rispetto ad $\Ont$? \textbf{NO}.
\[
 Q\sigma_2 = \mathbf{Bob\,childOf\,Alice}\,\wedge\,Male(Bob) .
\]
\end{frame}

\begin{frame}
\frametitle{Soluzioni per una Query - Esempio 3}
Siano $\sigma=[x_1 \rightarrow a_1, \ldots, x_n \rightarrow a_n]$ una sostituzione,
$Q=T_1 \wedge \ldots \wedge T_m$ una query congiuntiva e $\Ont$ una ontologia.
\vspace{\baselineskip}

$\sigma$ \`e detta essere una \emph{soluzione} per $Q$ rispetto ad $\Ont$ se
e solo se $T_1\sigma, \ldots, T_2\sigma$ compaiono in $\Ont$. 
\vspace{\baselineskip}

Consideriamo l'ontologia $\Ont$ e la query $Q$ definite come segue:
\[
\begin{array}{cl}
  \Ont  =  &  \{Female(Elise), Female(Alice), Male(Bob), \\
  &\phantom{\{}Male(Charlie), Male(Daniel), \\
  &\phantom{\{}Alice\,childOf\,Elise, Charlie\,childOf\,Elise, \\
  &\phantom{\{}Daniel\,childOf\,Alice, Daniel\,childOf\,Bob, \\
  &\phantom{\{}Francis\,childOf\,Charlie \}\\
  &\\
  Q = & y\,childOf\,x\,\wedge\,Male(y) .
 \end{array}
\]
Sia $\sigma_3=[x \rightarrow Charlie, y \rightarrow Francis]$. $\sigma_3$ \`e una soluzione per $Q$
rispetto ad $\Ont$?
\end{frame}

\begin{frame}
\frametitle{Soluzioni per una Query - Esempio 3}
Siano $\sigma=[x_1 \rightarrow a_1, \ldots, x_n \rightarrow a_n]$ una sostituzione,
$Q=T_1 \wedge \ldots \wedge T_m$ una query congiuntiva e $\Ont$ una ontologia.
\vspace{\baselineskip}

$\sigma$ \`e detta essere una \emph{soluzione} per $Q$ rispetto ad $\Ont$ se
e solo se $T_1\sigma, \ldots, T_2\sigma$ compaiono in $\Ont$. 
\vspace{\baselineskip}

Consideriamo l'ontologia $\Ont$ e la query $Q$ (``Chi sono gli individui con almeno un figlio maschio?'') definite come segue:
\[
\begin{array}{cl}
  \Ont  =  &  \{Female(Elise), Female(Alice), Male(Bob), \\
  &\phantom{\{}Male(Charlie), Male(Daniel), \\
  &\phantom{\{}Alice\,childOf\,Elise, Charlie\,childOf\,Elise, \\
  &\phantom{\{}Daniel\,childOf\,Alice, Daniel\,childOf\,Bob, \\
  &\phantom{\{}Francis\,childOf\,Charlie \}\\
  &\\
  Q = & y\,childOf\,x\,\wedge\,Male(y) .
 \end{array}
\]
Sia $\sigma_3=[x \rightarrow Charlie, y \rightarrow Francis]$. $\sigma_2$ \`e una soluzione per $Q$
rispetto ad $\Ont$? \textbf{NO}.
\[
 Q\sigma_3 = Francis\,childOf\,Charlie\,\wedge\,\mathbf{Male(Francis)} .
\]
\end{frame}

\begin{frame}
\frametitle{Soluzioni per una Query - Esempio 4}
Siano $\sigma=[x_1 \rightarrow a_1, \ldots, x_n \rightarrow a_n]$ una sostituzione,
$Q=T_1 \wedge \ldots \wedge T_m$ una query congiuntiva e $\Ont$ una ontologia.
\vspace{\baselineskip}

$\sigma$ \`e detta essere una \emph{soluzione} per $Q$ rispetto ad $\Ont$ se
e solo se $T_1\sigma, \ldots, T_2\sigma$ compaiono in $\Ont$. 
\vspace{\baselineskip}

Consideriamo l'ontologia $\Ont$ e la query $Q$ definite come segue:
\[
\begin{array}{cl}
  \Ont  =  &  \{Female(Elise), Female(Alice), Male(Bob), \\
  &\phantom{\{}Male(Charlie), Male(Daniel), \\
  &\phantom{\{}Alice\,childOf\,Elise, Charlie\,childOf\,Elise, \\
  &\phantom{\{}Daniel\,childOf\,Alice, Daniel\,childOf\,Bob, \\
  &\phantom{\{}Francis\,childOf\,Charlie \}\\
  &\\
  Q = & y\,childOf\,x\,\wedge\,Male(y) .
 \end{array}
\]
Sia $\sigma_4=[y \rightarrow Daniel]$. $\sigma_4$ \`e una soluzione per $Q$
rispetto ad $\Ont$?
\phantom{Affinch\`e una sostituzione $\sigma$ sia una soluzione per una query $Q$
(a prescindere dall'ontologia) \`e necessario che in $\sigma$ compaiano
tutte le variabili di $Q$.}
\end{frame}

\begin{frame}
\frametitle{Soluzioni per una Query - Esempio 4}
Siano $\sigma=[x_1 \rightarrow a_1, \ldots, x_n \rightarrow a_n]$ una sostituzione,
$Q=T_1 \wedge \ldots \wedge T_m$ una query congiuntiva e $\Ont$ una ontologia.
\vspace{\baselineskip}

$\sigma$ \`e detta essere una \emph{soluzione} per $Q$ rispetto ad $\Ont$ se
e solo se $T_1\sigma, \ldots, T_2\sigma$ compaiono in $\Ont$. 
\vspace{\baselineskip}

Consideriamo l'ontologia $\Ont$ e la query $Q$ (``Chi sono gli individui con almeno un figlio maschio?'') definite come segue:
\[
\begin{array}{cl}
  \Ont  =  &  \{Female(Elise), Female(Alice), Male(Bob), \\
  &\phantom{\{}Male(Charlie), Male(Daniel), \\
  &\phantom{\{}Alice\,childOf\,Elise, Charlie\,childOf\,Elise, \\
  &\phantom{\{}Daniel\,childOf\,Alice, Daniel\,childOf\,Bob, \\
  &\phantom{\{}Francis\,childOf\,Charlie \}\\
  &\\
  Q = & y\,childOf\,x\,\wedge\,Male(y) .
 \end{array}
\]
Sia $\sigma_4=[y \rightarrow Daniel]$. $\sigma_2$ \`e una soluzione per $Q$
rispetto ad $\Ont$? \textbf{NO}.
\[
 Q\sigma_4 = \mathbf{Daniel\,childOf\,x}\,\wedge\,Male(Daniel) .
\]
Affinch\`e una sostituzione $\sigma$ sia una soluzione per una query $Q$
(a prescindere dall'ontologia) \`e necessario che in $\sigma$ compaiano
tutte le variabili di $Q$.
\end{frame}

\begin{frame}
\frametitle{Soluzioni per una Query - Esempio 5}
Siano $\sigma=[x_1 \rightarrow a_1, \ldots, x_n \rightarrow a_n]$ una sostituzione,
$Q=T_1 \wedge \ldots \wedge T_m$ una query congiuntiva e $\Ont$ una ontologia.
\vspace{\baselineskip}

$\sigma$ \`e detta essere una \emph{soluzione} per $Q$ rispetto ad $\Ont$ se
e solo se $T_1\sigma, \ldots, T_2\sigma$ compaiono in $\Ont$. 
\vspace{\baselineskip}

Consideriamo l'ontologia $\Ont$ e la query $Q$ definite come segue:
\[
\begin{array}{cl}
  \Ont  =  &  \{Female(Elise), Female(Alice), Male(Bob), \\
  &\phantom{\{}Male(Charlie), Male(Daniel), \\
  &\phantom{\{}Alice\,childOf\,Elise, Charlie\,childOf\,Elise, \\
  &\phantom{\{}Daniel\,childOf\,Alice, Daniel\,childOf\,Bob, \\
  &\phantom{\{}Francis\,childOf\,Charlie \}\\
  &\\
  Q = & y\,childOf\,x\,\wedge\,Male(y) .
 \end{array}
\]
Sia $\sigma_5=[x \rightarrow Elise, y \rightarrow Charlie, z \rightarrow Francis]$. $\sigma_5$ \`e una soluzione per $Q$
rispetto ad $\Ont$?
\end{frame}

\begin{frame}
\frametitle{Soluzioni per una Query - Esempio 5}
Siano $\sigma=[x_1 \rightarrow a_1, \ldots, x_n \rightarrow a_n]$ una sostituzione,
$Q=T_1 \wedge \ldots \wedge T_m$ una query congiuntiva e $\Ont$ una ontologia.
\vspace{\baselineskip}

$\sigma$ \`e detta essere una \emph{soluzione} per $Q$ rispetto ad $\Ont$ se
e solo se $T_1\sigma, \ldots, T_2\sigma$ compaiono in $\Ont$. 
\vspace{\baselineskip}

Consideriamo l'ontologia $\Ont$ e la query $Q$ (``Chi sono gli individui con almeno un figlio maschio?'') definite come segue:
\[
\begin{array}{cl}
  \Ont  =  &  \{Female(Elise), Female(Alice), Male(Bob), \\
  &\phantom{\{}\mathbf{Male(Charlie)}, Male(Daniel), \\
  &\phantom{\{}Alice\,childOf\,Elise, \mathbf{Charlie\,childOf\,Elise}, \\
  &\phantom{\{}Daniel\,childOf\,Alice, Daniel\,childOf\,Bob, \\
  &\phantom{\{}Francis\,childOf\,Charlie \}\\
  &\\
  Q = & y\,childOf\,x\,\wedge\,Male(y) .
 \end{array}
\]
Sia $\sigma_5=[x \rightarrow Elise, y \rightarrow Charlie, z \rightarrow Francis]$. $\sigma_5$ \`e una soluzione per $Q$
rispetto ad $\Ont$? \textbf{SI}.
\[
 Q\sigma_5 = Charlie\,childOf\,Elise\,\wedge\,Male(Charlie) .
\]
\end{frame}

\begin{frame}
\frametitle{Soluzioni Minimali per una Query}
Siano $\sigma=[x_1 \rightarrow a_1, \ldots, x_n \rightarrow a_n]$ 
una sostituzione, $Q$ una query congiuntiva e $\Ont$ una ontologia.
\vspace{\baselineskip}

$\sigma$ \`e una \emph{soluzione minimale} per $Q$ rispetto a $\Ont$
se e solo se:
\begin{enumerate}
 \item $\sigma$ \`e una soluzione per $Q$ rispetto ad $\Ont$ e inoltre
 \item tutte le variabili $x_1, \ldots, x_n$ che compaiono in $\sigma$
 compaiono anche in $Q$ (criterio di minimalit\`a).
\end{enumerate}
Consideriamo ad esempio
\[
\begin{array}{cl}
  \Ont  =  &  \{Female(Elise), Female(Alice), Male(Bob), \\
  &\phantom{\{}Male(Charlie), Male(Daniel), \\
  &\phantom{\{}Alice\,childOf\,Elise, Charlie\,childOf\,Elise, \\
  &\phantom{\{}Daniel\,childOf\,Alice, Daniel\,childOf\,Bob, \\
  &\phantom{\{}Francis\,childOf\,Charlie \}\\
  &\\
  Q = & y\,childOf\,x\,\wedge\,Male(y) .
 \end{array}
\]
$\sigma_5=[x \rightarrow Elise, y \rightarrow Charlie, z \rightarrow Francis]$ \`e una 
soluzione minimale per $Q$ rispetto a $\Ont$? \phantom{\textbf{NO}.}
\vspace{\baselineskip}

\phantom{$\sigma_6=[x \rightarrow Elise, y \rightarrow Charlie]$ \`e una 
soluzione minimale per $Q$ rispetto a $\Ont$? \textbf{SI}.}
\end{frame}

\begin{frame}
\frametitle{Soluzioni Minimali per una Query}
Siano $\sigma=[x_1 \rightarrow a_1, \ldots, x_n \rightarrow a_n]$ 
una sostituzione, $Q$ una query congiuntiva e $\Ont$ una ontologia.
\vspace{\baselineskip}

$\sigma$ \`e una \emph{soluzione minimale} per $Q$ rispetto a $\Ont$
se e solo se:
\begin{enumerate}
 \item $\sigma$ \`e una soluzione per $Q$ rispetto ad $\Ont$ e inoltre
 \item tutte le variabili $x_1, \ldots, x_n$ che compaiono in $\sigma$
 compaiono anche in $Q$ (criterio di minimalit\`a).
\end{enumerate}
Consideriamo ad esempio
\[
\begin{array}{cl}
  \Ont  =  &  \{Female(Elise), Female(Alice), Male(Bob), \\
  &\phantom{\{}Male(Charlie), Male(Daniel), \\
  &\phantom{\{}Alice\,childOf\,Elise, Charlie\,childOf\,Elise, \\
  &\phantom{\{}Daniel\,childOf\,Alice, Daniel\,childOf\,Bob, \\
  &\phantom{\{}Francis\,childOf\,Charlie \}\\
  &\\
  Q = & y\,childOf\,x\,\wedge\,Male(y) .
 \end{array}
\]
$\sigma_5=[x \rightarrow Elise, y \rightarrow Charlie, \mathbf{z} \rightarrow Francis]$ \`e una 
soluzione minimale per $Q$ rispetto a $\Ont$? \textbf{NO}.
\vspace{\baselineskip}

$\sigma_6=[x \rightarrow Elise, y \rightarrow Charlie]$ \`e una 
soluzione minimale per $Q$ rispetto a $\Ont$? \phantom{\textbf{SI}}.
\end{frame}

\begin{frame}
\frametitle{Soluzioni Minimali per una Query}
Siano $\sigma=[x_1 \rightarrow a_1, \ldots, x_n \rightarrow a_n]$ 
una sostituzione, $Q$ una query congiuntiva e $\Ont$ una ontologia.
\vspace{\baselineskip}

$\sigma$ \`e una \emph{soluzione minimale} per $Q$ rispetto a $\Ont$
se e solo se:
\begin{enumerate}
 \item $\sigma$ \`e una soluzione per $Q$ rispetto ad $\Ont$ e inoltre
 \item tutte le variabili $x_1, \ldots, x_n$ che compaiono in $\sigma$
 compaiono anche in $Q$ (criterio di minimalit\`a).
\end{enumerate}
Consideriamo ad esempio
\[
\begin{array}{cl}
  \Ont  =  &  \{Female(Elise), Female(Alice), Male(Bob), \\
  &\phantom{\{}Male(Charlie), Male(Daniel), \\
  &\phantom{\{}Alice\,childOf\,Elise, Charlie\,childOf\,Elise, \\
  &\phantom{\{}Daniel\,childOf\,Alice, Daniel\,childOf\,Bob, \\
  &\phantom{\{}Francis\,childOf\,Charlie \}\\
  &\\
  Q = & y\,childOf\,x\,\wedge\,Male(y) .
 \end{array}
\]
$\sigma_5=[x \rightarrow Elise, y \rightarrow Charlie, \mathbf{z} \rightarrow Francis]$ \`e una 
soluzione minimale per $Q$ rispetto a $\Ont$? \textbf{NO}.
\vspace{\baselineskip}

$\sigma_6=[x \rightarrow Elise, y \rightarrow Charlie]$ \`e una 
soluzione minimale per $Q$ rispetto a $\Ont$? \textbf{SI}.
\end{frame}

\begin{frame}
\frametitle{Conjunctive Query Answering}
Il problema del \emph{Conjunctive Query Answering} consiste nel trovare
tutte le soluzioni minimali di una query congiuntiva rispetto ad una ontologia.
\vspace{\baselineskip}

Esse sono sempre in numero finito, infatti:
\begin{itemize}
 \item le variabili che compaiono nelle soluzioni sono esattamente quelle che compaiono nella query,
 \item i nomi di individui che compaiono nelle soluzioni sono un sottoinsieme di quelli che compaiono
 nell'ontologia.
\end{itemize}
\end{frame}

\begin{frame}
\frametitle{Conjunctive Query Answering - Esempio 1}
\begin{center}
``Trova tutti gli individui maschi.'' 
\end{center}
\[
 Q=Male(x)
\]
\vspace{\baselineskip}

\begin{tabular}{lc}
$\begin{array}{cl}
  \Ont  =  &  \{Female(Elise),  Female(Alice), Male(Bob), \\
  &\phantom{\{}Male(Charlie), Male(Daniel), \\
  &\phantom{\{}Alice\,childOf\,Elise, Charlie\,childOf\,Elise, \\
  &\phantom{\{}Daniel\,childOf\,Alice, Daniel\,childOf\,Bob, \\
  &\phantom{\{}Francis\,childOf\,Charlie \}\\
\end{array}$ & 
$\begin{array}{|c|}
  \hline
  x\\
  \hline
  \phantom{Bob}\\
  \phantom{Charlie}\\
  \phantom{Daniel}\\
  \hline
\end{array}$ \\
\end{tabular}
\end{frame}

\begin{frame}
\frametitle{Conjunctive Query Answering - Esempio 1}
\begin{center}
``Trova tutti gli individui maschi.'' 
\end{center}
\[
 Q=Male(x)
\]
\vspace{\baselineskip}

\begin{tabular}{lc}
$\begin{array}{cl}
  \Ont  =  &  \{Female(Elise),  Female(Alice), \mathbf{Male(Bob)}, \\
  &\phantom{\{}Male(Charlie), Male(Daniel), \\
  &\phantom{\{}Alice\,childOf\,Elise, Charlie\,childOf\,Elise, \\
  &\phantom{\{}Daniel\,childOf\,Alice, Daniel\,childOf\,Bob, \\
  &\phantom{\{}Francis\,childOf\,Charlie \}\\
\end{array}$ & 
$\begin{array}{|c|}
  \hline
  x\\
  \hline
  \mathbf{Bob}\\
  \phantom{Charlie}\\
  \phantom{Daniel}\\
  \hline
\end{array}$ \\
\end{tabular}
\end{frame}

\begin{frame}
\frametitle{Conjunctive Query Answering - Esempio 1}
\begin{center}
``Trova tutti gli individui maschi.'' 
\end{center}
\[
 Q=Male(x)
\]
\vspace{\baselineskip}

\begin{tabular}{lc}
$\begin{array}{cl}
  \Ont  =  &  \{Female(Elise),  Female(Alice), Male(Bob), \\
  &\phantom{\{}\mathbf{Male(Charlie)}, Male(Daniel), \\
  &\phantom{\{}Alice\,childOf\,Elise, Charlie\,childOf\,Elise, \\
  &\phantom{\{}Daniel\,childOf\,Alice, Daniel\,childOf\,Bob, \\
  &\phantom{\{}Francis\,childOf\,Charlie \}\\
\end{array}$ & 
$\begin{array}{|c|}
  \hline
  x\\
  \hline
  Bob\\
  \mathbf{Charlie}\\
  \phantom{Daniel}\\
  \hline
\end{array}$ \\
\end{tabular}
\end{frame}

\begin{frame}
\frametitle{Conjunctive Query Answering - Esempio 1}
\begin{center}
``Trova tutti gli individui maschi.'' 
\end{center}
\[
 Q=Male(x)
\]
\vspace{\baselineskip}

\begin{tabular}{lc}
$\begin{array}{cl}
  \Ont  =  &  \{Female(Elise),  Female(Alice), Male(Bob), \\
  &\phantom{\{}Male(Charlie), \mathbf{Male(Daniel)}, \\
  &\phantom{\{}Alice\,childOf\,Elise, Charlie\,childOf\,Elise, \\
  &\phantom{\{}Daniel\,childOf\,Alice, Daniel\,childOf\,Bob, \\
  &\phantom{\{}Francis\,childOf\,Charlie \}\\
\end{array}$ & 
$\begin{array}{|c|}
  \hline
  x\\
  \hline
  Bob\\
  Charlie\\
  \mathbf{Daniel}\\
  \hline
\end{array}$ \\
\end{tabular}
\end{frame}

\begin{frame}
\frametitle{Conjunctive Query Answering - Esempio 2}
\begin{center}
``Chi sono gli individui con almeno un figlio maschio?'' 
\end{center}
\[
 Q=y\,childOf\,x\,\wedge\,Male(y)
\]
\vspace{\baselineskip}

\begin{tabular}{lc}
$\begin{array}{cl}
  \Ont  =  &  \{Female(Elise),  Female(Alice), Male(Bob), \\
  &\phantom{\{}Male(Charlie), Male(Daniel), \\
  &\phantom{\{}Alice\,childOf\,Elise, Charlie\,childOf\,Elise, \\
  &\phantom{\{}Daniel\,childOf\,Alice, Daniel\,childOf\,Bob, \\
  &\phantom{\{}Francis\,childOf\,Charlie \}\\
\end{array}$ & 
$\begin{array}{|c|c|}
  \hline
  x & y\\
  \hline
  \phantom{Elise}&\phantom{Charlie}\\
  \phantom{Alice}&\phantom{Daniel}\\
  \phantom{Bob}&\phantom{Daniel}\\
  \hline
\end{array}$ \\
\end{tabular}
\end{frame}

\begin{frame}
\frametitle{Conjunctive Query Answering - Esempio 2}
\begin{center}
``Chi sono gli individui con almeno un figlio maschio?'' 
\end{center}
\[
 Q=y\,childOf\,x\,\wedge\,Male(y)
\]
\vspace{\baselineskip}

\begin{tabular}{lc}
$\begin{array}{cl}
  \Ont  =  &  \{Female(Elise),  Female(Alice), Male(Bob), \\
  &\phantom{\{}\mathbf{Male(Charlie)}, Male(Daniel), \\
  &\phantom{\{}Alice\,childOf\,Elise, \mathbf{Charlie\,childOf\,Elise}, \\
  &\phantom{\{}Daniel\,childOf\,Alice, Daniel\,childOf\,Bob, \\
  &\phantom{\{}Francis\,childOf\,Charlie \}\\
\end{array}$ & 
$\begin{array}{|c|c|}
  \hline
  x & y\\
  \hline
  \mathbf{Elise}&\mathbf{Charlie}\\
  \phantom{Alice}&\phantom{Daniel}\\
  \phantom{Bob}&\phantom{Daniel}\\
  \hline
\end{array}$ \\
\end{tabular}
\end{frame}

\begin{frame}
\frametitle{Conjunctive Query Answering - Esempio 2}
\begin{center}
``Chi sono gli individui con almeno un figlio maschio?'' 
\end{center}
\[
 Q=y\,childOf\,x\,\wedge\,Male(y)
\]
\vspace{\baselineskip}

\begin{tabular}{lc}
$\begin{array}{cl}
  \Ont  =  &  \{Female(Elise),  Female(Alice), Male(Bob), \\
  &\phantom{\{}Male(Charlie), \mathbf{Male(Daniel)}, \\
  &\phantom{\{}Alice\,childOf\,Elise, Charlie\,childOf\,Elise, \\
  &\phantom{\{}\mathbf{Daniel\,childOf\,Alice}, Daniel\,childOf\,Bob, \\
  &\phantom{\{}Francis\,childOf\,Charlie \}\\
\end{array}$ & 
$\begin{array}{|c|c|}
  \hline
  x & y\\
  \hline
  Elise&Charlie\\
  \mathbf{Alice}&\mathbf{Daniel}\\
  \phantom{Bob}&\phantom{Daniel}\\
  \hline
\end{array}$ \\
\end{tabular}
\end{frame}

\begin{frame}
\frametitle{Conjunctive Query Answering - Esempio 2}
\begin{center}
``Chi sono gli individui con almeno un figlio maschio?'' 
\end{center}
\[
 Q=y\,childOf\,x\,\wedge\,Male(y)
\]
\vspace{\baselineskip}

\begin{tabular}{lc}
$\begin{array}{cl}
  \Ont  =  &  \{Female(Elise),  Female(Alice), Male(Bob), \\
  &\phantom{\{}Male(Charlie), \mathbf{Male(Daniel)}, \\
  &\phantom{\{}Alice\,childOf\,Elise, Charlie\,childOf\,Elise, \\
  &\phantom{\{}Daniel\,childOf\,Alice, \mathbf{Daniel\,childOf\,Bob}, \\
  &\phantom{\{}Francis\,childOf\,Charlie \}\\
\end{array}$ & 
$\begin{array}{|c|c|}
  \hline
  x & y\\
  \hline
  Elise&Charlie\\
  Alice&Daniel\\
  \mathbf{Bob}&\mathbf{Daniel}\\
  \hline
\end{array}$ \\
\end{tabular}
\end{frame}

\begin{frame}
\frametitle{Conjunctive Query Answering - Esempio 3}
\begin{center}
``Chi sono i figli di $Elise$?''
\end{center}
\[
 Q=x\,childOf\,Elise
\]
\vspace{\baselineskip}

\begin{tabular}{lc}
$\begin{array}{cl}
  \Ont  =  &  \{Female(Elise),  Female(Alice), Male(Bob), \\
  &\phantom{\{}Male(Charlie), Male(Daniel), \\
  &\phantom{\{}Alice\,childOf\,Elise, Charlie\,childOf\,Elise, \\
  &\phantom{\{}Daniel\,childOf\,Alice, Daniel\,childOf\,Bob, \\
  &\phantom{\{}Francis\,childOf\,Charlie \}\\
\end{array}$ & 
$\begin{array}{|c|}
  \hline
  x\\
  \hline
  \phantom{Alice}\\
  \phantom{Charlie}\\
  \hline
\end{array}$ \\
\end{tabular}
\end{frame}

\begin{frame}
\frametitle{Conjunctive Query Answering - Esempio 3}
\begin{center}
``Chi sono i figli di $Elise$?'' 
\end{center}
\[
 Q=x\,childOf\,Elise
\]
\vspace{\baselineskip}

\begin{tabular}{lc}
$\begin{array}{cl}
  \Ont  =  &  \{Female(Elise),  Female(Alice), Male(Bob), \\
  &\phantom{\{}Male(Charlie), Male(Daniel), \\
  &\phantom{\{}\mathbf{Alice\,childOf\,Elise}, Charlie\,childOf\,Elise, \\
  &\phantom{\{}Daniel\,childOf\,Alice, Daniel\,childOf\,Bob, \\
  &\phantom{\{}Francis\,childOf\,Charlie \}\\
\end{array}$ & 
$\begin{array}{|c|}
  \hline
  x\\
  \hline
  \mathbf{Alice}\\
  \phantom{Charlie}\\
  \hline
\end{array}$ \\
\end{tabular}
\end{frame}

\begin{frame}
\frametitle{Conjunctive Query Answering - Esempio 3}
\begin{center}
``Chi sono i figli di $Elise$?''
\end{center}
\[
 Q=x\,childOf\,Elise
\]
\vspace{\baselineskip}

\begin{tabular}{lc}
$\begin{array}{cl}
  \Ont  =  &  \{Female(Elise),  Female(Alice), Male(Bob), \\
  &\phantom{\{}Male(Charlie), Male(Daniel), \\
  &\phantom{\{}Alice\,childOf\,Elise, \mathbf{Charlie\,childOf\,Elise}, \\
  &\phantom{\{}Daniel\,childOf\,Alice, Daniel\,childOf\,Bob, \\
  &\phantom{\{}Francis\,childOf\,Charlie \}\\
\end{array}$ & 
$\begin{array}{|c|}
  \hline
  x\\
  \hline
  Alice\\
  \mathbf{Charlie}\\
  \hline
\end{array}$ \\
\end{tabular}
\end{frame}

\begin{frame}
\frametitle{Conjunctive Query Answering - Esempio 4}
\begin{center}
``Chi sono gli individui con almeno un figlio maschio ed una femmina?''
\end{center}
\[
 \phantom{Q=y\,childOf\,x\,\wedge\,z\,childOf\,x\,\wedge\,Male(y)\,\wedge\,Female(z)}
\]
\vspace{\baselineskip}

\begin{tabular}{lc}
$\begin{array}{cl}
  \Ont  =  &  \{Female(Elise),  Female(Alice), Male(Bob), \\
  &\phantom{\{}Male(Charlie), Male(Daniel), \\
  &\phantom{\{}Alice\,childOf\,Elise, Charlie\,childOf\,Elise, \\
  &\phantom{\{}Daniel\,childOf\,Alice, Daniel\,childOf\,Bob, \\
  &\phantom{\{}Francis\,childOf\,Charlie \}\\
\end{array}$ & 
$\begin{array}{|c|c|c|}
  \hline
  x&y&z\\
  \hline
  \phantom{Elise}&\phantom{Charlie}&\phantom{Alice}\\
  \hline
\end{array}$ \\
\end{tabular}
\end{frame}

\begin{frame}
\frametitle{Conjunctive Query Answering - Esempio 4}
\begin{center}
``Chi sono gli individui con almeno un figlio maschio ed una femmina?''
\end{center}
\[
 Q=y\,childOf\,x\,\wedge\,z\,childOf\,x\,\wedge\,Male(y)\,\wedge\,Female(z)
\]
\vspace{\baselineskip}

\begin{tabular}{lc}
$\begin{array}{cl}
  \Ont  =  &  \{Female(Elise),  Female(Alice), Male(Bob), \\
  &\phantom{\{}Male(Charlie), Male(Daniel), \\
  &\phantom{\{}Alice\,childOf\,Elise, Charlie\,childOf\,Elise, \\
  &\phantom{\{}Daniel\,childOf\,Alice, Daniel\,childOf\,Bob, \\
  &\phantom{\{}Francis\,childOf\,Charlie \}\\
\end{array}$ & 
$\begin{array}{|c|c|c|}
  \hline
  x&y&z\\
  \hline
  \phantom{Elise}&\phantom{Charlie}&\phantom{Alice}\\
  \hline
\end{array}$ \\
\end{tabular}
\end{frame}

\begin{frame}
\frametitle{Conjunctive Query Answering - Esempio 4}
\begin{center}
``Chi sono gli individui con almeno un figlio maschio ed una femmina?''
\end{center}
\[
 Q=y\,childOf\,x\,\wedge\,z\,childOf\,x\,\wedge\,Male(y)\,\wedge\,Female(z)
\]
\vspace{\baselineskip}

\begin{tabular}{lc}
$\begin{array}{cl}
  \Ont  =  &  \{Female(Elise), \mathbf{Female(Alice)}, Male(Bob), \\
  &\phantom{\{}\mathbf{Male(Charlie)}, Male(Daniel), \\
  &\phantom{\{}\mathbf{Alice\,childOf\,Elise}, \mathbf{Charlie\,childOf\,Elise}, \\
  &\phantom{\{}Daniel\,childOf\,Alice, Daniel\,childOf\,Bob, \\
  &\phantom{\{}Francis\,childOf\,Charlie \}\\
\end{array}$ & 
$\begin{array}{|c|c|c|}
  \hline
  x&y&z\\
  \hline
  Elise&Charlie&Alice\\
  \hline
\end{array}$ \\
\end{tabular}
\end{frame}

\begin{frame}
\frametitle{Conjunctive Query Answering - Esempio 5}
\begin{center}
``Chi sono gli individui maschi con almeno una figlia femmina?''
\end{center}
\[
 \phantom{Q=Male(x)\,\wedge\,y\,childOf\,x\,\wedge\,Female(y)}
\]
\vspace{\baselineskip}

\begin{tabular}{lc}
$\begin{array}{cl}
  \Ont  =  &  \{Female(Elise), Female(Alice), Male(Bob), \\
  &\phantom{\{}Male(Charlie), Male(Daniel), \\
  &\phantom{\{}Alice\,childOf\,Elise, Charlie\,childOf\,Elise, \\
  &\phantom{\{}Daniel\,childOf\,Alice, Daniel\,childOf\,Bob, \\
  &\phantom{\{}Francis\,childOf\,Charlie \}\\
\end{array}$ & 
$\begin{array}{|c|c|}
  \hline
  x&y\\
  \hline
  &\\
  &\\
  \hline
\end{array}$\\
\end{tabular}
\end{frame}

\begin{frame}
\frametitle{Conjunctive Query Answering - Esempio 5}
\begin{center}
``Chi sono gli individui maschi con almeno una figlia femmina?''
\end{center}
\[
 Q=Male(x)\,\wedge\,y\,childOf\,x\,\wedge\,Female(y)
\]
\vspace{\baselineskip}

\begin{tabular}{lc}
$\begin{array}{cl}
  \Ont  =  &  \{Female(Elise), Female(Alice), Male(Bob), \\
  &\phantom{\{}Male(Charlie), Male(Daniel), \\
  &\phantom{\{}Alice\,childOf\,Elise, Charlie\,childOf\,Elise, \\
  &\phantom{\{}Daniel\,childOf\,Alice, Daniel\,childOf\,Bob, \\
  &\phantom{\{}Francis\,childOf\,Charlie \}\\
\end{array}$ & 
$\begin{array}{|c|c|}
  \hline
  x&y\\
  \hline
  &\\
  &\\
  \hline
\end{array}$\\
\end{tabular}
\end{frame}

\begin{frame}
\frametitle{Conjunctive Query Answering - Esempio 5}
\begin{center}
``Chi sono gli individui maschi con almeno una figlia femmina?''
\end{center}
\[
 Q=Male(x)\,\wedge\,y\,childOf\,x\,\wedge\,Female(y)
\]
\vspace{\baselineskip}

\begin{tabular}{lc}
$\begin{array}{cl}
  \Ont  =  &  \{Female(Elise), Female(Alice), Male(Bob), \\
  &\phantom{\{}Male(Charlie), Male(Daniel), \\
  &\phantom{\{}Alice\,childOf\,Elise, Charlie\,childOf\,Elise, \\
  &\phantom{\{}Daniel\,childOf\,Alice, Daniel\,childOf\,Bob, \\
  &\phantom{\{}Francis\,childOf\,Charlie \}\\
\end{array}$ & Nessuna Soluzione\\
\end{tabular}
\end{frame}

\section{Vocabolari}
\begin{frame}
\frametitle{Vocabolari}
Classi e propriet\`a vengono raggruppati in \emph{vocabolari}
che trattano specifici domini di conoscenza (eg. organizzazioni, 
pubblica amministrazione, biologia, commercio, etc.). 
\vspace{\baselineskip}

Un vocabolario
pu\`o contenere anche alcuni vincoli sulle classi e le propriet\`a 
del vocabolario stesso.
\vspace{\baselineskip}

\end{frame}

\begin{frame}
\frametitle{Definizione di Vocabolario}
Una definizione di vocabolario pu\`o essere la seguente:
\[
 V = (C, P, \Omega)
\]
dove
\begin{enumerate}
 \item $C$ \`e un sottoinsieme finito di $\CNames$,
 \item $P$ \`e un sottoinsieme finito di $\PNames$,
 \item $\Omega$ \`e un insieme finito di vincoli che coinvolgano solo
 nomi di classi in $C$ e nomi di propriet\`a in $P$.
\end{enumerate}
\end{frame}

\begin{frame}
\frametitle{Vocabolari condivisi}

L'utilizzo di vocabolari condivisi (ben noti) favorisce la 
scalabilit\`a orizzontale delle applicazioni. 
\vspace{\baselineskip}

Ad esempio, una 
applicazione sviluppata sull'ontologia di un comune che utilizzi 
i vocabolari standard per le pubbliche amministrazioni (vedi le \emph{Linee 
Guida per la Valorizzazione del Patrimonio Informativo Pubblico} dell'\emph{Agenzia per l'Italia Digitale}) 
pu\`o essere estesa senza sforzi aggiuntivi per utilizzare i dati provenienti
dalle ontologie di tutti i comuni.
\end{frame}


\begin{frame}
\frametitle{Il Vocabolario FOAF}
Uno dei primi e pi\`u utilizzati vocabolari definiti nell'ambito
del Web semantico \`e \emph{Friend OF A Friend} (\emph{FOAF}, vedi \url{http://foaf-project.org}).
\vspace{\baselineskip}


\begin{quote}
FOAF is a project devoted to linking people and information using the Web. 
\end{quote}

In questa sede ci limiteremo solo alla parte \emph{Core}.
\begin{quote}
\textbf{Core} - These classes and properties form the core of FOAF. They
describe characteristics of people and social groups that are
independent of time and technology; as such they can be used to
describe basic information about people in present day, historical,
cultural heritage and digital library contexts. In addition to various
characteristics of people, FOAF defines classes for Project,
Organization and Group as other kinds of agent. Related work: 
\end{quote}
(tratto da \emph{FOAF Vocabulary Specification 0.99},
Namespace Document 14 January 2014, Paddington Edition, \url{http://xmlns.com/foaf/spec/})
\end{frame}

\newcommand{\foafcore}{\mathtt{FOAFCore}}
\begin{frame}
\frametitle{FOAF Core}
Il vocabolario \emph{Foaf Core} \`e definito come segue:
\[
 \begin{array}{ccl}
 \foafcore & \defAs & (C_{foaf}, P_{foaf}, \Omega_{foaf}) \\
 &&\\
 C_{foaf} & \defAs & \{Agent, Person, Project, Organization, Group, Document, Image\}\\
 &&\\
 P_{foaf} & \defAs & \{name, title, img, depiction, depicts, familyName, givenName,\\
 && \phantom{\}}, based\_near, age, made, maker, primaryTopic, primaryTopicOf, member \}\\
 \Omega_{foaf} & \defAs & \{ Person \Issub Agent, Group \Issub Agent, Organization \Issub Agent, \\
 && \phantom{\{} Image \Issub Document, \dom(title) \Issub Document, \\
 && \phantom{\{}\range(depiction) \Issub Image, img \Issub depiction, \dom(img) \Issub Person,\\
 && \phantom{\{}\dom(knows) \Issub Person, \range(knows) \Issub Person, \ldots \}\\
 \end{array}
\]
\end{frame}

\begin{frame}
\frametitle{Descrizioni Intuitive degli Elementi dei Vocabolari}
Le classi e le propriet\`a di un vocabolario vengono spesso fornite
di una descrizione intuitiva nel documento che descrive il vocabolario.
Ad esempio, le classi $Agent$ e $Person$ vengono descritte come segue 
in \url{http://xmlns.com/foaf/spec/}
\begin{quote}
\textbf{Agent} - The Agent class is the class of agents; things that do stuff. A well known 
sub-class is Person, representing people. Other kinds of agents include 
Organization and Group.

The Agent class is useful in a few places in FOAF where Person would have 
been overly specific. For example, the IM chat ID properties such as 
jabberID are typically associated with people, but sometimes belong to 
software bots.  
\end{quote}

\begin{quote}
\textbf{Person} - The Person class represents people. Something is a Person if it is a person.
We don't nitpic about whether they're alive, dead, real, or imaginary. 
The Person class is a sub-class of the Agent class, since all people are 
considered 'agents' in FOAF.  
\end{quote}
\end{frame}

\begin{frame}
\frametitle{Descrizioni Rigorose degli Elementi dei Vocabolari}
Tuttavia, gi\`a nei vincoli di un vocabolario si trovano indicazioni importanti
sulla \emph{semantica} dei nomi di classe e di propriet\`a del vocabolario stesso.
\[
\begin{array}{l}
Person \Issub Agent\\
\range(depiction) \Issub Image\\
img \Issub depiction, \\
\dom(img) \Issub Person,\\
\dom(knows) \Issub Person,\\
\range(knows) \Issub Person,\\
\ldots
\end{array}
\]
\end{frame}

\begin{frame}
\frametitle{Vocabolari Compositi}

\`E possibile costruire un vocabolario \emph{estendendone} un'altro.
Ad esempio, il vocabolario \emph{Organization Ontology} (in breve \emph{ORG}, vedi \url{http://www.w3.org/TR/vocab-org/})
estende FOAF con classi e propriet\`a, per modellare in dettaglio 
le strutture organizzative di aziende, associazioni e tutte le forme 
di organizzazioni.
\vspace{\baselineskip}

Formalmente, dati due vocabolari
\[
 \begin{array}{lcl}
 V & \defAs & (C, P, \Omega) \\ 
 V' & \defAs & (C', P', \Omega')
 \end{array}
\]
si dice che $V'$ \emph{importa} (o \emph{estende}) $V$ se e 
solo se
\[
 \begin{array}{rcl}
 C & \subseteq & C' \\
 P & \subseteq & P' \\
 \Omega & \subseteq & \Omega'  
\end{array} 
\]
\end{frame}

\newcommand{\org}{\mathtt{ORG}}

\begin{frame}
\frametitle{Il Vocabolario Organization Ontology}
\emph{Organization Ontology},in breve \emph{ORG},\footnote{\url{http://www.w3.org/TR/vocab-org/}}
estende FOAF con classi e propriet\`a, per modellare in dettaglio 
le strutture organizzative di aziende, associazioni e tutte le forme 
di organizzazioni.
\vspace{\baselineskip}

\uncover<2->{
Il vocabolario \`e disponibile alle URL
\begin{center}
  \url{http://www.w3.org/ns/org.rdf}
\end{center}

\begin{center}
  \url{http://www.w3.org/ns/org.ttl}
\end{center}
}

\uncover<3->{
Il namespace del vocabolario ORG \`e
\begin{center}
  \url{org} : \url{http://www.w3.org/ns/org\#}
\end{center}
}

\[
 \begin{array}{ccl}
 \org & \defAs & (C_{org}, P_{org}, \Omega_{org}) \\
 &&\\
 C_{org} & \defAs & C_{foaf} \cup \{FormalOrganization, OrganizationalUnit,  Site \\
 && \phantom{C_{foaf} \cup \{}Post, Role, \ldots \}\\
 &&\\
 P_{org} & \defAs & P_{foaf} \cup \{ basedAt, classification, hasPost, hasPrimarySite, \\
 &&\phantom{P_{foaf} \cup \{}hasRegisteredSite, hasSite, hasSubOrganization, \\
 &&\phantom{P_{foaf} \cup \{}hasUnit, headOf, heldBy, holds, location, postIn,\\
 &&\phantom{P_{foaf} \cup \{}purpose, role, siteAddress, siteOf, unitOf, \\
 &&\phantom{P_{foaf} \cup \{}subOrganizationOf,  transitiveSubOrganizationOf, \ldots \}\\
 &&\phantom{P_{foaf} \cup \{}  \\
 &&\\
 \Omega_{org} & \defAs & \Omega_{foaf} \cup \{ FormalOrganization \Issub Organization, \\
 &&\phantom{\Omega_{foaf} \cup \{}OrganizationalUnit \Issub Organization,\\
 &&\phantom{\Omega_{foaf} \cup \{}headOf \Issub member, \dom(hasUnit)\Issub Organization, \\
 &&\phantom{\Omega_{foaf} \cup \{}\range(hasUnit) \Issub OrganizationalUnit, \ldots \}
 \end{array}
\]
\phantom{NOTA: I vincoli semantici esplicitano la relazione tra i due vocabolari.}
\end{frame}

\begin{frame}
\frametitle{Il Vocabolario Organization Ontology}
Ad esempio, il vocabolario \emph{Organization Ontology} (in breve \emph{ORG}, vedi \url{http://www.w3.org/TR/vocab-org/})
estende FOAF con classi e propriet\`a, per modellare in dettaglio 
le strutture organizzative di aziende, associazioni e tutte le forme 
di organizzazioni.

\[
 \begin{array}{ccl}
 \org & \defAs & (C_{org}, P_{org}, \Omega_{org}) \\
 &&\\
 C_{org} & \defAs & C_{foaf} \cup \{FormalOrganization, OrganizationalUnit,  Site \\
 && \phantom{C_{foaf} \cup \{}Post, Role, \ldots \}\\
 &&\\
 P_{org} & \defAs & P_{foaf} \cup \{ basedAt, classification, hasPost, hasPrimarySite, \\
 &&\phantom{P_{foaf} \cup \{}hasRegisteredSite, hasSite, hasSubOrganization, \\
 &&\phantom{P_{foaf} \cup \{}hasUnit, headOf, heldBy, holds, location, postIn,\\
 &&\phantom{P_{foaf} \cup \{}purpose, role, siteAddress, siteOf, unitOf, \\ 
 &&\phantom{P_{foaf} \cup \{}subOrganizationOf,  transitiveSubOrganizationOf, \ldots \}\\ 
 &&\phantom{P_{foaf} \cup \{}  \\
 &&\\
 \Omega_{org} & \defAs & \Omega_{foaf} \cup \{ \mathbf{FormalOrganization \Issub Organization,} \\
 &&\phantom{\Omega_{foaf} \cup \{}\mathbf{OrganizationalUnit \Issub Organization,}\\
 &&\phantom{\Omega_{foaf} \cup \{}headOf \Issub memberOf, \dom(hasUnit)\Issub Organization, \\ 
 &&\phantom{\Omega_{foaf} \cup \{}\range(hasUnit) \Issub OrganizationalUnit, \ldots \}
 \end{array}
\]
NOTA: I vincoli semantici esplicitano la relazione tra i due vocabolari.
\end{frame}

\begin{frame}
\frametitle{Vocabolari e Ontologie - Struttura del Comune di Catania}
\`E buona norma definire ontologie facendo riferimento a vocabolari
ben noti.
\vspace{\baselineskip}

Definiamo ad esempio una ontologia per descrivere la struttura
organizzativa del \emph{comune di Catania} (vedi \url{http://www.dmi.unict.it/~longo/comunect/})
a partire dal vocabolario ORG.
\vspace{\baselineskip}

Il sindaco del comune di Catania è \emph{Enzo Bianco}
\[
\begin{array}{rl}
\Ont_{CT} = & \{ EnzoBianco\,headOf\,ComuneCT \\
\end{array} 
\]
\end{frame}

\begin{frame}
\frametitle{Vocabolari e Ontologie - Struttura del Comune di Catania}
\`E buona norma definire ontologie facendo riferimento a vocabolari
ben noti.
\vspace{\baselineskip}

Definiamo ad esempio una ontologia per descrivere la struttura
organizzativa del \emph{comune di Catania} (vedi \url{http://www.dmi.unict.it/~longo/comunect/})
a partire dal vocabolario ORG.
\vspace{\baselineskip}

Fanno direttamente capo al comune il \emph{gabinetto del sindaco}, la \emph{polizia municipale}
e la \emph{direzione affari legali}.
\[
\begin{array}{rl}
\Ont_{CT} = \Sigma_{org}\,\cup & \{ EnzoBianco\,headOf\,ComuneCT, \\
& ComuneCT\,hasUnit\,GabinettoDelSindaco,\\
& ComuneCT\,hasUnit\,PoliziaMunicipale,\\
& ComuneCT\,hasUnit\,AffariLegali,
\end{array} 
\]
\end{frame}

\begin{frame}
\frametitle{Vocabolari e Ontologie - Struttura del Comune di Catania}
\`E buona norma definire ontologie facendo riferimento a vocabolari
ben noti.
\vspace{\baselineskip}

Definiamo ad esempio una ontologia per descrivere la struttura
organizzativa del \emph{comune di Catania} (vedi \url{http://www.dmi.unict.it/~longo/comunect/})
a partire dal vocabolario ORG.
\vspace{\baselineskip}

Il comune comprende inoltre diverse \emph{direzioni}, raggruppate in due macro-aree. 
\[
\begin{array}{rll}
\Ont_{CT} = \Sigma_{org}\,\cup & \{ EnzoBianco\,headOf\,ComuneCT, & \\
& ComuneCT\,hasUnit\,GabinettoDelSindaco, & \\
& ComuneCT\,hasUnit\,PoliziaMunicipale, & \\
& ComuneCT\,hasUnit\,AffariLegali, & \\
& ComuneCT\,hasUnit\,Area1, & \\
& ComuneCT\,hasUnit\,Area2, & \\
\end{array} 
\]
\end{frame}

\begin{frame}
\frametitle{Vocabolari e Ontologie - Struttura del Comune di Catania}
\`E buona norma definire ontologie facendo riferimento a vocabolari
ben noti.
\vspace{\baselineskip}

Definiamo ad esempio una ontologia per descrivere la struttura
organizzativa del \emph{comune di Catania} (vedi \url{http://www.dmi.unict.it/~longo/comunect/})
a partire dal vocabolario ORG.
\vspace{\baselineskip}

Il comune comprende inoltre diverse \emph{direzioni}, raggruppate in due macro-aree. 
\[
\begin{array}{rll}
\Ont_{CT} = \Sigma_{org}\,\cup & \{ EnzoBianco\,headOf\,ComuneCT, & \\
& ComuneCT\,hasUnit\,GabinettoDelSindaco, & \\
& ComuneCT\,hasUnit\,PoliziaMunicipale, & \\
& ComuneCT\,hasUnit\,AffariLegali, & \\
& ComuneCT\,hasUnit\,Area1, & \\
& ComuneCT\,hasUnit\,Area2, & \\
& Area1\,hasUnit\,RisorseUmane,\\
& Area1\,hasUnit\,Istruzione,\\
& Area1\,hasUnit\,Famiglia, \\
& Area1\,hasUnit\,Ecologia,\\
&Area1\,hasUnit\,Manutenzione,\\
&Area1\,hasUnit\,LavoriPubblici,\\
\end{array} 
\]
\end{frame}

% \begin{frame}
% \frametitle{Vocabolari e Ontologie - Struttura del Comune di Catania}
% \`E buona norma definire ontologie facendo riferimento a vocabolari
% ben noti.
% \vspace{\baselineskip}
% 
% Definiamo ad esempio una ontologia per descrivere la struttura
% organizzativa del \emph{comune di Catania} (vedi \url{http://www.dmi.unict.it/~longo/comunect/})
% a partire dal vocabolario ORG.
% \vspace{\baselineskip}
% 
% Il comune comprende inoltre diverse \emph{direzioni}, raggruppate in due macro-aree. 
% \[
% \begin{array}{rll}
% \Ont_{CT} = \Sigma_{org}\,\cup & \{ EnzoBianco\,headOf\,ComuneCT, & Area2\,hasUnit\,Patrimonio\\
% & ComuneCT\,hasUnit\,GabinettoDelSindaco, & Area2\,hasUnit\,ServiziDemografici\\
% & ComuneCT\,hasUnit\,PoliziaMunicipale, & Area2\,hasUnit\,Turismo\\
% & ComuneCT\,hasUnit\,AffariLegali, & Area2\,hasUnit\,Urbanistica\\
% & ComuneCT\,hasUnit\,Area1, & Area2\,hasUnit\,AttivitaProduttive\\
% & ComuneCT\,hasUnit\,Area2, & Area2\,hasUnit\,Economato \}\\
% & Area1\,hasUnit\,RisorseUmane,\\
% & Area1\,hasUnit\,Istruzione,\\
% & Area1\,hasUnit\,Famiglia, \\
% & Area1\,hasUnit\,Ecologia,\\
% &Area1\,hasUnit\,Manutenzione,\\
% &Area1\,hasUnit\,LavoriPubblici,\\
% \end{array} 
% \]
% \end{frame}

\begin{frame}
\frametitle{Vocabolari e Ontologie - Struttura del Comune di Catania}
\`E buona norma definire ontologie facendo riferimento a vocabolari
ben noti.
\vspace{\baselineskip}

Definiamo ad esempio una ontologia per descrivere la struttura
organizzativa del \emph{comune di Catania} (vedi \url{http://www.dmi.unict.it/~longo/comunect/})
a partire dal vocabolario ORG.
\vspace{\baselineskip}

Il comune comprende inoltre diverse \emph{direzioni}, raggruppate in due macro-aree. 
\[
\begin{small}
\begin{array}{rll}
\Ont_{CT} = \Sigma_{org}\,\cup & \{ EnzoBianco\,headOf\,ComuneCT, & \\
& ComuneCT\,hasUnit\,GabinettoDelSindaco, & \\
& ComuneCT\,hasUnit\,PoliziaMunicipale, & \\
& ComuneCT\,hasUnit\,AffariLegali, & \\
& ComuneCT\,hasUnit\,Area1, & \\
& ComuneCT\,hasUnit\,Area2, & \\
& Area1\,hasUnit\,RisorseUmane,\\
& Area1\,hasUnit\,Istruzione,\\
& Area1\,hasUnit\,Famiglia, \\
& Area1\,hasUnit\,Ecologia,\\
&Area1\,hasUnit\,Manutenzione,\\
&Area1\,hasUnit\,LavoriPubblici,\\
&Area2\,hasUnit\,Patrimonio,\\
&Area2\,hasUnit\,ServiziDemografici,\\
&Area2\,hasUnit\,Turismo,\\
&Area2\,hasUnit\,Urbanistica,\\
&Area2\,hasUnit\,AttivitaProduttive,\\
&Area2\,hasUnit\,Economato \}
\end{array} 
\end{small}
\]
\end{frame}




% 
% Ad esempio, \emph{Organization Ontology} (in breve \emph{org}, vedi \url{http://www.w3.org/TR/vocab-org/})
% fornisce gli strumenti, in termini di classi e propriet\`a, per modellare
% le strutture organizzative di aziende, associazioni e tutte le forme 
% di organizzazioni.
% \vspace{\baselineskip}
% 
% \emph{Classi in ORG}: $ChangeEvent$, $FormalOrganization$, $Membership$, $OrganizationalCollaboration$,
% $OrganizationalUnit$, $Organization$, $Post$, $Role$, $Site$.
% \vspace{\baselineskip}
% 
% \emph{Propriet\`a in ORG}: $basedAt$, $changedBy$, $classification$, $hasMember$, $hasMembership$, $hasPost$,
% $hasPrimarySite$, $hasRegisteredSite$, $hasSite$, $hasSubOrganization$, $hasUnit$, $headOf$, $heldBy$, $holds$,
% $identifier$, $linkedTo$, $location$, $memberDuring$, $memberOf$, $member$, $organization$, 
% $originalOrganization$, $postIn$, $purpose$, $remuneration$, $reportsTo$, $resultedFrom$, $resultingOrganization$,
% $role$, $roleProperty$, $siteAddress$, $siteOf$, $subOrganizationOf$, $transitiveSubOrganizationOf$, $unitOf$.

\end{document}
