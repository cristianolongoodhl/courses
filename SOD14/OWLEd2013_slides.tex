\documentclass[8pt]{beamer}
\usepackage[nobglogo]{beamerthemedmi-owled}
\usepackage{graphicx}
\usepackage{dl}
\usepackage{mls}
\usepackage{xspace}
\usepackage{amssymb}
\usepackage{amsfonts}
\usepackage{amsthm}
\usepackage{amsmath}
\usepackage{enumerate}
\usepackage{color}

\mode<presentation>
{
  \usetheme{dmi-owled}
  %\usetheme{Warsaw}
  % or ...

  \setbeamercovered{transparent}
  % or whatever (possibly just delete it)
}


\usepackage[english]{babel}
% or whatever

\usepackage[latin1]{inputenc}
% or whatever

%TERMS

\newcommand{\Persons}{\mathtt{Person}}
\newcommand{\relative}{\mathtt{relativeOf}}
\newcommand{\partner}{\mathtt{partnerOf}}
\newcommand{\child}{\mathtt{childOf}}
\newcommand{\parent}{\mathtt{parentOf}} 
\newcommand{\ancestor}{\mathtt{ancestorOf}} 
\newcommand{\descendant}{\mathtt{descendantOf}}

%INDIVIDUALS
\newcommand{\Alice}{A}
\newcommand{\Bob}{B}
\newcommand{\Charlie}{C}

%KNOWLEDGE BASES
\newcommand{\KBsets}{\mathcal{K}_{\mathcal{L}}}
\newcommand{\KBlcons}{\widehat{\mathcal{K}}_{\mathcal{L}}}
\newcommand{\KBsetsq}{\mathcal{K}^{q}_\mathcal{L}}
\newcommand{\KBdl}{\mathcal{K}}
\newcommand{\KBzerozero}{\mathcal{K}_{\mathcal{S},0}}
\newcommand{\KBzeroone}{\mathcal{K}_{\mathcal{S},1}}
\newcommand{\KBzero}{\mathcal{K}_{\mathcal{S}}}
\newcommand{\KBtra}{\mathcal{K}'_{\mathcal{S}}}
\newcommand{\KBirr}{\mathcal{K}''_{\mathcal{S}}}
\newcommand{\KBq}{\mathcal{K}^{q}_{\mathcal{S}}}
\newcommand{\KBel}{\mathcal{K_E}}
\newcommand{\Kcand}{\mathcal{O}}

%ADDITIONALS
% Queries
\newcommand{\TestCases}{\mathit{T}}
\newcommand{\italian}{\mathtt{Italian}}
\newcommand{\withitrel}{\mathtt{WithItRelative}}
\newcommand{\withitanc}{\mathtt{ItDescendant}}

% Tests
%\newcommand{\testpassed}{\textsf{Y}}
\newcommand{\testpassed}{\textsf{Y}}
\newcommand{\testfailed}{\bf{\textsf{N}}}

\title{Representing Kinship Relations on the Semantic Web}

\author{Domenico Cantone \inst{1} \and Aldo Gangemi \inst{2,3} \and Cristiano Longo \inst{4}}
%\institute{Dipartimento di Matematica e Informatica, Universit\`a di Catania, Italy\\
%\email{\{cantone, longo, nicolosi\}@dmi.unict.it}}
\institute{\inst{1}Dipartimento di Matematica e Informatica, Universit\`a di Catania, Italy \and 
  \inst{2} STLab, ISTC-CNR, Rome, Italy \and 
  \inst{3} LIPN, Universit\'e Paris 13-CNRS-Sorbonne Cit\'e, France \and 
  \inst{4} Network Consulting Engineering (NCE) S.r.l, Valverde (CT), Italy}

\begin{document}
\maketitle
\setcounter{tocdepth}{1}
%\tableofcontents

\section{Topics}

\begin{frame}
\frametitle{Topics}
    This paper is about:
    \begin{itemize}[<+->]
     \item Kinship relationships
     \item ontology design
     \item \Large{an ontology design method applied to kinship}
    \end{itemize}
\end{frame}

\section{Methodology}
\begin{frame}
\frametitle{Methodology - Artifacts}
The proposed method produces the following artifacts:
\begin{itemize}[<+->]
 \item An informal (but possibly accurate) description of the knowledge domain. At this stage, vocabulary terms
 are defined, and their intuitive meaning is provided in natural 
 language: stories, scenarios, competency questions.
 \item \emph{Intuitive but formal} definitions for vocabulary terms, called the \emph{intuitive description} of
 the knowledge domain. It is provided in any formal language, \emph{not necessarily decidable}. It can be used to validate 
 the informal description. 
 \item A set of \emph{inference test cases}, say $\TestCases$, derived from the intuitive description. They are used to measure the \emph{inferencing
 capability} of a candidate ontology $\Kcand$, in order to provide an immediate view of which types of inferences will be 
 drawn adopting $\Kcand$ and which will not.
\end{itemize}
\end{frame}

\begin{frame}
\frametitle{Methodology - Building ontologies}
Candidate ontologies are built starting from the intuitive description (e.g. $KBsets$).
\vspace{\baselineskip}

\uncover<2->{ The \emph{coherence} of a candidate ontology $\Kcand$
with the intuitive description must be verified
\[
 \KBsets \Longrightarrow \Kcand
\]
in order to guarantee that no unexpected consequence will be thrown by $\Kcand$ (facts inferred by $\Kcand$ are
inferred by $\KBsets$ as well). 
}
\vspace{\baselineskip}

\uncover<3->{
If $\Kcand \not\Longrightarrow \KBsets$, 
then $\Kcand$ has to be tested againts $\TestCases$.
}
\vspace{\baselineskip}

\uncover<4-> {Informally, this is known as \emph{design by competency questions}. 
eXtreme Design is an example, but it lacks the formal method described here.
}

\end{frame}

\section{Kinship}

\begin{frame}
\frametitle{Kinship Ontology}
  Kinship pervades several knowledge domains: anthropology, 
  genetics, sociology, law, etc.
  \vspace{\baselineskip}
  
 Some research about kinship has been done in the Semantic Web (RELATIONSHIP vocabulary, Family History Knowledge base, etc.).
  \vspace{\baselineskip}
  
  Kinship is known as  a difficult domain for reasoning (cf. Stevens et al. at OWLEd).
   \vspace{\baselineskip}

  \uncover<2->{
  Our ontology is intended to be \emph{minimal} and \emph{reasoning-oriented}.
  Vocabulary terms taken into account are:
  \begin{itemize}
   \item the concept $\Persons$, denoting the set of all human beings;
   \item the roles 
\[
  \begin{array}{lll}
  \relative, & \partner, & \child, \\
  \parent, & \ancestor, & \descendant
  \end{array}
\]
  \end{itemize}
  }
\end{frame}

\begin{frame}
\frametitle{Kinship Ontology - Informal description for $\relative$}
  For each term, firstly, an informal description (requirements) is provided.
  \vspace{\baselineskip}
  
  {\Large $\relative$}: \begin{itemize}
   \item it is a relation over persons
   \item everyone is relative of its relatives
   \item nobody is relative of itself
   \item all the relatives of a person's relatives are relatives of the
      person itself
  \end{itemize}
  \vspace{\baselineskip}

  \phantom{\begin{small}
  \begin{tabular}{|l|l|l|}
  \hline
  \emph{Constraint} & \phantom{\emph{Premises}} & \phantom{\emph{Consequence}} \\
   \hline
   $\dom(\relative) \Issub \Persons$ & \phantom{$\Alice\,\relative\,\Bob$} & \phantom{$\Alice \in \Persons$}\\
  \hline
  $\issymmetric{\relative}$ & \phantom{$\Alice\,\relative\,\Bob$} & \phantom{$\Bob\,\relative\,\Alice$}\\
  \hline
  $\relative \equiv \relative^{+} \sqcap (\neg \roleidentity{\Persons})$&\phantom{$\Alice\,\relative\,\Bob$}&\phantom{$\Alice\,\relative\,\Charlie$}\\
    &\phantom{$\Bob\,\relative\,\Charlie$}&\\
    &\phantom{$\Alice\,\neq\,\Bob\,\neq\,\Charlie$}&\\
  \hline
  $\range(\relative) \Issub \Persons$ (derived)& \phantom{$\Alice\,\relative\,\Bob$} & \phantom{$\Bob \in \Persons$}\\
  \hline
  $\isirreflexive{\relative}$ (derived)& \phantom{$\Alice\,\relative\,x$ (derived)}& \phantom{$\Alice\neq x$} \\
 \hline
 \end{tabular}
 \end{small}}
\end{frame}

\begin{frame}
\frametitle{Kinship Ontology - Intuitive but precise description for $\relative$ (1/2)}
  Then, the requirements are encoded in some \emph{formal} language, e.g. a maximally expressive DL.
  \vspace{\baselineskip}
  
  {\Large $\relative$}: \begin{itemize}
   \item it is a relation over persons;
   \item everyone is relative of its relatives;
   \item nobody is relative of itself;
   \item all the relatives of a person's relatives are relatives of the
      person itself.
  \end{itemize}
  \vspace{\baselineskip}

  \begin{small}
  \begin{tabular}{|l|l|l|}
  \hline
  \emph{Constraint} & \phantom{\emph{Premises}} & \phantom{\emph{Consequence}} \\
   \hline
   $\dom(\relative) \Issub \Persons$ & \phantom{$\Alice\,\relative\,\Bob$} & \phantom{$\Alice \in \Persons$}\\
  \hline
  $\issymmetric{\relative}$ & \phantom{$\Alice\,\relative\,\Bob$} & \phantom{$\Bob\,\relative\,\Alice$}\\
  \hline
  \alert{$\relative \equiv \relative^{+} \sqcap (\neg \roleidentity{\Persons})$}&\phantom{$\Alice\,\relative\,\Bob$}&\phantom{$\Alice\,\relative\,\Charlie$}\\
    &\phantom{$\Bob\,\relative\,\Charlie$}&\\
    &\phantom{$\Alice\,\neq\,\Bob\,\neq\,\Charlie$}&\\
  \hline
  \phantom{$\range(\relative) \Issub \Persons$}& \phantom{$\Alice\,\relative\,\Bob$} & \phantom{$\Bob \in \Persons$}\\
  \hline
  \phantom{$\isirreflexive{\relative}$ (derived)}& \phantom{$\Alice\,\relative\,x$}& \phantom{$\Alice\neq x$} \\
 \hline
 \end{tabular}
 \end{small}
\end{frame}

\begin{frame}
\frametitle{Kinship Ontology - Intuitive but precise description for $\relative$ (2/2)}
  Next, additional constraints are derived from the provided ones. 
  \vspace{\baselineskip}
  
  {\Large $\relative$}: \begin{itemize}
   \item it is a relation over persons;
   \item everyone is relative of its relatives;
   \item nobody is relative of itself;
   \item all the relatives of a person's relatives are relatives of the
      person itself.
  \end{itemize}
  \vspace{\baselineskip}

  \begin{small}
  \begin{tabular}{|l|l|l|}
  \hline
  \emph{Constraint} & \phantom{\emph{Premises}} & \phantom{\emph{Consequence}} \\
   \hline
   $\dom(\relative) \Issub \Persons$ & \phantom{$\Alice\,\relative\,\Bob$} & \phantom{$\Alice \in \Persons$}\\
  \hline
  $\issymmetric{\relative}$ & \phantom{$\Alice\,\relative\,\Bob$} & \phantom{$\Bob\,\relative\,\Alice$}\\
  \hline
  $\relative \equiv \relative^{+} \sqcap (\neg \roleidentity{\Persons})$&\phantom{$\Alice\,\relative\,\Bob$}&\phantom{$\Alice\,\relative\,\Charlie$}\\
    &\phantom{$\Bob\,\relative\,\Charlie$}&\\
    &\phantom{$\Alice\,\neq\,\Bob\,\neq\,\Charlie$}&\\
  \hline
  $\range(\relative) \Issub \Persons$& \phantom{$\Alice\,\relative\,\Bob$} & \phantom{$\Bob \in \Persons$}\\
  \hline
  $\isirreflexive{\relative}$& \phantom{$\Alice\,\relative\,x$}& \phantom{$\Alice\neq x$} \\
 \hline
 \end{tabular}
 \end{small}
\end{frame}

\begin{frame}
\frametitle{Kinship Ontology - Test cases for $\relative$}
  Finally, a set of test cases is produced starting from these constraints.
  \vspace{\baselineskip}
  
  {\Large $\relative$}: \begin{itemize}
   \item it is a relation over persons;
   \item everyone is relative of its relatives;
   \item nobody is relative of itself;
   \item all the relatives of a person's relatives are relatives of the
      person itself.
  \end{itemize}
  \vspace{\baselineskip}

  \begin{small}
  \begin{tabular}{|l|l|l|}
  \hline
  \emph{Constraint} & \emph{Premises} & \emph{Consequence} \\
   \hline
   $\dom(\relative) \Issub \Persons$ & $\Alice\,\relative\,\Bob$ & $\Alice \in \Persons$\\
  \hline
  $\issymmetric{\relative}$ & $\Alice\,\relative\,\Bob$ & $\Bob\,\relative\,\Alice$\\
  \hline
  $\relative \equiv \relative^{+} \sqcap (\neg \roleidentity{\Persons})$&$\Alice\,\relative\,\Bob$&$\Alice\,\relative\,\Charlie$\\
    &$\Bob\,\relative\,\Charlie$&\\
    &$\Alice\,\neq\,\Bob\,\neq\,\Charlie$&\\
  \hline
  $\range(\relative) \Issub \Persons$& $\Alice\,\relative\,\Bob$ & $\Bob \in \Persons$\\
  \hline
  $\isirreflexive{\relative}$& $\Alice\,\relative\,x$& $\Alice\neq x$ \\
 \hline
 \end{tabular}
 \end{small}
\end{frame}

\begin{frame}
\frametitle{Kinship Ontology - Intuitive description (1/2)}
  Iterating this process for each vocabulary term, we obtain: 
  \vspace{.3cm}
  
  - a set of constraints, \ldots
  
\[
\begin{array}{rcl}
\KBsets & = & \{\;\; \dom(\relative) \Issub \Persons,\\
&&\phantom{\{\;\;} \issymmetric{\relative},\\
&&\phantom{\{\;\;} \relative \equiv \relative^{+} \sqcap (\neg 
\roleidentity{\Persons}),~~~~~~~ \\
&&\phantom{\{\;\;} \partner \Issub \relative,\\
&&\phantom{\{\;\;} \issymmetric{\partner},\\
&&\phantom{\{\;\;} \child \Issub \relative,\\
&&\phantom{\{\;\;} \parent \equiv \child^{-},\\
&&\phantom{\{\;\;} \descendant \equiv \child^{+},\\
&&\phantom{\{\;\;} \descendant \Issub \relative,\\
&&\phantom{\{\;\;} \ancestor \equiv \parent^{+}\;\;\}\,.
\end{array}
\]
\end{frame}

\begin{frame}
\frametitle{Kinship Ontology - Intuitive description (2/2)}
  \center{\ldots, some additional constraints deriving from $\KBsets$, \ldots}
  
\[
\begin{small}
\begin{array}{rcl}
\KBlcons & = & \{\;\; \range(\relative) \Issub \Persons,\\
&&\phantom{\{\;} \isirreflexive{\relative}, \\
&&\phantom{\{\;} \isirreflexive{\partner}, \\
&&\phantom{\{\;} \isirreflexive{\child}, \\
&&\phantom{\{\;} \isantisymmetric{\child}, \\ %child è sottorelazione di descendant, che è transitiva e irriflessiva
&&\phantom{\{\;} \parent \Issub \relative, \\
&&\phantom{\{\;} \isirreflexive{\parent}, \\
&&\phantom{\{\;} \isantisymmetric{\parent}, \\ %parent è inversa di child, che è antisimmertica
&&\phantom{\{\;} \isirreflexive{\descendant}, \\
&&\phantom{\{\;} \isantisymmetric{\descendant}, \\ %parent è inversa di child, che è antisimmertica
&&\phantom{\{\;} \child \Issub \descendant, \\
%&&\phantom{\{\;} \parent^{-} \Issub \descendant, & \constr\label{PARENTINVDESCENDANT}\\
&&\phantom{\{\;} \ancestor \equiv \descendant^{-}, \\
&&\phantom{\{\;} \ancestor \Issub \relative, \\
&&\phantom{\{\;} \isirreflexive{\ancestor}, \\
&&\phantom{\{\;} \isantisymmetric{\ancestor}, \\ %parent è inversa di child, che è antisimmertica
&&\phantom{\{\;} \parent \Issub \ancestor,  \\
&&\phantom{\{\;} \disjointRoles{\ancestor}{\descendant} 
\;\}\,~~~~~~~  
%&&\phantom{\{\;\;} \child^{-} \Issub \ancestor & \constr\label{CHILDINVANCESTOR} & \;\;\}
\end{array} 
\end{small}
\]
\end{frame}


\begin{frame}
\frametitle{Kinship Ontology - Test cases}
  \center{\ldots, and a test-bed for expected inferences.}
\begin{small}
\begin{table} 
  \begin{tabular}{ccc}
  \begin{tabular}{|l|l|}
  \hline
  \bf{Premises} & \bf{Consequence} \\
  \hline
  $\Alice\,\relative\,\Bob$ & $\Alice \in \Persons$     \\
  \hline
  $\Alice\,\relative\,\Bob$ & $\Bob\,\relative\,\Alice$     \\
  \hline
  $\Alice\,\relative\,\Bob$ & $\Alice\,\relative\,\Charlie$     \\
  $\Bob\,\relative\,\Charlie$&\\
  $\Alice\,\neq\,\Bob\,\neq\,\Charlie$&\\
  \hline
  $\Alice\,\partner\,\Bob$ & $\Alice\,\relative\,\Bob$     \\
  \hline
  $\Alice\,\partner\,\Bob$ & $\Bob\,\partner\,\Alice$     \\
  \hline
  $\Alice\,\child\,\Bob$ & $\Alice\,\relative\,\Bob$     \\
  \hline
  $\Alice\,\parent\,\Bob$ & $\Bob\,\child\,\Alice$     \\
  \hline
  $\Alice\,\descendant\,\Bob$ & $\Alice\,\descendant\,\Charlie$     \\
  $\Bob\,\descendant\,\Charlie$ & \\
  $\Alice \neq \Bob \neq \Charlie$ & \\
  \hline
  $\Alice\,\descendant\,\Bob$ & $\Alice\,\relative\,\Bob$     \\
  \hline
  $\Alice\,\ancestor\,\Bob$ & $\Alice\,\ancestor\,\Charlie$     \\
  $\Bob\,\ancestor\,\Charlie$ & \\
  $\Alice\neq\Bob\neq\Charlie$ & \\
  \hline
  $\Alice\,\relative\,\Bob$ & $\Bob \in \Persons$     \\
  \hline
  $\Alice\,\relative\,x$ & $\Alice\neq x$     \\
  \hline
  \end{tabular} &  &
    \begin{tabular}{|l|l|}
      \hline
      \bf{Premises} & \bf{Consequence} \\
      \hline
      $\Alice\,\partner\,x$ & $\Alice\neq x$     \\
      \hline
      $\Alice\,\child\,x$ & $\Alice \neq x$     \\
      \hline
      $\Alice\,\child\,\Bob$ & $\neg ( \Bob\,\child\,\Alice)$     \\
      \hline
      $\Alice\,\parent\,\Bob$ & $\Alice\,\relative\,\Bob$     \\
      \hline
      $\Alice\,\parent\,x$ & $\Alice\,\neq\,x$     \\
      \hline
      $\Alice\,\parent\,\Bob$ & $\neg( \Bob\,\parent\,\Alice )$     \\
      \hline
      $\Alice\,\descendant\,x$ & $\Alice \neq x$     \\
      \hline
      $\Alice\,\descendant\,\Bob$ & $\neg( \Bob\,\descendant\,\Alice )$     \\
      \hline
      $\Alice\,\child\,\Bob$ & $\Alice\,\descendant\,\Bob$     \\
      \hline
      $\Alice\,\ancestor,\Bob$ & $\Bob\,\descendant\,\Alice$     \\
      \hline
      $\Alice\,\ancestor\,\Bob$ & $\Alice\,\relative\,\Bob$     \\
      \hline
      $\Alice\,\ancestor\,x$ & $\Alice \neq x$     \\
      \hline
      $\Alice\,\ancestor\,\Bob$ & $\neg( \Bob\,\ancestor\,\Alice )$     \\
      \hline
      $\Alice\,\parent,\Bob$ & $\Alice\,\ancestor\,\Bob$     \\
      \hline
      $\Alice\,\ancestor\,\Bob$ & $\neg( \Alice\,\descendant\,\Bob )$     \\
      \hline
      \end{tabular}
\end{tabular}  
\caption{Test Cases}
\end{table}
\end{small}
\end{frame}

\begin{frame}
\frametitle{Kinship Ontology - Implementation}
  
  These \emph{artifacts} will be used when building descriptions
  of the knowledge domain using \emph{decidable} languages.
  %TODO definire decidable
  \vspace{\baselineskip}

  In this case, the chosen languages are the description logics
  \begin{center}
    \begin{itemize}
      \item {\Large\sroiq (OWL-DL)} and %TODO verificare
      \item {\Large\elplusplus (OWL-EL)}.
    \end{itemize}
  \end{center}
\end{frame}

\section{\sroiq}

%%QUA
\begin{frame}
\frametitle{Kinship Ontology - \sroiq (1/5)}

  \sroiq is a very expressive description logic. However it has some
  limitations.
  
  For example, Boolean operators and the transitive closure operator on roles are not allowed.
  \vspace{\baselineskip}
  
  \begin{small}
  \[  
   \begin{array}{rcl}
    \KBsets & = & \{\;\; \dom(\relative) \Issub \Persons,\\
    &&\phantom{\{\;\;} \issymmetric{\relative},\\
    &&\phantom{\{\;\;} \alert{\relative \equiv \relative^{+} \sqcap (\neg 
    \roleidentity{\Persons})},~~~~~~~ \\
    &&\phantom{\{\;\;} \partner \Issub \relative,\\
    &&\phantom{\{\;\;} \issymmetric{\partner},\\
    &&\phantom{\{\;\;} \child \Issub \relative,\\
    &&\phantom{\{\;\;} \parent \equiv \child^{-},\\
    &&\phantom{\{\;\;} \alert{\descendant \equiv \child^{+}},\\
    &&\phantom{\{\;\;} \descendant \Issub \relative,\\
    &&\phantom{\{\;\;} \alert{\ancestor \equiv \parent^{+}}\;\;\}\,.
    \end{array}
  \]
  \end{small}
\end{frame}

\begin{frame}
\frametitle{Kinship Ontology - $\sroiq$ (2/5) }

Thus, let us consider the \sroiq \emph{subset} (the domain is expressed by means of an existential restriction)  of $\KBsets$:

  \begin{small}
  \[  
   \begin{array}{rcl}
    \KBzerozero & = & \{\;\; \exists\relative.\top \Issub \Persons,\\
    &&\phantom{\{\;\;} \issymmetric{\relative},\\
    &&\phantom{\{\;\;} \partner \Issub \relative,\\
    &&\phantom{\{\;\;} \issymmetric{\partner},\\
    &&\phantom{\{\;\;} \child \Issub \relative,\\
    &&\phantom{\{\;\;} \parent \equiv \child^{-},\\
    &&\phantom{\{\;\;} \descendant \Issub \relative \;\;\}\,.
    \end{array}
  \]
  \end{small}
  
  \uncover<2->{
  Clearly, $\KBsets \Longrightarrow \KBzerozero$, i.e., $\KBzerozero$ is \emph{coherent}
  with the intuitive description $\KBsets$.
  }
  
  \uncover<3->{
  But, $\KBzerozero \not\Longrightarrow \KBsets$.\\
  For example
  \[
    \KBzerozero \cup \{ \Alice \child \Bob, \Bob \child \Charlie \} \Longrightarrow \{\Alice \descendant \Bob \}.
  \]
  }
\end{frame}

\begin{frame}
\frametitle{Kinship Ontology - $\sroiq$ (3/5) }

We augment the \emph{inferencing capability} of our ontology for kinship
by extending $\KBzerozero$ with some constraints of $\KBlcons$.

  \begin{small}
  \[  
   \begin{array}{rcl}
    \KBzeroone & = & \{\;\; \exists\relative.\top \Issub \Persons,\\
    &&\phantom{\{\;\;} \issymmetric{\relative},\\
    &&\phantom{\{\;\;} \partner \Issub \relative,\\
    &&\phantom{\{\;\;} \issymmetric{\partner},\\
    &&\phantom{\{\;\;} \parent \equiv \child^{-},\\
    &&\phantom{\{\;\;} \descendant \Issub \relative, \\
    &&\phantom{\{\;\;} \isirreflexive{\partner},\\
    &&\phantom{\{\;\;} \ancestor\equiv\descendant^{-},\\
    &&\phantom{\{\;\;} \child\Issub\descendant, \\
    &&\phantom{\{\;\;} \istransitive{\descendant}, \\
    &&\phantom{\{\;\;} \isirreflexive{\relative},\\
    &&\phantom{\{\;\;} \isantisymmetric{\descendant} \;\;\}\,.
    \end{array}
  \]
  \end{small}

Note:  $\KBzeroone \Longrightarrow \{\child \Issub \relative\}$.
\end{frame}

\begin{frame}
\frametitle{Kinship Ontology - $\sroiq$ (4/5) }

Unfortunately, $\KBzeroone$ is not \sroiq, as irreflexivity and asymmetry cannot be stated for
transitive roles in this description logic. 
  \begin{small}
  \[  
   \begin{array}{rcl}
    \KBzeroone & = & \{\;\; \exists\relative.\top \Issub \Persons,\\
    &&\phantom{\{\;\;} \issymmetric{\relative},\\
    &&\phantom{\{\;\;} \partner \Issub \relative,\\
    &&\phantom{\{\;\;} \issymmetric{\partner},\\
    &&\phantom{\{\;\;} \parent \equiv \child^{-},\\
    &&\phantom{\{\;\;} \descendant \Issub \relative, \\
    &&\phantom{\{\;\;} \isirreflexive{\partner},\\
    &&\phantom{\{\;\;} \ancestor\equiv\descendant^{-},\\
    &&\phantom{\{\;\;} \child\Issub\descendant, \\
    &&\phantom{\{\;\;} \alert{\istransitive{\descendant}}, \\
    &&\phantom{\{\;\;} \alert{\isirreflexive{\relative}},\\
    &&\phantom{\{\;\;} \alert{\isantisymmetric{\descendant}} \;\;\}\,.
    \end{array}
  \]
  \end{small}
  
NOTE: $\{\descendant \Issub \relative, \isirreflexive{\relative}\} \Longrightarrow \{\isirreflexive{\descendant}\}$. 
\end{frame}

\begin{frame}
\frametitle{Kinship Ontology - $\sroiq$ (5/5) }

Thus, we \emph{split} $\KBzeroone$ into two \sroiq ontologies \ldots

  \begin{small}
  \[  
   \begin{array}{rcl}
    \KBzero & = & \{\;\; \exists\relative.\top \Issub \Persons,\\
    &&\phantom{\{\;\;} \issymmetric{\relative},\\
    &&\phantom{\{\;\;} \partner \Issub \relative,\\
    &&\phantom{\{\;\;} \issymmetric{\partner},\\
    &&\phantom{\{\;\;} \parent \equiv \child^{-},\\
    &&\phantom{\{\;\;} \descendant \Issub \relative, \\
    &&\phantom{\{\;\;} \isirreflexive{\partner},\\
    &&\phantom{\{\;\;} \ancestor\equiv\descendant^{-},\\
    &&\phantom{\{\;\;} \child\Issub\descendant \;\;\}\,.\\
    &&\\
    \KBtra & = & \KBzero \cup \{ \istransitive{\descendant} \} \\
    &&\\
    \KBirr & = & \KBzero \cup \{ \isirreflexive{\relative},\\
    &&\phantom{\KBzero \cup \{}  \isantisymmetric{\descendant} \}\,.
    \end{array}  
  \]
  \end{small}  
\end{frame}

\begin{frame}
\frametitle{Implementation - $\sroiq$ - Coherence }

Clearly, both $\KBtra$ and $\KBirr$ are coherent with $\KBsets$, as

\[
\begin{array}{rclr}
 \KBtra & \subseteq & \KBsets \cup \KBlcons & \mbox{ and }\\
 \KBirr & \subseteq & \KBsets \cup \KBlcons . 
\end{array}
\]
\end{frame}



\begin{frame}
\frametitle{Kinship Ontology - $\sroiq$ - Inferencing Capability (1/2) }

Then, we compare them in terms of expected inference types.
\vspace{\baselineskip}
\begin{center}
\begin{small}
\begin{tabular}{|c|l|c|c|}
\hline
\bf{Premises} & \bf{Consequence} & $\KBtra$ & $\KBirr$ \\
\hline
$\Alice\,\relative\,\Bob$ & $\Alice \in \Persons$ & \testpassed & \testpassed\\
\hline
$\Alice\,\relative\,\Bob$ & $\Bob\,\relative\,\Alice$ & \testpassed & \testpassed \\
\hline
$\Alice\,\relative\,\Bob$ & $\Alice\,\relative\,\Charlie$ & \testfailed & \testfailed\\
$\Bob\,\relative\,\Charlie$&&&\\
$\Alice\,\neq\,\Bob\,\neq\,\Charlie$&&&\\
\hline
$\Alice\,\partner\,\Bob$ & $\Alice\,\relative\,\Bob$ & \testpassed & \testpassed\\
\hline
$\Alice\,\partner\,\Bob$ & $\Bob\,\partner\,\Alice$ & \testpassed & \testpassed\\
\hline
$\Alice\,\child\,\Bob$ & $\Alice\,\relative\,\Bob$  & \testpassed & \testpassed\\
\hline
$\Alice\,\parent\,\Bob$ & $\Bob\,\child\,\Alice$ & \testpassed & \testpassed\\
\hline
$\Alice\,\descendant\,\Bob$ & $\Alice\,\descendant\,\Charlie$ & \testpassed & \testfailed\\
$\Bob\,\descendant\,\Charlie$ & & &\\
$\Alice \neq \Bob \neq \Charlie$ & & &\\
\hline
$\Alice\,\descendant\,\Bob$ & $\Alice\,\relative\,\Bob$ & \testpassed & \testpassed\\
\hline
$\Alice\,\ancestor\,\Bob$ & $\Alice\,\ancestor\,\Charlie$ & \testpassed & \testfailed\\
$\Bob\,\ancestor\,\Charlie$ & & &\\
$\Alice\neq\Bob\neq\Charlie$& & &\\
\hline
$\Alice\,\relative\,\Bob$ & $\Bob \in \Persons$ & \testpassed & \testpassed\\
\hline
$\Alice\,\relative\,x$ & $\Alice\neq x$ & \testfailed & \testpassed\\
\hline
$\Alice\,\partner\,x$ & $\Alice\neq x$ & \testpassed & \testpassed\\
\hline
$\Alice\,\child\,x$ & $\Alice \neq x$ & \testfailed & \testpassed\\
\hline
$\Alice\,\child\,\Bob$ & $\neg ( \Bob\,\child\,\Alice)$ & \testfailed & \testpassed\\
\hline
$\Alice\,\parent\,\Bob$ & $\Alice\,\relative\,\Bob$ & \testpassed & \testpassed\\
\hline
\end{tabular}
\end{small}
\end{center}
\end{frame}

\begin{frame}
\frametitle{Kinship Ontology - $\sroiq$ - Inferencing Capability (2/2) }
\begin{center}
\begin{small}
\begin{tabular}{|c|l|c|c|}
\hline
\bf{Premises} & \bf{Consequence} & $\KBtra$ & $\KBirr$\\
\hline
$\Alice\,\parent\,x$ & $\Alice\,\neq\,x$ & \testfailed & \testpassed\\
\hline
$\Alice\,\parent\,\Bob$ & $\neg( \Bob\,\parent\,\Alice )$ & \testfailed & \testpassed\\
\hline
$\Alice\,\descendant\,x$ & $\Alice \neq x$ & \testfailed & \testpassed\\
\hline
$\Alice\,\descendant\,\Bob$ & $\neg( \Bob\,\descendant\,\Alice )$ & \testfailed & \testpassed\\
\hline
$\Alice\,\child\,\Bob$ & $\Alice\,\descendant\,\Bob$ & \testpassed & \testpassed\\
\hline
$\Alice\,\ancestor,\Bob$ & $\Bob\,\descendant\,\Alice$ & \testpassed & \testpassed\\
\hline
$\Alice\,\ancestor\,\Bob$ & $\Alice\,\relative\,\Bob$ & \testpassed & \testpassed\\
\hline
$\Alice\,\ancestor\,x$ & $\Alice \neq x$& \testfailed & \testpassed\\
\hline
$\Alice\,\ancestor\,\Bob$ & $\neg( \Bob\,\ancestor\,\Alice )$& \testfailed & \testpassed\\
\hline
$\Alice\,\parent\,\Bob$ & $\Alice\,\ancestor\,\Bob$& \testpassed & \testpassed\\
\hline
$\Alice\,\child\,\Bob$ & $\Bob\,\ancestor\,\Alice$ & \testpassed & \testpassed\\
\hline
$\Alice\,\ancestor\,\Bob$ & $\neg( \Alice\,\descendant\,\Bob )$ & \testfailed & \testpassed\\
\hline
\end{tabular}
\end{small}
\end{center}
\end{frame}

\begin{frame}
\frametitle{Kinship Ontology - $\sroiq$ - Use Case (1/4)}

$\KBirr$ overtakes $\KBtra$ in terms of inferencing capability, but
it does not enforce the transitivity of $\ancestor$ and $\descendant$.
\vspace{\baselineskip}

But transitivity can be somehow \emph{emulated}.
\vspace{\baselineskip}

Let us consider a use case in which a user needs to retrieve all 
the persons with an Italian ancestor.
\end{frame}

\begin{frame}
\frametitle{Kinship Ontology - $\sroiq$ - Use Case (2/4)}
The intuitive description is extended with:
\begin{small}
\[
  \KBsetsq = \{\; \italian \Issub \Persons,~~ \withitanc \equiv \exists
  \descendant.\italian \;\} \,.
\]
\end{small}%

Some additional use cases are provided:
\vspace{\baselineskip}

\begin{center}
\begin{small}
\begin{tabular}{|l|l|}
\hline
\bf{Premises} & \bf{Consequence}\\
\hline
  $\Alice\,\descendant\,\Bob$ & $\Alice \in \withitanc$\\
  $\Bob \in \italian$ & \\
\hline
  $\Alice\,\descendant\,\Bob$ & $\Alice \in \withitanc$\\
  $\Bob\,\descendant\,\Charlie$  &\\
  $\Charlie \in \italian$  &\\
\hline
  $\Alice\,\child\,\Bob$ & $\Alice \in \withitanc$\\
  $\Bob \in \italian$  &\\
\hline
  $\Bob\,\ancestor\,\Alice$ & $\Alice \in \withitanc$\\
  $\Bob \in \italian$  &\\
\hline
  $\Bob\,\ancestor\,\Alice$ & $\Alice \in \withitanc$\\
  $\Charlie\,\ancestor\,\Bob$  &\\
  $\Charlie \in \italian$ &\\
\hline
  $\Bob\,\parent\,\Alice$ & $\Alice \in \withitanc$\\
  $\Bob \in \italian$  &\\
\hline
\end{tabular}
\end{small}
\end{center}
\end{frame}

\begin{frame}
\frametitle{Kinship Ontology - $\sroiq$ - Use Case (3/4)}
As expected, just extending $\KBirr$ with $\KBq$ is not satisfactory.
\vspace{\baselineskip}

\begin{small}
\begin{center}
\begin{tabular}{|l|l|c|c|c|c|}
\hline
\bf{Premises} & \bf{Consequence} & $\KBtra \cup \KBsetsq$ & $\KBirr \cup \KBsetsq$\\
\hline
  $\Alice\,\descendant\,\Bob$ & $\Alice \in \withitanc$ & \testpassed & \testpassed \\
  $\Bob \in \italian$ & & &\\
\hline
  $\Alice\,\descendant\,\Bob$ & $\Alice \in \withitanc$ & \testpassed & \testfailed\\
  $\Bob\,\descendant\,\Charlie$ & & &\\
  $\Charlie \in \italian$ & & &\\
\hline
  $\Alice\,\child\,\Bob$ & $\Alice \in \withitanc$ & \testpassed & \testpassed\\
  $\Bob \in \italian$ & & & \\
\hline
  $\Bob\,\ancestor\,\Alice$ & $\Alice \in \withitanc$ & \testpassed & \testpassed\\
  $\Bob \in \italian$ & & & \\
\hline
  $\Bob\,\ancestor\,\Alice$ & $\Alice \in \withitanc$ & \testpassed & \testfailed\\
  $\Charlie\,\ancestor\,\Bob$ & & &\\
  $\Charlie \in \italian$ & & &\\
\hline
  $\Bob\,\parent\,\Alice$ & $\Alice \in \withitanc$ & \testpassed & \testpassed\\
  $\Bob \in \italian$ & & &\\
\hline
\end{tabular}
\end{center}
\end{small}
\end{frame}

\begin{frame}
\frametitle{Kinship Ontology - $\sroiq$ - Use Case (4/4)}
Thus, we adopt a \emph{less intuitive} definition for $\withitanc$.
\begin{small}
\[
 \begin{array}{rcl}
  \KBq & = & \{\; \italian \Issub \Persons, \\
  && \phantom{\{\;} \withitanc \equiv \exists\descendant.\italian 
  \sqcup \exists\descendant.\withitanc \;\}\;.
 \end{array} 
\] 
\end{small}

\uncover<2->{
\emph{Coherence : OK}
\[
 \KBsets \cup \KBsetsq \Longrightarrow \KBirr \cup \KBq
\]
}

\uncover<3->{
\begin{small}
\begin{center}
\begin{tabular}{|l|l|c|c|c|}
\hline
\bf{Premises} & \bf{Consequence} & $\KBtra \cup \KBsetsq$ & $\KBirr \cup \KBsetsq$ & $\KBirr \cup \KBq$\\
\hline
  $\Alice\,\descendant\,\Bob$ & $\Alice \in \withitanc$ & \testpassed & \testpassed & \testpassed \\
  $\Bob \in \italian$ & & & &\\
\hline
  $\Alice\,\descendant\,\Bob$ & $\Alice \in \withitanc$ & \testpassed & \testfailed& \testpassed \\
  $\Bob\,\descendant\,\Charlie$ & & & &\\
  $\Charlie \in \italian$ & & & &\\
\hline
  $\Alice\,\child\,\Bob$ & $\Alice \in \withitanc$ & \testpassed & \testpassed& \testpassed \\
  $\Bob \in \italian$ & & & &\\
\hline
  $\Bob\,\ancestor\,\Alice$ & $\Alice \in \withitanc$ & \testpassed & \testpassed& \testpassed \\
  $\Bob \in \italian$ & & & &\\
\hline
  $\Bob\,\ancestor\,\Alice$ & $\Alice \in \withitanc$ & \testpassed & \testfailed & \testpassed \\
  $\Charlie\,\ancestor\,\Bob$ & & & &\\
  $\Charlie \in \italian$ & & & &\\
\hline
  $\Bob\,\parent\,\Alice$ & $\Alice \in \withitanc$ & \testpassed & \testpassed & \testpassed\\
  $\Bob \in \italian$ & & & &\\
\hline
\end{tabular}
\end{center}
\end{small}
}
\end{frame}

\section{\elplusplus}
\begin{frame}
\frametitle{Kinship Ontology - $\elplusplus$}
\elplusplus is a description logic which admits \emph{polynomial-time} reasoning algorithms.
\vspace{\baselineskip}

It has some relevant expressive limitations, e.g., concepts disjunction is not allowed.
\vspace{\baselineskip}

Thus, \elplusplus is suitable for applications when the amount of available resources is
limited.
\vspace{\baselineskip}

\begin{center}
\begin{small}
\[
 \begin{array}{rcl}
  \KBel& = & \{\; \dom(\relative) \Issub \Persons,\\
  && \phantom{\{\;} \partner\Issub\relative,  \\
  && \phantom{\{\;} \descendant\Issub\relative,\\
  && \phantom{\{\;} \istransitive{\descendant},\\
  && \phantom{\{\;} \ancestor\Issub\relative,\\
  && \phantom{\{\;} \parent\Issub\ancestor \\  
  && \phantom{\{\;} \range(\relative) \Issub \Persons,\\
  && \phantom{\{\;} \child\Issub\descendant,\\
  && \phantom{\{\;} \istransitive{\ancestor} \;\}\,.
 \end{array}
\]
\end{small}%
\end{center}
\end{frame}

\begin{frame}
\frametitle{Kinship Ontology - $\elplusplus$ - Coherence}
$\KBel$ is coherent with the intuitive description we provided for the kinship domain.
\[
\begin{array}{cccl}
 \KBsets &\Longrightarrow& \KBel & \mbox{ as } \KBel \subseteq \KBsets \cup \KBlcons\mbox{, and }\\
 &&&\\
 \KBsets \cup \KBsetsq & \Longrightarrow & \KBel \cup \KBsetsq & \mbox{ easily follows.}
\end{array} 
\]
\end{frame}

\begin{frame}
\frametitle{Kinship Ontology - $\elplusplus$ - Inferencing Capability (1/2)}

\begin{center}
\begin{small}
\begin{tabular}{|c|l|c|c|c|}
\hline
\bf{Premises} & \bf{Consequence} & $\KBtra$ & $\KBirr$ & $\KBel$\\
\hline
$\Alice\,\relative\,\Bob$ & $\Alice \in \Persons$ & \testpassed & \testpassed & \testpassed \\
\hline
$\Alice\,\relative\,\Bob$ & $\Bob\,\relative\,\Alice$ & \testpassed & \testpassed & \testfailed \\
\hline
$\Alice\,\relative\,\Bob$ & $\Alice\,\relative\,\Charlie$ & \testfailed & \testfailed & \testfailed \\
$\Bob\,\relative\,\Charlie$&&&&\\
$\Alice\,\neq\,\Bob\,\neq\,\Charlie$&&&&\\
\hline
$\Alice\,\partner\,\Bob$ & $\Alice\,\relative\,\Bob$ & \testpassed & \testpassed & \testpassed \\
\hline
$\Alice\,\partner\,\Bob$ & $\Bob\,\partner\,\Alice$ & \testpassed & \testpassed & \testfailed \\
\hline
$\Alice\,\child\,\Bob$ & $\Alice\,\relative\,\Bob$  & \testpassed & \testpassed & \testpassed \\
\hline
$\Alice\,\parent\,\Bob$ & $\Bob\,\child\,\Alice$ & \testpassed & \testpassed & \testfailed \\
\hline
$\Alice\,\descendant\,\Bob$ & $\Alice\,\descendant\,\Charlie$ & \testpassed & \testfailed & \testpassed \\
$\Bob\,\descendant\,\Charlie$ & & & &\\
$\Alice \neq \Bob \neq \Charlie$ & & & &\\
\hline
$\Alice\,\descendant\,\Bob$ & $\Alice\,\relative\,\Bob$ & \testpassed & \testpassed & \testpassed \\
\hline
$\Alice\,\ancestor\,\Bob$ & $\Alice\,\ancestor\,\Charlie$ & \testpassed & \testfailed & \testpassed \\
$\Bob\,\ancestor\,\Charlie$ & & & &\\
$\Alice\neq\Bob\neq\Charlie$& & & &\\
\hline
$\Alice\,\relative\,\Bob$ & $\Bob \in \Persons$ & \testpassed & \testpassed & \testpassed \\
\hline
$\Alice\,\relative\,x$ & $\Alice\neq x$ & \testfailed & \testpassed & \testfailed \\
\hline
$\Alice\,\partner\,x$ & $\Alice\neq x$ & \testpassed & \testpassed & \testfailed \\
\hline
$\Alice\,\child\,x$ & $\Alice \neq x$ & \testfailed & \testpassed & \testfailed \\
\hline
$\Alice\,\child\,\Bob$ & $\neg ( \Bob\,\child\,\Alice)$ & \testfailed & \testpassed & \testfailed \\
\hline
$\Alice\,\parent\,\Bob$ & $\Alice\,\relative\,\Bob$ & \testpassed & \testpassed & \testpassed \\
\hline
\end{tabular}
\end{small}
\end{center}
\end{frame}

\begin{frame}
\frametitle{Kinship Ontology - $\elplusplus$ - Inferencing Capability (2/2)}
\begin{center}
\begin{small}
\begin{tabular}{|c|l|c|c|c|}
\hline
\bf{Premises} & \bf{Consequence} & $\KBtra$ & $\KBirr$ & $\KBel$\\
\hline
$\Alice\,\parent\,x$ & $\Alice\,\neq\,x$ & \testfailed & \testpassed & \testfailed \\
\hline
$\Alice\,\parent\,\Bob$ & $\neg( \Bob\,\parent\,\Alice )$ & \testfailed & \testpassed & \testfailed \\
\hline
$\Alice\,\descendant\,x$ & $\Alice \neq x$ & \testfailed & \testpassed & \testfailed \\
\hline
$\Alice\,\descendant\,\Bob$ & $\neg( \Bob\,\descendant\,\Alice )$ & \testfailed & \testpassed & \testfailed \\
\hline
$\Alice\,\child\,\Bob$ & $\Alice\,\descendant\,\Bob$ & \testpassed & \testpassed & \testpassed \\
\hline
$\Alice\,\ancestor,\Bob$ & $\Bob\,\descendant\,\Alice$ & \testpassed & \testpassed & \testfailed \\
\hline
$\Alice\,\ancestor\,\Bob$ & $\Alice\,\relative\,\Bob$ & \testpassed & \testpassed & \testpassed \\
\hline
$\Alice\,\ancestor\,x$ & $\Alice \neq x$& \testfailed & \testpassed & \testfailed \\
\hline
$\Alice\,\ancestor\,\Bob$ & $\neg( \Bob\,\ancestor\,\Alice )$& \testfailed & \testpassed & \testfailed \\
\hline
$\Alice\,\parent\,\Bob$ & $\Alice\,\ancestor\,\Bob$& \testpassed & \testpassed & \testpassed \\
\hline
$\Alice\,\child\,\Bob$ & $\Bob\,\ancestor\,\Alice$ & \testpassed & \testpassed & \testfailed \\
\hline
$\Alice\,\ancestor\,\Bob$ & $\neg( \Alice\,\descendant\,\Bob )$ & \testfailed & \testpassed & \testfailed \\
\hline
\end{tabular}
\end{small}
\end{center}
\end{frame}

\begin{frame}
\frametitle{Kinship Ontology - $\elplusplus$ - Use Case}
\begin{small}
\begin{center}
\begin{tabular}{|l|l|c|c|c|c|}
\hline
\bf{Premises} & \bf{Consequence} & $\KBtra \cup \KBsetsq$ & $\KBirr \cup \KBsetsq$ & $\KBirr \cup \KBq$ & $\KBel \cup \KBsetsq$\\
\hline
  $\Alice\,\descendant\,\Bob$ & $\Alice \in \withitanc$ & \testpassed & \testpassed & \testpassed & \testpassed\\
  $\Bob \in \italian$ & & & & &\\
\hline
  $\Alice\,\descendant\,\Bob$ & $\Alice \in \withitanc$ & \testpassed & \testfailed& \testpassed  & \testpassed\\
  $\Bob\,\descendant\,\Charlie$ & & & & &\\
  $\Charlie \in \italian$ & & & & &\\
\hline
  $\Alice\,\child\,\Bob$ & $\Alice \in \withitanc$ & \testpassed & \testpassed& \testpassed & \testpassed\\
  $\Bob \in \italian$ & & & & &\\
\hline
  $\Bob\,\ancestor\,\Alice$ & $\Alice \in \withitanc$ & \testpassed & \testpassed& \testpassed & \testfailed\\
  $\Bob \in \italian$ & & & & &\\
\hline
  $\Bob\,\ancestor\,\Alice$ & $\Alice \in \withitanc$ & \testpassed & \testfailed & \testpassed & \testfailed\\
  $\Charlie\,\ancestor\,\Bob$ & & & & &\\
  $\Charlie \in \italian$ & & & &&\\
\hline
  $\Bob\,\parent\,\Alice$ & $\Alice \in \withitanc$ & \testpassed & \testpassed & \testpassed & \testfailed\\
  $\Bob \in \italian$ & & & & &\\
\hline
\end{tabular}
\end{center}
\end{small}
\end{frame}

\section{Conclusions}
\begin{frame}
\frametitle{Conclusions}
With reference to some basic kinship relations, in this work we have provided:
\begin{itemize}
 \item an intuitive but formal description of the relations, using a very expressive description logic
 \item a set of test cases, to measure the inferencing capability of candidate kinship ontologies
 \item a (limited) real word use case
 \item two $\sroiq$ ontologies and an $\elplusplus$ ontology for kinship
\end{itemize}
\end{frame}

\begin{frame}
\frametitle{Future Works - Kinship}
About kinship:
\begin{itemize}
 \item other aspectcs of kinship have to be explored, for example \emph{gender-specific}
  relations like $\mathtt{isFatherOf}$
 \item coherence and inference capabilities of existing ontologies and large KB will be investigated
\end{itemize}
\vspace{\baselineskip}

\uncover<2->{
About the ontology design methods:
\begin{itemize}
 \item a standard format for test cases and test results for inferencing capabilities
 can be useful to investigate (see \emph{EARL})
 \item coherence proofs should be automatically verifiable
\end{itemize}
}
\end{frame}

\begin{frame}
\frametitle{Conclusions}
  \vspace{1.5cm}
  \begin{center}
  \Large{Thank You!}
   
  \end{center}
\end{frame}

\end{document}