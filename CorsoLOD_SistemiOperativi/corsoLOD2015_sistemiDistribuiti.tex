\documentclass[8pt]{beamer}
\usepackage[nobglogo]{beamerthemedmi-owled}
\usepackage[utf8x]{inputenc}
\usepackage{default}
\usepackage{url}
\usepackage{verbatim}
\usepackage{graphicx}
\usepackage{mathrsfs}
\usepackage{dl}
\usepackage{mls}
\usepackage{fancyvrb}

%\usepackage{listings}


\mode<presentation>
{
  \usetheme{dmi-owled}
  %\usetheme{Warsaw}
  % or ...

  \setbeamercovered{transparent}
  % or whatever (possibly just delete it)
}

\title{Introduzione ai Linked Open Data e al Web Semantico}

\author{Cristiano Longo\\ 
{\small{longo@dmi.unict.it}}}



\date{Universit\`a di Catania, 11 Giugno 2015}
\newcommand{\urlsingle}[1]{{\small {\center {\url{#1}}}}}

\begin{document}
\maketitle
\setcounter{tocdepth}{1}

\section{Introduzione}

\begin{frame}
\frametitle{Argomenti}
Questa presentazione tratter\`a i seguenti argomenti:
\begin{itemize}
 \item Motivazioni del Web Semantico
 \item Definizione formale di Ontologie
 \item Interrogazioni sulle Ontologie
 \item Vocabolari
\end{itemize}
\end{frame}

\begin{frame}
\frametitle{Definizione di Open Data - Open Knowledge Foundation}

\begin{quote}
Open means anyone can freely access, use, modify, and share for any purpose (subject, at most, to requirements that preserve provenance and openness).\footnote{\url{http://opendefinition.org}} 
\end{quote}

da \emph{Open Knowledge Foundation: The Open Definition}
\end{frame}

\begin{frame}
\frametitle{Definizione di Open Data - Agenzia per l'Italia Digitale}

Nelle \emph{LINEE GUIDA NAZIONALI PER LA VALORIZZAZIONE DEL PATRIMONIO INFORMATIVO PUBBLICO},
emanate dall'\emph{Agenzia per l'Italia Digitale (AgID)} vengono riportate le seguenti
definizioni
\uncover<2->{
a) \emph{formato dei dati di tipo aperto}, un formato di dati reso pubblico, documentato esaustivamente e neutro rispetto agli
strumenti tecnologici necessari per la fruizione dei dati stessi;
\vspace{\baselineskip}
}

\uncover<3->{
b) \emph{dati di tipo aperto}, i dati che presentano le seguenti caratteristiche:
1) sono disponibili secondo i termini di una licenza che ne permetta l'utilizzo da parte di chiunque, anche per
finalità commerciali, in formato disaggregato;
}
\uncover<4->{
2) sono accessibili attraverso le tecnologie dell'informazione e della comunicazione, ivi comprese le reti telematiche
pubbliche e private, 
}
\uncover<5->{
in formati aperti ai sensi della lettera a), sono adatti all'utilizzo automatico da parte di
programmi per elaboratori 
}
\uncover<6->{
e sono provvisti dei relativi \emph{metadati};
\vspace{\baselineskip}
}

\uncover<7->{
3) sono resi disponibili gratuitamente attraverso le tecnologie dell'informazione e della comunicazione, ivi comprese
le reti telematiche pubbliche e private, oppure sono resi disponibili ai costi marginali sostenuti per la loro
riproduzione e divulgazione. L'Agenzia per l'Italia digitale deve stabilire, con propria deliberazione, i casi
eccezionali, individuati secondo criteri oggettivi, trasparenti e verificabili, in cui essi sono resi disponibili a tariffe
superiori ai costi marginali. [...]
}
\end{frame}

\begin{frame}
\frametitle{Limiti del World Wide Web (1/4)}

Tecnicamente, \`e sufficiente pubblicare le informazioni su un sito web o come documenti
\emph{non strutturati}, ma usando formati aperti?

\uncover<2->{
\begin{quote}
The Web was designed as an information space, with the goal that it should be 
useful not only for human-human communication, but also that machines would be
able to participate and help. One of the major obstacles to this has been the
fact \textbf{that most information on the Web is designed for human consumption}, and 
even if it was derived from a database with well defined meanings (in at least 
some terms) for its columns, that \textbf{the structure of the data is not evident to 
a robot browsing the web.}  
\end{quote}
\vspace{\baselineskip}
\emph{Semantic Web Roadmap}, Tim Berners-Lee, 1998.
}
\end{frame}

\begin{frame}
\frametitle{Limiti del World Wide Web (2/4)}
Alcuni problemi nell'interpretazione di testi derivano da:
\begin{description}
 \item[Lingue Differenti] e.g. $Parigi$ e $Paris$ possono indicare la stessa citt\`a.
 \item[Omonimie] e.g. esistono svariate citt\`a chiamate 
 \emph{Paris} nel mondo (Arkansas, Idaho, Illinois, Kentucky,
 Maine, Michigan, Missouri, New York, \ldots);
\end{description}
\end{frame}

\begin{frame}
\frametitle{Limiti del World Wide Web (3/4)}

La situazione si complica in presenza di contenuti multimediali.

\begin{figure}
    \includegraphics[width=250px]{unrecognizable.jpg} 
\end{figure}
\end{frame}

\begin{frame}
\frametitle{Limiti del World Wide Web (4/4)}
Come conseguenza, spesso \`e impossibile eseguire su web ricerce 
\emph{complesse} ottenendo risultati accurati. Ad esempio, cercando sul web
\emph{``Federico II places''} non si ottengono risultati in prima pagina su 
Federico II, ma solo sull'omonima universit\`a:
  
  \begin{small}
    \begin{enumerate}
   \item Universit\`a degli Studi di Napoli "Federico II" | OPEN Places
   \item AOU - Policlinico "Federico II" - Napoli, Italy - Hospital | Facebook
   \item Federico II Ingegneria Via Claudio - College and University | Facebook
   \item MARIA CATERINA FONTE - www.docenti.unina.it
  \end{enumerate}
  \end{small}
\end{frame}

% \begin{frame}
% \frametitle{World Wide Web Consortium}
% Le tecnologie del \emph{Web Semantico} rispondono ad un insieme di
% standard e protocolli promossi e mantenuti dal \emph{World Wide Web Consortium} 
% (in breve \emph{W3C}, vedi \url{http://www.w3.org}).
% \vspace{\baselineskip}
% 
% Il W3C \`e un consorzio di standardizzazione per il Web che
% conta 403 membri tra aziende e organizzazioni governative: CNR,
% Microsoft Corporation, Apple Inc., Intel Corporation, Facebook, Google Inc., \ldots
% \vspace{\baselineskip}
% 
% Altri standard sviluppati in seno al W3C sono: URL, HTTP, XML, HTML, CSS, SOAP, 
% WSDL, Javascript. 
% \end{frame}

\begin{frame}
\frametitle{Il Web Semantico}
\begin{quote}
[\ldots] the Semantic Web approach instead develops languages for expressing
information in a machine processable form. 
\end{quote}
\emph{Semantic Web Roadmap}, Tim Berners-Lee, 1998.

\begin{figure}
    \includegraphics[width=250px]{federicoII_dbpedia.png} 
    \caption{Federico II su dbpedia.org}
\end{figure}
\end{frame}

\begin{frame}
\frametitle{Linked Open Data Cloud (1/2)}
\begin{quote}
The Semantic Web is a web of data, in some ways like a global database.
\end{quote}
\small{\emph{Semantic Web Roadmap}, Tim Berners-Lee, 1998.}
\begin{figure}
    \includegraphics[width=250px]{lod-cloud_colored_1000px.png} 
    \caption{Linked Open Data Cloud}
\end{figure}
\end{frame}

%TODO usare dopo
% \begin{frame}
% \frametitle{Linked Open Data Cloud (2/2)}
% Nel \emph{Linked Open Data Cloud} sono presenti 365 dataset (fonte \url{http://stats.lod2.eu/}).
% \vspace{\baselineskip}
% 
% Alcuni dataset:
% \begin{itemize}
%  \item \emph{DBPedia} (\url{dbpedia.org}) corrispondente a \url{wikipedia.org};
%  \item \emph{Linked Movie Database} (\url{http://linkedmdb.org/}) controparte sul Web Semantico di \emph{Internet Movie Database} 
%  (\url{http://www.imdb.com/});
%  \item \emph{Linked GeoData} (\url{http://linkedgeodata.org}) 
%   contiene i dati di \emph{OpenStreetMap} (\url{http://www.openstreetmap.org/});
%   \item \emph{AGROVOC} (\url{http://aims.fao.org/agrovoc}) \`e il dataset della FAO (\url{http://fao.org});
%   \item \emph{Europeana} (\url{http://pro.europeana.eu/linked-open-data}) contiene dati su beni culturali e tradizioni Europee. 
% \end{itemize}
% \end{frame}


% \begin{frame}
% \frametitle{Linked Data (2/2)}
% \`E possibile effettuare interrogazioni che coinvolgano diversi
% dataset (anche eterogenei).
% \vspace{\baselineskip}
% 
% Ad esempio, la seguente query pu\`o essere eseguita
% interrogando un data set contenente dati storici ed uno 
% sulle strutture ricettive:
% 
% 
% \begin{center}
% Q = “Strutture ricettive nei luoghi di nascita di Federico II e dei suoi parenti stretti.”
% \end{center}
% \end{frame}

\begin{frame}
  \frametitle{Classificazione 5 Stelle}

  Il w3c propone un modello per la qualit\`a degli open data denominato
  \emph{classificazione a 5 stelle}.\footnote{\url{http://5stardata.info}}
  Riportiamo il diagramma della classificazione 5 stelle riportato nelle linee
  guida dall'Agid.
  \vspace{\baselineskip}  
  
  \begin{figure}
     \includegraphics[width=300px]{5stelle_classificazione.png} 
    %\caption{Modello di classificazione per i metadati.\footnote{Dalle \emph{Linee guida nazionali per il patrimonio informativo pubblico (2014)}.}  
  \end{figure}
\end{frame}

\begin{frame}
  \frametitle{Classificazione 5 Stelle - Dimensioni}
  La classificazione 5 stelle viene estesa dall'AgID con alcune \emph{dimensioni}
  esplicative:
  \begin{itemize}[<+->]
   \item \emph{INFORMAZIONE} - descrive la qualit\`a dell'informazione fornita insieme ai dati;
   \item \emph{ACCESSO} -  descrive la facilit\`a con cui utenti e programmi riescono ad accedere ai dati;
   \item \emph{SERVIZI} - riguarda le tipologie e l'\emph{efficienza} dei servizi che possono essere realizzati a partire 
   dai dati.
  \end{itemize}
\end{frame}

  
\begin{frame}
  \frametitle{Classificazione 5 Stelle - Una Stella}
  
  Una Stella: Open Licence
  
  \begin{figure}
     \includegraphics[width=180px]{stella1.png} 
    %\caption{Modello di classificazione per i metadati.\footnote{Dalle \emph{Linee guida nazionali per il patrimonio informativo pubblico (2014)}.}  
  \end{figure}
  
  La \emph{prima stella} si ottiene rilasciando i dati in qualunque formato
  ma con una \emph{licenza aperta}.
  Rientrano in questa categoria ad esempio le scansioni dei documenti.
  \vspace{\baselineskip}

  \begin{itemize}[<+->]
   \item \emph{INFORMAZIONE: documenti} - i dati sono incorporati all’interno di documenti senza struttura;
   \item \emph{ACCESSO: solo umano} - solo gli umani sono in grado di leggere i documenti senza struttura 
   e quindi dare un senso ai dati in esso presenti;
   \item \emph{SERVIZI: nessuno} può essere abilitato a meno di significativi interventi umani 
   di estrazione ed elaborazione;
  \end{itemize}
\end{frame}

\begin{frame}
  \frametitle{Classificazione 5 Stelle - Due Stelle}
  
    Due Stelle: Open Licence, (Machine) Readable

  \begin{figure}
     \includegraphics[width=180px]{stella2.png} 
    %\caption{Modello di classificazione per i metadati.\footnote{Dalle \emph{Linee guida nazionali per il patrimonio informativo pubblico (2014)}.}  
  \end{figure}
  
  La \emph{seconda stella} si ottiene se i dati sono forniti in un formato leggibile 
  da un agente automatico. 
  
  Rientrano in questa categoria ad esempio i files in formato \emph{excel}.
  \vspace{\baselineskip}

  \begin{itemize}[<+->]
   \item \emph{INFORMAZIONE: dati grezzi (o semi-strutturati)} - i dati sono leggibili anche da un programma 
   ma necessita un intervento umano per interpretarli;
   \item \emph{ACCESSO: umano e semi-automatico} - i software possono leggere i dati ma non sono in grado di
   interpretarli automaticamente;
   \item \emph{SERVIZI: non efficienti} - servizi realizzati ad-hoc e devono incorporare al loro
  interno i dati;
  \end{itemize}
\end{frame}

\begin{frame}
  \frametitle{Classificazione 5 Stelle - Tre Stelle}
  
  Tre Stelle: Open Licence, (Machine) Readable, Open Format

  \begin{figure}
     \includegraphics[width=180px]{stella3.png} 
    %\caption{Modello di classificazione per i metadati.\footnote{Dalle \emph{Linee guida nazionali per il patrimonio informativo pubblico (2014)}.}  
  \end{figure}
  
  La \emph{terza stella} viene attribuita se i dati sono rilasciati in un formato \emph{aperto}.
  Rientrano in questa categoria ad esempio i files \emph{jsono}, \emph{csv}, \emph{xml}.
  \vspace{\baselineskip}

  \begin{itemize}
   \item \emph{INFORMAZIONE: dati grezzi (o semi-strutturati)} - i dati sono leggibili anche da un programma 
   ma necessita un intervento umano per interpretarli;
   \item \emph{ACCESSO: umano e semi-automatico} - i software possono leggere i dati ma non sono in grado di
   interpretarli automaticamente;
   \item \emph{SERVIZI: non efficienti} - servizi realizzati ad-hoc e devono incorporare al loro
  interno i dati;
  \end{itemize}
\end{frame}

\begin{frame}
  \frametitle{Classificazione 5 Stelle - Quattro Stelle}
  
  Quattro Stelle: Open Licence, (Machine) Readable, Open Format, URI

  \begin{figure}
     \includegraphics[width=180px]{stella4.png} 
    %\caption{Modello di classificazione per i metadati.\footnote{Dalle \emph{Linee guida nazionali per il patrimonio informativo pubblico (2014)}.}  
  \end{figure}
  
  La \emph{quarta stella} si ottiene esponendo i dati con le tecnologie del web semantico (RDF e SPARQL).
  \vspace{\baselineskip}

  \begin{itemize}[<+->]
   \item \emph{INFORMAZIONE: dati arricchiti semanticamente} - i dati sono descritti usando tecnologie del Web Semantico;
   \item \emph{ACCESSO: umano e automatico} - i software sono in grado di elaborare i dati quasi senza ulteriori 
   interventi umani (livelli 4 e 5);
   \item \emph{SERVIZI: efficienti} - servizi che sfruttano accessi diretti a Web per reperire i dati.
  \end{itemize}
\end{frame}

\begin{frame}
  \frametitle{Classificazione 5 Stelle - Cinque Stelle}
  
  Cinque Stelle: Open Licence, (Machine) Readable, Open Format, URI, \emph{Linked} Data

  \begin{figure}
     \includegraphics[width=180px]{stella5.png} 
    %\caption{Modello di classificazione per i metadati.\footnote{Dalle \emph{Linee guida nazionali per il patrimonio informativo pubblico (2014)}.}  
  \end{figure}
  
  La \emph{quinta stella} viene attribuita quando i dati contengono riferimenti a dataset di terze parti.
  \vspace{\baselineskip}

  \begin{itemize}[<+->]
   \item \emph{INFORMAZIONE: dati arricchiti semanticamente} - i dati sono descritti usando tecnologie del Web Semantico;
   \item \emph{ACCESSO: umano e automatico} - i software sono in grado di elaborare i dati quasi senza ulteriori 
   interventi umani (livelli 4 e 5);
   \item \emph{SERVIZI: efficienti e con mashup di dati} - servizi che sfruttano sia accessi diretti a Web 
   sia l'informazione ulteriore catturata attraverso i \emph{link} dei dati di interesse.
  \end{itemize}
\end{frame}

\section{Ontologie}
\begin{frame}
\frametitle{Ontologie}
I dataset del Web Semantico vengono spesso definiti \emph{ontologie}.
\vspace{\baselineskip}

Una \emph{ontologia} \`e una descrizione \emph{parziale} del mondo:
\begin{itemize}
 \item descrive una porzione del mondo, spesso \`e limitata ad un'unico \emph{dominio di conoscenza};
 \item non si assume che i fatti non esplicitamente presenti nell'ontologia siano falsi (\emph{Open World Assumption}).
\end{itemize}
\vspace{\baselineskip}

Essa \`e costituita da un insieme finito di \emph{affermazioni}. Ad esempio:
\begin{itemize}
 \item Tutti gli esseri umani sono mortali;
 \item Socrate \`e mortale;
 \item Alice \`e la madre di Roberto.
\end{itemize}
\end{frame}

\begin{frame}
\frametitle{Ontologie - Affermazioni}
Le affermazioni contenute in una ontologia sono di tre tipi:
\vspace{\baselineskip}

\emph{Constraints:} impongono dei vincoli \emph{semantici} sul dominio di conoscenza 
che si va a rappresentare. La notazione richiama quella insiemistica;
\vspace{\baselineskip}
\[
 HumanBeing \Issub Mortal 
\]
\emph{Property Assertions}: impongono una relazione tra due elementi del dominio;
\[
 Alice\,motherOf\,Bob 
\]
\emph{Class Assertions}: indicano l'appartenenza di un elemento ad un insieme.
\[
 HumanBeing(Socrate) 
\]
\end{frame}

\begin{frame}
\frametitle{Ontologie - Reasoning}

Con il termine \emph{reasoning} si intende l'attivit\`a di estrazione di 
conoscenza \emph{implicita} in una ontologia.
\[
\begin{array}{ccc}
\left \{ \begin{array}{l}
HumanBeing \Issub Mortal , \\
HumanBeing(Socrate)  
\end{array} \right \}&\Longrightarrow &  Mortal(Socrate)\\
\end{array}
\]
Le attivit\`a di reasoning sono rese possibili dalle \emph{semantiche formali}
associate ai linguaggi di rappresentazione utilizzati.
\end{frame}


\newcommand{\CNames}{N_C}
\newcommand{\PNames}{N_P}
\newcommand{\INames}{N_I}
\newcommand{\VNames}{V}

\begin{frame}
\frametitle{Ontologie - Definizione}

Siano $\CNames$, $\PNames$, $\INames$ tre insiemi infiniti, numerabili e 
a due a due disgiunti di nomi di \emph{classe}, \emph{propriet\`a} e \emph{individuo},
rispettivamente.
\vspace{\baselineskip}

Una \emph{ontologia} \`e un insieme finito di asserzioni dei seguenti tipi:
\[
 \begin{array}{l|l|l}
  & Sintassi & Semantica \\
  \hline
  &&\\
  \mbox{Constraints} & C \Issub D & (\forall x)(x \in C \rightarrow x \in D) \\
  & R \Issub S & (\forall x, y)([x, y] \in R \rightarrow [x,y] \in S) \\
  & \dom(R) \Issub C & (\forall x, y)([x, y] \in R \rightarrow x \in C) \\
  & \range(R) \Issub C & (\forall x, y)([x, y] \in R \rightarrow y \in C) \\
  &&\\
  \hline
  &&\\
  \mbox{Class Assertions} & C(a) & a \in C\\
  &&\\
  \hline
  &&\\
  \mbox{Property Assertions} & a\,P\,b\;(\mbox{equivalente }P(a,b)) & [a,b] \in P \\
  &&\\
  \hline  
 \end{array}
\]
dove $C, D \in \CNames$, $R, S \in \PNames$ e $a, b \in \INames$.
\end{frame}

\newcommand{\Ont}{\mathcal{O}}
\newcommand{\Ontp}{\mathcal{O'}}

\begin{frame}
\frametitle{Ontologie - Esempio}
Riportiamo un esempio di ontologia. Siano $HumanBeing, Mortal \in \CNames$,
$teacherOf \in \PNames$, $Socrate, Platone \in \INames$.
\vspace{\baselineskip}

\phantom{Mediante \emph{reasoning} \`e possibile esplicitare ulteriori 
affermazioni.}
\[
 \begin{array}{clccl}
  \Ont  =  &\{HumanBeing \Issub Mortal, & &\phantom{\Ontp =} & \phantom{\{Mortal(Socrate), }\\
  &\phantom{\{}\range(teacherOf) \Issub HumanBeing, & \phantom{\Longrightarrow} && \phantom{\{HumanBeing(Platone)}\\
  &\phantom{\{}HumanBeing(Socrate), &&&\phantom{\{Mortal(Platone) \}}\\
  &\phantom{\{}Socrate\,teacherOf\,Platone \}\\
 \end{array}
\]
\end{frame}

\begin{frame}
\frametitle{Ontologie - Esempio}
Riportiamo un esempio di ontologia. Siano $HumanBeing, Mortal \in \CNames$,
$teacherOf \in \PNames$, $Socrate, Platone \in \INames$.
\vspace{\baselineskip}

Mediante \emph{reasoning} \`e possibile esplicitare ulteriori 
affermazioni.
\[
 \begin{array}{clccl}
  \Ont  =  &\{\mathbf{HumanBeing \Issub Mortal}, & &\Ontp = & \{\mathbf{Mortal(Socrate)}, \\
  &\phantom{\{}\range(teacherOf) \Issub HumanBeing, & \Longrightarrow&& \phantom{\{HumanBeing(Platone)}\\
  &\phantom{\{}\mathbf{HumanBeing(Socrate)}, &&&\phantom{\{Mortal(Platone)}\}\\
  &\phantom{\{}Socrate\,teacherOf\,Platone \}\\
 \end{array}
\]
\end{frame}

\begin{frame}
\frametitle{Ontologie - Esempio}
Riportiamo un esempio di ontologia. Siano $HumanBeing, Mortal \in \CNames$,
$teacherOf \in \PNames$, $Socrate, Platone \in \INames$.
\vspace{\baselineskip}

Mediante \emph{reasoning} \`e possibile esplicitare ulteriori 
affermazioni.
\[
 \begin{array}{clccl}
  \Ont  =  &\{HumanBeing \Issub Mortal, & &\Ontp = & \{Mortal(Socrate), \\
  &\phantom{\{}\mathbf{\range(teacherOf) \Issub HumanBeing}, & \Longrightarrow&& \phantom{\}}\mathbf{HumanBeing(Platone)},\\
  &\phantom{\{}HumanBeing(Socrate), &&&\phantom{\{Mortal(Platone)}\}\\
  &\phantom{\{}\mathbf{Socrate\,teacherOf\,Platone} \}\\
 \end{array}
\]
\end{frame}

\begin{frame}
\frametitle{Ontologie - Esempio}
Riportiamo un esempio di ontologia. Siano $HumanBeing, Mortal \in \CNames$,
$teacherOf \in \PNames$, $Socrate, Platone \in \INames$.
\vspace{\baselineskip}

Mediante \emph{reasoning} \`e possibile esplicitare ulteriori 
affermazioni.
\[
 \begin{array}{clccl}
  \Ont  =  &\{\mathbf{HumanBeing \Issub Mortal} & &\Ontp = & \{Mortal(Socrate), \\
  &\phantom{\{}\range(teacherOf) \Issub HumanBeing, & \Longrightarrow && \phantom{\{}\mathbf{HumanBeing(Platone)}\\
  &\phantom{\{}HumanBeing(Socrate), &&&\phantom{\{}\mathbf{Mortal(Platone)} \}\\
  &\phantom{\{}Socrate\,teacherOf\,Platone \}\\
 \end{array}
\]
\end{frame}

\begin{frame}[fragile]
\frametitle{Ontologie nel Web Semantico}

Le ontologie definite usando tecnologie del Web Semantico hanno particolari caratteristiche.

Innanzitutto, tutti i nomi sono \emph{URI}:
\[
 \CNames \cup \PNames \cup \INames \subseteq URI .
\]

\uncover<2->{
  Possono contenere dei \emph{letterali}, che vengono usati per rappresentare tipi di dato \emph{concreti},
  come ad esempio stringhe di testo, numeri, date, \ldots
  \vspace{\baselineskip}
}

\uncover<3->{
  I letterali possono comparire come \emph{oggetto}
  di una property assertion. Alcuni esempi di role assertion che coinvolgono
  letterali sono
  \[
  \begin{array}{l}
    Mario \, surname \, ``Rossi'' \\
    Cristiano \, hasbirth \, ``March-22-1979''\\
  \end{array}
  \]
  con $Mario, Cristiano \in \INames$, $surname, hasbirth \in \PNames$ e 
  $``Rossi''$, $``March-22-1979''$ letterali.
  \vspace{\baselineskip}
}

\uncover<4->{
   \`E possibile vincolare una propriet\`a  ad avere come codominio
   solo \emph{letterali} con un certo data type (\emph{datatype property}).
    \[
    \begin{array}{l}
      \range(surname) \Issub xsd:string \\
      \range(hasbirth) \Issub xsd:date\\
    \end{array}
    \]
}
\end{frame}

\section{Vocabolari}
\begin{frame}
\frametitle{Vocabolari}
Classi e propriet\`a vengono raggruppati in \emph{vocabolari}
che trattano specifici domini di conoscenza (eg. organizzazioni, 
pubblica amministrazione, biologia, commercio, etc.). 
\vspace{\baselineskip}

Un vocabolario
pu\`o contenere anche alcuni vincoli sulle classi e le propriet\`a 
del vocabolario stesso.
\vspace{\baselineskip}

\end{frame}

\begin{frame}
\frametitle{Definizione di Vocabolario}
Una definizione di vocabolario pu\`o essere la seguente:
\[
 V = (C, P, \Omega)
\]
dove
\begin{enumerate}
 \item $C$ \`e un sottoinsieme finito di $\CNames$,
 \item $P$ \`e un sottoinsieme finito di $\PNames$,
 \item $\Omega$ \`e un insieme finito di vincoli che coinvolgano solo
 nomi di classi in $C$ e nomi di propriet\`a in $P$.
\end{enumerate}
\end{frame}

\begin{frame}
\frametitle{Vocabolari condivisi}

L'utilizzo di vocabolari condivisi (ben noti) favorisce la 
scalabilit\`a orizzontale delle applicazioni. 
\vspace{\baselineskip}

Ad esempio, una 
applicazione sviluppata sull'ontologia di un comune che utilizzi 
i vocabolari standard per le pubbliche amministrazioni (vedi le \emph{Linee 
Guida per la Valorizzazione del Patrimonio Informativo Pubblico} dell'\emph{Agenzia per l'Italia Digitale}) 
pu\`o essere estesa senza sforzi aggiuntivi per utilizzare i dati provenienti
dalle ontologie di tutti i comuni.
\end{frame}


\begin{frame}
\frametitle{Il Vocabolario FOAF}
Uno dei primi e pi\`u utilizzati vocabolari definiti nell'ambito
del Web semantico \`e \emph{Friend OF A Friend} (\emph{FOAF}, vedi \url{http://foaf-project.org}).
\vspace{\baselineskip}


\begin{quote}
FOAF is a project devoted to linking people and information using the Web. 
\end{quote}

In questa sede ci limiteremo solo alla parte \emph{Core}.
\begin{quote}
\textbf{Core} - These classes and properties form the core of FOAF. They
describe characteristics of people and social groups that are
independent of time and technology; as such they can be used to
describe basic information about people in present day, historical,
cultural heritage and digital library contexts. In addition to various
characteristics of people, FOAF defines classes for Project,
Organization and Group as other kinds of agent.
\end{quote}
(tratto da \emph{FOAF Vocabulary Specification 0.99},
Namespace Document 14 January 2014, Paddington Edition, \url{http://xmlns.com/foaf/spec/})
\end{frame}

\newcommand{\foafcore}{\mathtt{FOAFCore}}
\begin{frame}
\frametitle{FOAF Core}
Il vocabolario \emph{Foaf Core} \`e definito come segue:
\[
 \begin{array}{ccl}
 \foafcore & \defAs & (C_{foaf}, P_{foaf}, \Omega_{foaf}) \\
 &&\\
 C_{foaf} & \defAs & \{Agent, Person, Project, Organization, Group, Document, Image\}\\
 &&\\
 P_{foaf} & \defAs & \{name, title, img, depiction, depicts, familyName, givenName,\\
 && \phantom{\}} based\_near, age, made, maker, primaryTopic, primaryTopicOf, member \}\\
 \Omega_{foaf} & \defAs & \{ Person \Issub Agent, Group \Issub Agent, Organization \Issub Agent, \\
 && \phantom{\{} Image \Issub Document, \dom(title) \Issub Document, \\
 && \phantom{\{}\range(depiction) \Issub Image, img \Issub depiction, \dom(img) \Issub Person,\\
 && \phantom{\{}\dom(knows) \Issub Person, \range(knows) \Issub Person, \ldots \}\\
 \end{array}
\]
\end{frame}

\begin{frame}
\frametitle{Descrizioni Intuitive degli Elementi dei Vocabolari}
Le classi e le propriet\`a di un vocabolario vengono spesso fornite
di una descrizione intuitiva nel documento che descrive il vocabolario.
Ad esempio, le classi $Agent$ e $Person$ vengono descritte come segue 
in \url{http://xmlns.com/foaf/spec/}
\begin{quote}
\textbf{Agent} - The Agent class is the class of agents; things that do stuff. A well known 
sub-class is Person, representing people. Other kinds of agents include 
Organization and Group.

The Agent class is useful in a few places in FOAF where Person would have 
been overly specific. For example, the IM chat ID properties such as 
jabberID are typically associated with people, but sometimes belong to 
software bots.  
\end{quote}

\begin{quote}
\textbf{Person} - The Person class represents people. Something is a Person if it is a person.
We don't nitpic about whether they're alive, dead, real, or imaginary. 
The Person class is a sub-class of the Agent class, since all people are 
considered 'agents' in FOAF.  
\end{quote}
\end{frame}

\begin{frame}
\frametitle{Descrizioni Rigorose degli Elementi dei Vocabolari}
Tuttavia, gi\`a nei vincoli di un vocabolario si trovano indicazioni importanti
sulla \emph{semantica} dei nomi di classe e di propriet\`a del vocabolario stesso.
\[
\begin{array}{l}
Person \Issub Agent\\
\range(depiction) \Issub Image\\
img \Issub depiction, \\
\dom(img) \Issub Person,\\
\dom(knows) \Issub Person,\\
\range(knows) \Issub Person,\\
\ldots
\end{array}
\]
\end{frame}

\section{Interrogazioni}

\begin{frame}
\frametitle{Interrogazioni}

Il metodo pi\`u immediato per ottenere informazioni da una ontologia \`e il 
\emph{Conjunctive Query Answering}. Consideriamo ad esempio la seguente ontologia:

\begin{tabular}{cc}
\hline
$\begin{array}{cl}
  \Ont  =  &  \{Female(Elise), Female(Alice), Male(Bob), \\
  &\phantom{\{}Male(Charlie), Male(Daniel), \\
  &\phantom{\{}Alice\,childOf\,Elise, Charlie\,childOf\,Elise, \\
  &\phantom{\{}Daniel\,childOf\,Alice, Daniel\,childOf\,Bob, \\
  &\phantom{\{}Francis\,childOf\,Charlie \}
 \end{array}$ & \includegraphics[width=120px]{family.png} \\
\hline
\end{tabular}

Alcune interrogazioni che \`e possibile effettuare con il conjunctive query 
answering sono:
\begin{itemize}
 \item ``Trova tutti gli individui maschi.''
 \item ``Chi sono gli individui con almeno un figlio maschio?''
 \item ``Chi sono i figli di $Alice$?''
 \item ``Chi sono gli individui con almeno un figlio maschio ed una femmina?''
 \item ``Chi sono gli individui maschi con almeno un figlio maschio?''
\end{itemize}
\end{frame}

\begin{frame}
\frametitle{Formule Atomiche}

Per definire in maniera rigorosa le query congiuntive \`e necessario
definire preliminarmente l'insieme delle \emph{formule atomiche}. 
\vspace{\baselineskip}

Sia $\VNames = \{x, y, z, ... \}$ l'insieme infinito, numerabile 
e disgiunto da $\CNames$, $\PNames$ e $\INames$ delle \emph{variabili}.
Le \emph{formule atomiche} sono espressioni dei due seguenti tipi:
\[
 C(x), \quad P(x, y)
\]
con $x, y \in \INames \cup \VNames$, $C \in \CNames$ e $P \in \PNames$.
\vspace{\baselineskip}

Esempi di formule atomiche sono:
\begin{itemize}
 \item $HumanBeing(x)$,
 \item $x\,childOf\,Alice$,
 \item $Bob\,childOf\,x$,
 \item $x\,childOf\,y$,
 \item $Mortal(Socrate)$,
 \item $Alice\,childOf\,Elise$
\end{itemize}
con $HumanBeing, Mortal \in \CNames$, $childOf \in \PNames$, 
$Alice, Bob, Elise \in \INames$ e $x, y \in \VNames$.
\end{frame}

\begin{frame}
\frametitle{Formule Atomiche Chiuse}
Una formula atomica nella quale non compaiano variabili si dice \emph{chiusa}.
\vspace{\baselineskip}

Negli esempi che seguono sono evidenziate le formule atomiche chiuse:
\begin{itemize}
 \item $HumanBeing(x)$,
 \item $x\,childOf\,Alice$,
 \item $Bob\,childOf\,x$,
 \item $x\,childOf\,y$,
 \item $Mortal(Socrate)$,
 \item $Alice\,childOf\,Elise$
\end{itemize}
con $HumanBeing, Mortal \in \CNames$, $childOf \in \PNames$, 
$Alice, Bob, Elise \in \INames$ e $x, y \in \VNames$.
\vspace{\baselineskip}

\phantom{Le asserzioni presenti nelle ontologie sono formule atomiche chiuse.}
\end{frame}

\begin{frame}
\frametitle{Formule Atomiche Chiuse}
Una formula atomica nella quale non compaiano variabili si dice \emph{chiusa}.
\vspace{\baselineskip}

Negli esempi che seguono sono evidenziate le formule atomiche chiuse:
\begin{itemize}
 \item $HumanBeing(x)$,
 \item $x\,childOf\,Alice$,
 \item $Bob\,childOf\,x$,
 \item $x\,childOf\,y$,
 \item $\mathbf{Mortal(Socrate)}$,
 \item $\mathbf{Alice\,childOf\,Elise}$
\end{itemize}
con $HumanBeing, Mortal \in \CNames$, $childOf \in \PNames$, 
$Alice, Bob, Elise \in \INames$ e $x, y \in \VNames$.
\vspace{\baselineskip}

Le asserzioni presenti nelle ontologie sono formule atomiche chiuse.
\end{frame}

\begin{frame}
\frametitle{Query Congiuntive}
Una \emph{query congiuntiva} \`e una congiunzione finita di formule atomiche $T_1 \wedge \ldots \wedge T_n$.
\vspace{\baselineskip}

Alcuni esempi di query congiuntive:
\begin{itemize}
 \item ``Trova tutti gli individui maschi.''
 \[
  Male(x)
 \]
 \item ``Chi sono gli individui con almeno un figlio maschio?''
\[
 y\,childOf\,x\,\wedge\,Male(y)  
\]
 \item ``Chi sono i figli di $Alice$?''
\[
 x\,childOf\,Alice
\]
%  \item ``Chi sono gli individui con almeno un figlio maschio ed una femmina?''
% \[
%  y\,childOf\,x\,\wedge\,Male(y)\,\wedge\,z\,childOf\,x\,\wedge\,Female(z)   
% \]
%  \item ``Chi sono gli individui maschi con almeno un figlio maschio?''
% \[
%  Male(x)\,\wedge\,y\,childOf\,x\,\wedge\,Male(y)
% \]
con $x, y \in \VNames$, $Male, Female \in \CNames$, $childOf \in \PNames$ e $Alice \in \INames$.
 \end{itemize}
\end{frame}

\begin{frame}
\frametitle{Sostituzioni (1/2)}

Per definire le \emph{soluzioni} (risposte) delle query congiuntive introduciamo la
nozione di \emph{sostituzione}.
\vspace{\baselineskip}

Una \emph{sostituzione} $\sigma=[x_1 \rightarrow a_1, \ldots, x_n \rightarrow a_n]$
($x_1, \ldots, x_n \in \VNames$, $a_1, \ldots, a_n \in \INames$)
\`e una mappa finita che associa nomi di individui a variabili.
\vspace{\baselineskip}

Sia $T$ una formula atomica e $\sigma=[x_1 \rightarrow a_1, \ldots, x_n \rightarrow a_n]$
una sostituzione. L'\emph{applicazione} $T\sigma$ di $\sigma$ a $T$ \`e la formula atomica 
che si ottiene sostituendo in $T$ ad ogni occorrenza della variabile $x_i$ il
corrispondente nome di individuo $a_i$, per ogni $1\leq i\leq n$.
\vspace{\baselineskip}

Alcuni esempi:
\[
 \begin{array}{lcl}
  Male(x)[x \rightarrow Bob] & = & Male(Bob) \\
  Male(x)[y \rightarrow Bob] & = & \phantom{Male(x)} \\
  (x\,childOf\,y)[x \rightarrow Alice] & = & \phantom{Alice\,childOf\,y} \\
  (x\,childOf\,y)[x \rightarrow Alice, y \rightarrow Elise] & = & \phantom{Alice\,childOf\,Elise}
 \end{array}
\]
con $x,y \in \VNames$, $Male \in \CNames$, $childOf \in \PNames$ e $Alice, Bob, Elise \in \INames$.
\end{frame}

\begin{frame}
\frametitle{Sostituzioni (1/2)}

Per definire le \emph{soluzioni} (risposte) delle query congiuntive introduciamo la
nozione di \emph{sostituzione}.
\vspace{\baselineskip}

Una \emph{sostituzione} $\sigma=[x_1 \rightarrow a_1, \ldots, x_n \rightarrow a_n]$
($x_1, \ldots, x_n \in \VNames$, $a_1, \ldots, a_n \in \INames$)
\`e una mappa finita che associa nomi di individui a variabili.
\vspace{\baselineskip}

Sia $T$ una formula atomica e $\sigma=[x_1 \rightarrow a_1, \ldots, x_n \rightarrow a_n]$
una sostituzione. L'\emph{applicazione} $T\sigma$ di $\sigma$ a $T$ \`e la formula atomica 
che si ottiene sostituendo in $T$ ad ogni occorrenza della variabile $x_i$ il
corrispondente nome di individuo $a_i$, per ogni $1\leq i\leq n$.
\vspace{\baselineskip}

Alcuni esempi:
\[
 \begin{array}{lcl}
  Male(x)[x \rightarrow Bob] & = & Male(Bob) \\
  Male(x)[y \rightarrow Bob] & = & Male(x) \\
  (x\,childOf\,y)[x \rightarrow Alice] & = & \phantom{Alice\,childOf\,y} \\
  (x\,childOf\,y)[x \rightarrow Alice, y \rightarrow Elise] & = & \phantom{Alice\,childOf\,Elise}
 \end{array}
\]
con $x,y \in \VNames$, $Male \in \CNames$, $childOf \in \PNames$ e $Alice, Bob, Elise \in \INames$.
\end{frame}

\begin{frame}
\frametitle{Sostituzioni (1/2)}

Per definire le \emph{soluzioni} (risposte) delle query congiuntive introduciamo la
nozione di \emph{sostituzione}.
\vspace{\baselineskip}

Una \emph{sostituzione} $\sigma=[x_1 \rightarrow a_1, \ldots, x_n \rightarrow a_n]$
($x_1, \ldots, x_n \in \VNames$, $a_1, \ldots, a_n \in \INames$)
\`e una mappa finita che associa nomi di individui a variabili.
\vspace{\baselineskip}

Sia $T$ una formula atomica e $\sigma=[x_1 \rightarrow a_1, \ldots, x_n \rightarrow a_n]$
una sostituzione. L'\emph{applicazione} $T\sigma$ di $\sigma$ a $T$ \`e la formula atomica
che si ottiene sostituendo in $T$ ad ogni occorrenza della variabile $x_i$ il
corrispondente nome di individuo $a_i$, per ogni $1\leq i\leq n$.
\vspace{\baselineskip}

Alcuni esempi:
\[
 \begin{array}{lcl}
  Male(x)[x \rightarrow Bob] & = & Male(Bob) \\
  Male(x)[y \rightarrow Bob] & = & Male(x) \\
  (x\,childOf\,y)[x \rightarrow Alice] & = & Alice\,childOf\,y \\
  (x\,childOf\,y)[x \rightarrow Alice, y \rightarrow Elise] & = & \phantom{Alice\,childOf\,Elise}
 \end{array}
\]
con $x,y \in \VNames$, $Male \in \CNames$, $childOf \in \PNames$ e $Alice, Bob, Elise \in \INames$.
\end{frame}

\begin{frame}
\frametitle{Sostituzioni (1/2)}

Per definire le \emph{soluzioni} (risposte) delle query congiuntive introduciamo la
nozione di \emph{sostituzione}.
\vspace{\baselineskip}

Una \emph{sostituzione} $\sigma=[x_1 \rightarrow a_1, \ldots, x_n \rightarrow a_n]$
($x_1, \ldots, x_n \in \VNames$, $a_1, \ldots, a_n \in \INames$)
\`e una mappa finita che associa nomi di individui a variabili.
\vspace{\baselineskip}

Sia $T$ una formula atomica e $\sigma=[x_1 \rightarrow a_1, \ldots, x_n \rightarrow a_n]$
una sostituzione. L'\emph{applicazione} $T\sigma$ di $\sigma$ a $T$ \`e la formula atomica
che si ottiene sostituendo in $T$ ad ogni occorrenza della variabile $x_i$ il
corrispondente nome di individuo $a_i$, per ogni $1\leq i\leq n$.
\vspace{\baselineskip}

Alcuni esempi:
\[
 \begin{array}{lcl}
  Male(x)[x \rightarrow Bob] & = & Male(Bob) \\
  Male(x)[y \rightarrow Bob] & = & Male(x) \\
  (x\,childOf\,y)[x \rightarrow Alice] & = & Alice\,childOf\,y \\
  (x\,childOf\,y)[x \rightarrow Alice, y \rightarrow Elise] & = & Alice\,childOf\,Elise
 \end{array}
\]
con $x,y \in \VNames$, $Male \in \CNames$, $childOf \in \PNames$ e $Alice, Bob, Elise \in \INames$.
\end{frame}

\begin{frame}
\frametitle{Sostituzioni (2/2)}

L'applicazione di sostituzioni a query congiuntive si definisce come segue.
\vspace{\baselineskip}

Sia $\sigma=[x_1 \rightarrow a_1, \ldots, x_n \rightarrow a_n]$ una sostituzione
e siano $T_1, \ldots, T_m$ formule atomiche. Allora
\[
 (T_1 \wedge \ldots \wedge T_m)\sigma \defAs T_1\sigma \wedge \ldots \wedge T_m\sigma . 
\]
\vspace{\baselineskip}

Alcuni esempi:
\[
 \begin{array}{lcl}
 (y\,childOf\,x\,\wedge\,Male(y))[x \rightarrow Alice] & \defAs & \phantom{y\,childOf\,Alice\,\wedge\,Male(y)}  \\
 (y\,childOf\,x\,\wedge\,Male(y))[x \rightarrow Alice, z \rightarrow Bob] & \defAs & \phantom{y\,childOf\,Alice\,\wedge\,Male(y)}  \\
 (y\,childOf\,x\,\wedge\,Male(y))[x \rightarrow Alice, y \rightarrow Bob] & \defAs & \phantom{Bob\,childOf\,Alice\,\wedge\,Male(Bob)}  \\
 \end{array}
\]
\end{frame}

\begin{frame}
\frametitle{Sostituzioni (2/2)}

L'applicazione di sostituzioni a query congiuntive si definisce come segue.
\vspace{\baselineskip}

Sia $\sigma=[x_1 \rightarrow a_1, \ldots, x_n \rightarrow a_n]$ una sostituzione
e siano $T_1, \ldots, T_m$ formule atomiche. Allora
\[
 (T_1 \wedge \ldots \wedge T_m)\sigma \defAs T_1\sigma \wedge \ldots \wedge T_m\sigma . 
\]
\vspace{\baselineskip}

Alcuni esempi:
\[
 \begin{array}{lcl}
 (y\,childOf\,x\,\wedge\,Male(y))[x \rightarrow Alice] & \defAs & y\,childOf\,Alice\,\wedge\,Male(y)  \\
 (y\,childOf\,x\,\wedge\,Male(y))[x \rightarrow Alice, z \rightarrow Bob] & \defAs & \phantom{y\,childOf\,Alice\,\wedge\,Male(y)}  \\
 (y\,childOf\,x\,\wedge\,Male(y))[x \rightarrow Alice, y \rightarrow Bob] & \defAs & \phantom{Bob\,childOf\,Alice\,\wedge\,Male(Bob)}  \\
 \end{array}
\]
\end{frame}

\begin{frame}
\frametitle{Sostituzioni (2/2)}

L'applicazione di sostituzioni a query congiuntive si definisce come segue.
\vspace{\baselineskip}

Sia $\sigma=[x_1 \rightarrow a_1, \ldots, x_n \rightarrow a_n]$ una sostituzione
e siano $T_1, \ldots, T_m$ formule atomiche. Allora
\[
 (T_1 \wedge \ldots \wedge T_m)\sigma \defAs T_1\sigma \wedge \ldots \wedge T_m\sigma . 
\]
\vspace{\baselineskip}

Alcuni esempi:
\[
 \begin{array}{lcl}
 (y\,childOf\,x\,\wedge\,Male(y))[x \rightarrow Alice] & \defAs & y\,childOf\,Alice\,\wedge\,Male(y)  \\
 (y\,childOf\,x\,\wedge\,Male(y))[x \rightarrow Alice, z \rightarrow Bob] & \defAs & y\,childOf\,Alice\,\wedge\,Male(y)  \\
 (y\,childOf\,x\,\wedge\,Male(y))[x \rightarrow Alice, y \rightarrow Bob] & \defAs & \phantom{Bob\,childOf\,Alice\,\wedge\,Male(Bob)}  \\
 \end{array}
\]
\end{frame}

\begin{frame}
\frametitle{Sostituzioni (2/2)}

L'applicazione di sostituzioni a query congiuntive si definisce come segue.
\vspace{\baselineskip}

Sia $\sigma=[x_1 \rightarrow a_1, \ldots, x_n \rightarrow a_n]$ una sostituzione
e siano $T_1, \ldots, T_m$ formule atomiche. Allora
\[
 (T_1 \wedge \ldots \wedge T_m)\sigma \defAs T_1\sigma \wedge \ldots \wedge T_m\sigma . 
\]
\vspace{\baselineskip}

Alcuni esempi:
\[
 \begin{array}{lcl}
 (y\,childOf\,x\,\wedge\,Male(y))[x \rightarrow Alice] & \defAs & y\,childOf\,Alice\,\wedge\,Male(y)  \\
 (y\,childOf\,x\,\wedge\,Male(y))[x \rightarrow Alice, z \rightarrow Bob] & \defAs & y\,childOf\,Alice\,\wedge\,Male(y)  \\
 (y\,childOf\,x\,\wedge\,Male(y))[x \rightarrow Alice, y \rightarrow Bob] & \defAs & Bob\,childOf\,Alice\,\wedge\,Male(Bob)  \\
 \end{array}
\]
\end{frame}

\begin{frame}
\frametitle{Soluzioni per una Query}
Siano $\sigma=[x_1 \rightarrow a_1, \ldots, x_n \rightarrow a_n]$ una sostituzione,
$Q=T_1 \wedge \ldots \wedge T_m$ una query congiuntiva e $\Ont$ una ontologia.
\vspace{\baselineskip}

$\sigma$ \`e detta essere una \emph{soluzione} per $Q$ rispetto ad $\Ont$ se
e solo se $T_1\sigma, \ldots, T_2\sigma$ compaiono in $\Ont$. 
\vspace{\baselineskip}

Consideriamo l'ontologia $\Ont$ e la query $Q$ definite come segue:
\[
\begin{array}{ll}
  \begin{array}{cl}
    \Ont  =  &  \{Female(Elise), Female(Alice), Male(Bob), \\
    &\phantom{\{}Male(Charlie), Male(Daniel), \\
    &\phantom{\{}Alice\,childOf\,Elise, Charlie\,childOf\,Elise, \\
    &\phantom{\{}Daniel\,childOf\,Alice, Daniel\,childOf\,Bob, \\
    &\phantom{\{}Francis\,childOf\,Charlie \}\\
    &\\
    Q = & y\,childOf\,x\,\wedge\,Male(y) 
  \end{array} & 
  \includegraphics[width=130px]{family_ex1.png} \\
 \end{array}
\]
Sia $\sigma_1=[x \rightarrow Alice, y \rightarrow Daniel]$. $\sigma_1$ \`e una soluzione per $Q$
rispetto ad $\Ont$?
\end{frame}

\begin{frame}
\frametitle{Soluzioni per una Query - Esempio 1}
Siano $\sigma=[x_1 \rightarrow a_1, \ldots, x_n \rightarrow a_n]$ una sostituzione,
$Q=T_1 \wedge \ldots \wedge T_m$ una query congiuntiva e $\Ont$ una ontologia.
\vspace{\baselineskip}

$\sigma$ \`e detta essere una \emph{soluzione} per $Q$ rispetto ad $\Ont$ se
e solo se $T_1\sigma, \ldots, T_2\sigma$ compaiono in $\Ont$. 
\vspace{\baselineskip}

Consideriamo l'ontologia $\Ont$ e la query $Q$ (``Chi sono gli individui con almeno un figlio maschio?'') definite come segue:
\[
\begin{array}{ll}
  \begin{array}{cl}
    \Ont  =  &  \{Female(Elise), Female(Alice), Male(Bob), \\
    &\phantom{\{}Male(Charlie), \mathbf{Male(Daniel)}, \\
    &\phantom{\{}Alice\,childOf\,Elise, Charlie\,childOf\,Elise, \\
    &\phantom{\{}\mathbf{Daniel\,childOf\,Alice}, Daniel\,childOf\,Bob, \\
    &\phantom{\{}Francis\,childOf\,Charlie \}\\
    &\\
    Q = & y\,childOf\,x\,\wedge\,Male(y) 
  \end{array} & 
  \includegraphics[width=130px]{family_ex1.png} \\
 \end{array}
\]
Sia $\sigma_1=[x \rightarrow Alice, y \rightarrow Daniel]$. $\sigma_1$ \`e una soluzione per $Q$
rispetto ad $\Ont$? \textbf{SI}.
\[
 Q\sigma_1 = Daniel\,childOf\,Alice\,\wedge\,Male(Daniel) .
\]
\end{frame}

\begin{frame}
\frametitle{Soluzioni per una Query - Esempio 2}
Siano $\sigma=[x_1 \rightarrow a_1, \ldots, x_n \rightarrow a_n]$ una sostituzione,
$Q=T_1 \wedge \ldots \wedge T_m$ una query congiuntiva e $\Ont$ una ontologia.
\vspace{\baselineskip}

$\sigma$ \`e detta essere una \emph{soluzione} per $Q$ rispetto ad $\Ont$ se
e solo se $T_1\sigma, \ldots, T_2\sigma$ compaiono in $\Ont$. 
\vspace{\baselineskip}

Consideriamo l'ontologia $\Ont$ e la query $Q$ definite come segue:
\[
\begin{array}{ll}
  \begin{array}{cl}
    \Ont  =  &  \{Female(Elise), Female(Alice), Male(Bob), \\
    &\phantom{\{}Male(Charlie), Male(Daniel), \\
    &\phantom{\{}Alice\,childOf\,Elise, Charlie\,childOf\,Elise, \\
    &\phantom{\{}Daniel\,childOf\,Alice, Daniel\,childOf\,Bob, \\
    &\phantom{\{}Francis\,childOf\,Charlie \}\\
    &\\
    Q = & y\,childOf\,x\,\wedge\,Male(y) .
  \end{array} & 
  \includegraphics[width=130px]{family_ex2.png} \\
 \end{array}
\]
Sia $\sigma_2=[x \rightarrow Alice, y \rightarrow Bob]$. $\sigma_2$ \`e una soluzione per $Q$
rispetto ad $\Ont$?
\end{frame}

\begin{frame}
\frametitle{Soluzioni per una Query - Esempio 2}
Siano $\sigma=[x_1 \rightarrow a_1, \ldots, x_n \rightarrow a_n]$ una sostituzione,
$Q=T_1 \wedge \ldots \wedge T_m$ una query congiuntiva e $\Ont$ una ontologia.
\vspace{\baselineskip}

$\sigma$ \`e detta essere una \emph{soluzione} per $Q$ rispetto ad $\Ont$ se
e solo se $T_1\sigma, \ldots, T_2\sigma$ compaiono in $\Ont$. 
\vspace{\baselineskip}

Consideriamo l'ontologia $\Ont$ e la query $Q$ (``Chi sono gli individui con almeno un figlio maschio?'') definite come segue:
\[
\begin{array}{ll}
  \begin{array}{cl}
    \Ont  =  &  \{Female(Elise), Female(Alice), Male(Bob), \\
    &\phantom{\{}Male(Charlie), Male(Daniel), \\
    &\phantom{\{}Alice\,childOf\,Elise, Charlie\,childOf\,Elise, \\
    &\phantom{\{}Daniel\,childOf\,Alice, Daniel\,childOf\,Bob, \\
    &\phantom{\{}Francis\,childOf\,Charlie \}\\
    &\\
    Q = & y\,childOf\,x\,\wedge\,Male(y) .
  \end{array} & 
  \includegraphics[width=130px]{family_ex2.png} \\
 \end{array}
\]
Sia $\sigma_2=[x \rightarrow Alice, y \rightarrow Bob]$. $\sigma_2$ \`e una soluzione per $Q$
rispetto ad $\Ont$? \textbf{NO}.
\[
 Q\sigma_2 = \mathbf{Bob\,childOf\,Alice}\,\wedge\,Male(Bob) .
\]
\end{frame}

\begin{frame}
\frametitle{Soluzioni per una Query - Esempio 3}
Siano $\sigma=[x_1 \rightarrow a_1, \ldots, x_n \rightarrow a_n]$ una sostituzione,
$Q=T_1 \wedge \ldots \wedge T_m$ una query congiuntiva e $\Ont$ una ontologia.
\vspace{\baselineskip}

$\sigma$ \`e detta essere una \emph{soluzione} per $Q$ rispetto ad $\Ont$ se
e solo se $T_1\sigma, \ldots, T_2\sigma$ compaiono in $\Ont$. 
\vspace{\baselineskip}

Consideriamo l'ontologia $\Ont$ e la query $Q$ definite come segue:
\[
 \begin{array}{ll}
  \begin{array}{cl}
    \Ont  =  &  \{Female(Elise), Female(Alice), Male(Bob), \\
    &\phantom{\{}Male(Charlie), Male(Daniel), \\
    &\phantom{\{}Alice\,childOf\,Elise, Charlie\,childOf\,Elise, \\
    &\phantom{\{}Daniel\,childOf\,Alice, Daniel\,childOf\,Bob, \\
    &\phantom{\{}Francis\,childOf\,Charlie \}\\
    &\\
    Q = & y\,childOf\,x\,\wedge\,Male(y) .
  \end{array} & 
  \includegraphics[width=130px]{family_ex3.png} \\
 \end{array}
\]
Sia $\sigma_3=[x \rightarrow Charlie, y \rightarrow Francis]$. $\sigma_3$ \`e una soluzione per $Q$
rispetto ad $\Ont$?
\end{frame}

\begin{frame}
\frametitle{Soluzioni per una Query - Esempio 3}
Siano $\sigma=[x_1 \rightarrow a_1, \ldots, x_n \rightarrow a_n]$ una sostituzione,
$Q=T_1 \wedge \ldots \wedge T_m$ una query congiuntiva e $\Ont$ una ontologia.
\vspace{\baselineskip}

$\sigma$ \`e detta essere una \emph{soluzione} per $Q$ rispetto ad $\Ont$ se
e solo se $T_1\sigma, \ldots, T_2\sigma$ compaiono in $\Ont$. 
\vspace{\baselineskip}

Consideriamo l'ontologia $\Ont$ e la query $Q$ (``Chi sono gli individui con almeno un figlio maschio?'') definite come segue:
\[
 \begin{array}{ll}
  \begin{array}{cl}
    \Ont  =  &  \{Female(Elise), Female(Alice), Male(Bob), \\
    &\phantom{\{}Male(Charlie), Male(Daniel), \\
    &\phantom{\{}Alice\,childOf\,Elise, Charlie\,childOf\,Elise, \\
    &\phantom{\{}Daniel\,childOf\,Alice, Daniel\,childOf\,Bob, \\
    &\phantom{\{}Francis\,childOf\,Charlie \}\\
    &\\
    Q = & y\,childOf\,x\,\wedge\,Male(y) .
  \end{array} & 
  \includegraphics[width=130px]{family_ex3.png} \\
 \end{array}
\]
Sia $\sigma_3=[x \rightarrow Charlie, y \rightarrow Francis]$. $\sigma_2$ \`e una soluzione per $Q$
rispetto ad $\Ont$? \textbf{NO}.
\[
 Q\sigma_3 = Francis\,childOf\,Charlie\,\wedge\,\mathbf{Male(Francis)} .
\]
\end{frame}

\begin{frame}
\frametitle{Soluzioni Minimali per una Query}
Siano $\sigma=[x_1 \rightarrow a_1, \ldots, x_n \rightarrow a_n]$ 
una sostituzione, $Q$ una query congiuntiva e $\Ont$ una ontologia.
\vspace{\baselineskip}

$\sigma$ \`e una \emph{soluzione minimale} per $Q$ rispetto a $\Ont$
se e solo se:
\begin{enumerate}
 \item $\sigma$ \`e una soluzione per $Q$ rispetto ad $\Ont$ e inoltre
 \item tutte le variabili $x_1, \ldots, x_n$ che compaiono in $\sigma$
 compaiono anche in $Q$ (criterio di minimalit\`a).
\end{enumerate}
Consideriamo ad esempio
\[
\begin{array}{cl}
  \Ont  =  &  \{Female(Elise), Female(Alice), Male(Bob), \\
  &\phantom{\{}Male(Charlie), Male(Daniel), \\
  &\phantom{\{}Alice\,childOf\,Elise, Charlie\,childOf\,Elise, \\
  &\phantom{\{}Daniel\,childOf\,Alice, Daniel\,childOf\,Bob, \\
  &\phantom{\{}Francis\,childOf\,Charlie \}\\
  &\\
  Q = & y\,childOf\,x\,\wedge\,Male(y) .
 \end{array}
\]
$\sigma_5=[x \rightarrow Elise, y \rightarrow Charlie, z \rightarrow Francis]$ \`e una 
soluzione minimale per $Q$ rispetto a $\Ont$? \phantom{\textbf{NO}.}
\vspace{\baselineskip}

\phantom{$\sigma_6=[x \rightarrow Elise, y \rightarrow Charlie]$ \`e una 
soluzione minimale per $Q$ rispetto a $\Ont$? \textbf{SI}.}
\end{frame}

\begin{frame}
\frametitle{Soluzioni Minimali per una Query}
Siano $\sigma=[x_1 \rightarrow a_1, \ldots, x_n \rightarrow a_n]$ 
una sostituzione, $Q$ una query congiuntiva e $\Ont$ una ontologia.
\vspace{\baselineskip}

$\sigma$ \`e una \emph{soluzione minimale} per $Q$ rispetto a $\Ont$
se e solo se:
\begin{enumerate}
 \item $\sigma$ \`e una soluzione per $Q$ rispetto ad $\Ont$ e inoltre
 \item tutte le variabili $x_1, \ldots, x_n$ che compaiono in $\sigma$
 compaiono anche in $Q$ (criterio di minimalit\`a).
\end{enumerate}
Consideriamo ad esempio
\[
\begin{array}{cl}
  \Ont  =  &  \{Female(Elise), Female(Alice), Male(Bob), \\
  &\phantom{\{}Male(Charlie), Male(Daniel), \\
  &\phantom{\{}Alice\,childOf\,Elise, Charlie\,childOf\,Elise, \\
  &\phantom{\{}Daniel\,childOf\,Alice, Daniel\,childOf\,Bob, \\
  &\phantom{\{}Francis\,childOf\,Charlie \}\\
  &\\
  Q = & y\,childOf\,x\,\wedge\,Male(y) .
 \end{array}
\]
$\sigma_5=[x \rightarrow Elise, y \rightarrow Charlie, \mathbf{z} \rightarrow Francis]$ \`e una 
soluzione minimale per $Q$ rispetto a $\Ont$? \textbf{NO}.
\vspace{\baselineskip}

$\sigma_6=[x \rightarrow Elise, y \rightarrow Charlie]$ \`e una 
soluzione minimale per $Q$ rispetto a $\Ont$? \phantom{\textbf{SI}}.
\end{frame}

\begin{frame}
\frametitle{Soluzioni Minimali per una Query}
Siano $\sigma=[x_1 \rightarrow a_1, \ldots, x_n \rightarrow a_n]$ 
una sostituzione, $Q$ una query congiuntiva e $\Ont$ una ontologia.
\vspace{\baselineskip}

$\sigma$ \`e una \emph{soluzione minimale} per $Q$ rispetto a $\Ont$
se e solo se:
\begin{enumerate}
 \item $\sigma$ \`e una soluzione per $Q$ rispetto ad $\Ont$ e inoltre
 \item tutte le variabili $x_1, \ldots, x_n$ che compaiono in $\sigma$
 compaiono anche in $Q$ (criterio di minimalit\`a).
\end{enumerate}
Consideriamo ad esempio
\[
\begin{array}{cl}
  \Ont  =  &  \{Female(Elise), Female(Alice), Male(Bob), \\
  &\phantom{\{}Male(Charlie), Male(Daniel), \\
  &\phantom{\{}Alice\,childOf\,Elise, Charlie\,childOf\,Elise, \\
  &\phantom{\{}Daniel\,childOf\,Alice, Daniel\,childOf\,Bob, \\
  &\phantom{\{}Francis\,childOf\,Charlie \}\\
  &\\
  Q = & y\,childOf\,x\,\wedge\,Male(y) .
 \end{array}
\]
$\sigma_5=[x \rightarrow Elise, y \rightarrow Charlie, \mathbf{z} \rightarrow Francis]$ \`e una 
soluzione minimale per $Q$ rispetto a $\Ont$? \textbf{NO}.
\vspace{\baselineskip}

$\sigma_6=[x \rightarrow Elise, y \rightarrow Charlie]$ \`e una 
soluzione minimale per $Q$ rispetto a $\Ont$? \textbf{SI}.
\end{frame}

\begin{frame}
\frametitle{Conjunctive Query Answering}
Il problema del \emph{Conjunctive Query Answering} consiste nel trovare
tutte le soluzioni minimali di una query congiuntiva rispetto ad una ontologia.
\vspace{\baselineskip}

\begin{center}
Esempio 1:``Trova tutti gli individui maschi.'' 
\end{center}
%\vspace{\baselineskip}

\begin{tabular}{lc}
$\begin{array}{cl}
  \Ont  =  &  \{Female(Elise),  Female(Alice), Male(Bob), \\
  &\phantom{\{}Male(Charlie), Male(Daniel), \\
  &\phantom{\{}Alice\,childOf\,Elise, Charlie\,childOf\,Elise, \\
  &\phantom{\{}Daniel\,childOf\,Alice, Daniel\,childOf\,Bob, \\
  &\phantom{\{}Francis\,childOf\,Charlie \}\\
\end{array}$ &   \includegraphics[width=130px]{family.png} \\
 $Q=Male(x)$ &
 $\begin{array}{|c|}
  \hline
  x\\
  \hline
  Bob\\
  Charlie\\
  Daniel\\
  \hline
\end{array}$ \\
\end{tabular}
\end{frame}


\begin{frame}
\frametitle{Conjunctive Query Answering - Esempio 2}
\begin{center}
``Chi sono gli individui con almeno un figlio maschio?'' 
\end{center}

\begin{tabular}{lc}
$\begin{array}{cl}
  \Ont  =  &  \{Female(Elise),  Female(Alice), Male(Bob), \\
  &\phantom{\{}Male(Charlie), Male(Daniel), \\
  &\phantom{\{}Alice\,childOf\,Elise, Charlie\,childOf\,Elise, \\
  &\phantom{\{}Daniel\,childOf\,Alice, Daniel\,childOf\,Bob, \\
  &\phantom{\{}Francis\,childOf\,Charlie \}\\
\end{array}$ &   \includegraphics[width=130px]{family.png} \\
 $Q=y\,childOf\,x\,\wedge\,Male(y)$ &
$\begin{array}{|c|c|}
  \hline
  x & y\\
  \hline
  Elise&Charlie\\
  Alice&Daniel\\
  Bob&Daniel\\
  \hline
\end{array}$\\
\end{tabular}
\end{frame}


\begin{frame}
\frametitle{Conjunctive Query Answering - Esempio 3}
\begin{center}
``Chi sono i figli di $Elise$?''
\end{center}

\begin{tabular}{lc}
$\begin{array}{cl}
  \Ont  =  &  \{Female(Elise),  Female(Alice), Male(Bob), \\
  &\phantom{\{}Male(Charlie), Male(Daniel), \\
  &\phantom{\{}Alice\,childOf\,Elise, Charlie\,childOf\,Elise, \\
  &\phantom{\{}Daniel\,childOf\,Alice, Daniel\,childOf\,Bob, \\
  &\phantom{\{}Francis\,childOf\,Charlie \}\\
\end{array}$ & \includegraphics[width=130px]{family.png} \\
$Q=x\,childOf\,Elise$ &
$\begin{array}{|c|}
  \hline
  x\\
  \hline
  \phantom{Alice}\\
  \phantom{Charlie}\\
  \hline
\end{array}$ \\
\end{tabular}
\end{frame}

\begin{frame}
\frametitle{Conjunctive Query Answering - Esempio 3}
\begin{center}
``Chi sono i figli di $Elise$?''
\end{center}

\begin{tabular}{lc}
$\begin{array}{cl}
  \Ont  =  &  \{Female(Elise),  Female(Alice), Male(Bob), \\
  &\phantom{\{}Male(Charlie), Male(Daniel), \\
  &\phantom{\{}Alice\,childOf\,Elise, Charlie\,childOf\,Elise, \\
  &\phantom{\{}Daniel\,childOf\,Alice, Daniel\,childOf\,Bob, \\
  &\phantom{\{}Francis\,childOf\,Charlie \}\\
\end{array}$ & \includegraphics[width=130px]{family.png} \\
$Q=x\,childOf\,Elise$ &
$\begin{array}{|c|}
  \hline
  x\\
  \hline
  Alice\\
  Charlie\\
  \hline
\end{array}$ \\
\end{tabular}
\end{frame}

\begin{frame}
\frametitle{Conjunctive Query Answering - Esempio 4}
\begin{center}
``Chi sono gli individui con almeno un figlio maschio ed una femmina?''
\end{center}

\begin{tabular}{lc}
$\begin{array}{cl}
  \Ont  =  &  \{Female(Elise),  Female(Alice), Male(Bob), \\
  &\phantom{\{}Male(Charlie), Male(Daniel), \\
  &\phantom{\{}Alice\,childOf\,Elise, Charlie\,childOf\,Elise, \\
  &\phantom{\{}Daniel\,childOf\,Alice, Daniel\,childOf\,Bob, \\
  &\phantom{\{}Francis\,childOf\,Charlie \}\\
\end{array}$ & \includegraphics[width=130px]{family.png} \\
$\phantom{Q=y\,childOf\,x\,\wedge\,z\,childOf\,x\,\wedge\,Male(y)\,\wedge\,Female(z)}$ &
$\begin{array}{|c|c|c|}
  \hline
  x&y&z\\
  \hline
  \phantom{Elise}&\phantom{Charlie}&\phantom{Alice}\\
  \hline
\end{array}$ \\
\end{tabular}
\end{frame}

\begin{frame}
\frametitle{Conjunctive Query Answering - Esempio 4}
\begin{center}
``Chi sono gli individui con almeno un figlio maschio ed una femmina?''
\end{center}

\begin{tabular}{lc}
$\begin{array}{cl}
  \Ont  =  &  \{Female(Elise),  Female(Alice), Male(Bob), \\
  &\phantom{\{}Male(Charlie), Male(Daniel), \\
  &\phantom{\{}Alice\,childOf\,Elise, Charlie\,childOf\,Elise, \\
  &\phantom{\{}Daniel\,childOf\,Alice, Daniel\,childOf\,Bob, \\
  &\phantom{\{}Francis\,childOf\,Charlie \}\\
\end{array}$ & \includegraphics[width=130px]{family.png} \\
$Q=y\,childOf\,x\,\wedge\,z\,childOf\,x\,\wedge\,Male(y)\,\wedge\,Female(z)$ &
$\begin{array}{|c|c|c|}
  \hline
  x&y&z\\
  \hline
  \phantom{Elise}&\phantom{Charlie}&\phantom{Alice}\\
  \hline
\end{array}$ \\
\end{tabular}
\end{frame}

\begin{frame}
\frametitle{Conjunctive Query Answering - Esempio 4}
\begin{center}
``Chi sono gli individui con almeno un figlio maschio ed una femmina?''
\end{center}

\begin{tabular}{lc}
$\begin{array}{cl}
  \Ont  =  &  \{Female(Elise),  Female(Alice), Male(Bob), \\
  &\phantom{\{}Male(Charlie), Male(Daniel), \\
  &\phantom{\{}Alice\,childOf\,Elise, Charlie\,childOf\,Elise, \\
  &\phantom{\{}Daniel\,childOf\,Alice, Daniel\,childOf\,Bob, \\
  &\phantom{\{}Francis\,childOf\,Charlie \}\\
\end{array}$ & \includegraphics[width=130px]{family.png} \\
$Q=y\,childOf\,x\,\wedge\,z\,childOf\,x\,\wedge\,Male(y)\,\wedge\,Female(z)$ &
$\begin{array}{|c|c|c|}
  \hline
  x&y&z\\
  \hline
  Elise&Charlie&Alice\\
  \hline
\end{array}$ \\
\end{tabular}
\end{frame}


\begin{frame}
\frametitle{Conjunctive Query Answering - Esempio 5}
\begin{center}
``Chi sono gli individui maschi con almeno una figlia femmina?''
\end{center}

\begin{tabular}{lc}
$\begin{array}{cl}
  \Ont  =  &  \{Female(Elise), Female(Alice), Male(Bob), \\
  &\phantom{\{}Male(Charlie), Male(Daniel), \\
  &\phantom{\{}Alice\,childOf\,Elise, Charlie\,childOf\,Elise, \\
  &\phantom{\{}Daniel\,childOf\,Alice, Daniel\,childOf\,Bob, \\
  &\phantom{\{}Francis\,childOf\,Charlie \}\\
\end{array}$ & \includegraphics[width=130px]{family.png} \\
$\phantom{Q=Male(x)\,\wedge\,y\,childOf\,x\,\wedge\,Female(y)}$ &
$\begin{array}{|c|c|}
  \hline
  x&y\\
  \hline
  &\\
  &\\
  \hline
\end{array}$\\
\end{tabular}
\end{frame}

\begin{frame}
\frametitle{Conjunctive Query Answering - Esempio 5}
\begin{center}
``Chi sono gli individui maschi con almeno una figlia femmina?''
\end{center}

\begin{tabular}{lc}
$\begin{array}{cl}
  \Ont  =  &  \{Female(Elise), Female(Alice), Male(Bob), \\
  &\phantom{\{}Male(Charlie), Male(Daniel), \\
  &\phantom{\{}Alice\,childOf\,Elise, Charlie\,childOf\,Elise, \\
  &\phantom{\{}Daniel\,childOf\,Alice, Daniel\,childOf\,Bob, \\
  &\phantom{\{}Francis\,childOf\,Charlie \}\\
\end{array}$ & \includegraphics[width=130px]{family.png} \\
$Q=Male(x)\,\wedge\,y\,childOf\,x\,\wedge\,Female(y)$ &
$\begin{array}{|c|c|}
  \hline
  x&y\\
  \hline
  &\\
  &\\
  \hline
\end{array}$\\
\end{tabular}
\end{frame}

\begin{frame}
\frametitle{Conjunctive Query Answering - Esempio 5}
\begin{center}
``Chi sono gli individui maschi con almeno una figlia femmina?''
\end{center}

\begin{tabular}{lc}
$\begin{array}{cl}
  \Ont  =  &  \{Female(Elise), Female(Alice), Male(Bob), \\
  &\phantom{\{}Male(Charlie), Male(Daniel), \\
  &\phantom{\{}Alice\,childOf\,Elise, Charlie\,childOf\,Elise, \\
  &\phantom{\{}Daniel\,childOf\,Alice, Daniel\,childOf\,Bob, \\
  &\phantom{\{}Francis\,childOf\,Charlie \}\\
\end{array}$ & \includegraphics[width=130px]{family.png} \\
$Q=Male(x)\,\wedge\,y\,childOf\,x\,\wedge\,Female(y)$ & Nessuna soluzione\\
\end{tabular}
\end{frame}


\section{SPARQL}

\begin{frame}
 \frametitle{Il protocollo SPARQL}
 Le basi di conoscenza presenti sul Web Semantico usualmente mettono 
 a disposizione uno \emph{SPARQL endpoint} che permette di interrogarle e,
 ove permesso, di modificarle.
 
 \begin{table}
 \begin{tabular}{|c|l|}
  \hline
  \bf{Knowledge Base} & \bf{Endpoint IRI} \\
  Europeana & \url{http://europeana.ontotext.com/sparql} \\
  CNR & \url{http://data.cnr.it/sparql/}\\
  Camera dei Deputati & \url{http://dati.camera.it/sparql}\\
  DBPedia & \url{http://dbpedia.org/sparql}\\
  \hline
 \end{tabular} 
 \caption{Alcuni endpoint sparql}
 \end{table}

\uncover<2->{
Il \emph{protocollo SPARQL} (vedi \url{http://www.w3.org/TR/sparql11-protocol/})
\`e basato sul protocollo HTTP le richieste SPARQL vengono inviate agli
endpoint come richieste GET o POST e l'endpoint risponde con un \emph{esito}.
}
\end{frame}

\begin{frame}[fragile]
\frametitle{SPARQL Query Language}
Le richieste di tipo \emph{query} vanno specificate nel linguaggio
denominato \emph{SPARQL Query}.
%
La specifica di questo linguaggio \`e disponibile all'indirizzo
\begin{center}
 \url{http://www.w3.org/TR/sparql11-query/} .
\end{center}

Una \emph{query SPARQL} ha la seguente sintassi
\begin{Verbatim}[fontsize=\small]
SELECT ?x1 ... ?xm WHERE { Q }
\end{Verbatim}
dove: 
\begin{itemize}
 \item $?x1, \ldots, ?xm$ sono \emph{variabili} ($m>0$);
 \item $Q$ \`e una query congiuntiva nella quale compaiono tutte le variabili
 $?x1, \ldots, ?xm$;
 \item le formule atomiche di tipo $C(?x)$ sono scritte come $?x\,a\,C$.
\end{itemize}
\end{frame}

\begin{frame}[fragile]
\frametitle{SPARQL - Esempi}

Alcuni esempi di query SPARQL sono:
  \begin{itemize}
 \item ``Trova tutti gli individui maschi.''
  \begin{Verbatim}[fontsize=\small]
    SELECT ?x WHERE { ?x a <http://example.org/Male> }
  \end{Verbatim}
  \item ``Chi sono gli individui con almeno un figlio maschio?''
  \begin{Verbatim}[fontsize=\small]
    SELECT ?x WHERE { ?y <http://example.org/childOf> ?x . 
                                      ?y a <http://example.org/Male> }
  \end{Verbatim}
  \item ``Chi sono i figli di $Alice$?''
  \begin{Verbatim}[fontsize=\small]
    SELECT ?x WHERE { ?x <http://example.org/childOf> <http://example.org/Alice> }
  \end{Verbatim}
 \end{itemize}
\end{frame}

\end{document}
