\documentclass[8pt]{beamer}
\usepackage[nobglogo]{beamerthemedmi-owled}
\usepackage[utf8x]{inputenc}
\usepackage{default}
\usepackage{url}
\usepackage{verbatim}
\usepackage{graphicx}
\usepackage{mathrsfs}
\usepackage[official]{eurosym}

%\usepackage{listings}


\mode<presentation>
{
  \usetheme{dmi-owled}
  %\usetheme{Warsaw}
  % or ...

  \setbeamercovered{transparent}
  % or whatever (possibly just delete it)
}

\title{Introduzione agli Open Data\\
Dati Strutturati}

\author{Cristiano Longo\\ 
{\small{longo@dmi.unict.it}}}



\date{Universit\`a di Catania}

\begin{document}
\maketitle
\setcounter{tocdepth}{1}

\begin{frame}
 \frametitle{Dati Strutturati}
 
 Alcuni dataset vengono forniti attraverso formati pi\`u strutturati,
 che consentono ad esempio svariati livelli di annidamento.
 \vspace{\baselineskip}
 
 Vedremo i formati XML e JSON.
\end{frame}

\section{XML}

\begin{frame}
 \frametitle{XML}
 Il linguaggio \emph{eXtended Markup Language} (in breve \emph{XML})\footnote{\url{http://www.w3.org/XML/}} 
 \`e un liguaggio di marcatura basato su SGML ed \`e alla base del 
 linguaggio HTML (in particolare xHTML). XML \`e una specifica
 w3c, attualmente alla versione 1.1.\footnote{\url{http://www.w3.org/TR/2006/REC-xml11-20060816/}}
 \vspace{\baselineskip}

\uncover<2->{
 Un documento XML \`e un documento di testo (codificato in UTF-8 o UTF16) 
 che rispetta le regole di produzione indicate nella specifica. 
 Un validatore per XML \`e disponibile al seguente indirizzo.
\begin{center}
\begin{small}
  \url{http://www.w3schools.com/xml/xml_validator.asp}
\end{small} 
\end{center}
 \vspace{\baselineskip}
}

\uncover<3->{
 Escludendo le dichiarazioni, un documento XML \`e strutturato ad albero.
 Ogni nodo dell'albero pu\`o essere un \emph{elemento}, che pu\`o
 a sua volta contenere altri elementi, o un nodo di testo(TextNode).\footnote{Alcuni 
 esempi sono disponibili in \url{http://www.w3schools.com/xml/}}
}
\end{frame}

\begin{frame}[fragile]
 \frametitle{Sintassi XML - Elementi}

 
 Un elemento \`e caratterizzato da un \emph{element name}
 ed \`e delimitato da uno \emph{start tag} e un \emph{end tag},
 tranne nel caso di elementi vuoti.
 Un esempio di elemento con element name \texttt{myElement} \`e
 il seguente:
\begin{verbatim}
  <myElement>
    just a text content
  </myElement>
\end{verbatim}

 L'esempio seguente mostra un elemento vuoto.
\begin{verbatim}
  <emptyElement />
\end{verbatim}
  
 Ogni elemento pu\`o avere altri nodi come figli.
 Nell'esempio seguente viene mostrato un elemento 
 di tipo \texttt{myElement} con due nodi figli:
 un secondo element di tipo \texttt{elementChild}
 ed un nodo di testo.
\begin{verbatim}
  <myElement>
    <elementChild>child text content</elementChild>
    another text node
  </myElement>
\end{verbatim}
\end{frame}

\begin{frame}[fragile]
 \frametitle{XML - Attributi}
 Ogni elemento pu\`o avere degli \emph{attributi}.
 Un attributo di un elemento \`e una coppia nome-valore. 
 Nell'esempio seguente viene mostrato un elemento vuoto 
 di tipo \texttt{myElement} con un attributo \texttt{myAttr} 
 con valore \texttt{attrValue}.
\begin{verbatim}
  <myElement myAttribute="attrValue" />
\end{verbatim}
\end{frame}

\begin{frame}[fragile]
 \frametitle{XML - Struttura}
 Un \emph{documento XML} \`e suddiviso in due parti: prologo e 
 contenuto. Il prologo deve contenere la \emph{XML Declaration},
 mentre il contenuto \`e riservato agli elementi e deve contenere
 almeno un elemento (la radice).
 \vspace{\baselineskip}

 Riportiamo un esempio di documento XML:\footnote{Esempio da \url{http://www.w3schools.com/xml/xml_tree.asp} e lievemente modificato.}
\begin{small}
\begin{verbatim}
<?xml version="1.1" encoding="UTF-16"?> <!-- Prologo -->

<!-- Contenuto -->
<bookstore> <!-- Elemento radice -->
  <book category="COOKING">
    <title lang="en">Everyday Italian</title>
    <author>Giada De Laurentiis</author>
    <year>2005</year>
    <price>30.00</price>
  </book>
  <book category="WEB">
    <title lang="en">Learning XML</title>
    <author>Erik T. Ray</author>
    <year>2003</year>
    <price>39.95</price>
  </book>
</bookstore> 
\end{verbatim}
\end{small}
\end{frame}

\begin{frame}
 \frametitle{XML - Entit\`a}
 
 Le \emph{Entity References} sono particolari token che 
 vengono sostituiti con dei valori predefiniti (o dichiarati nel prologo)
 quando viene effettuato il parsing XML. Le entity references possono 
 comparire nei nodi di testo e come valore degli attributi degli elementi.
 \vspace{\baselineskip}
 
 Una entity reference inizia con il carttere \texttt{\&} e termina con \texttt{;}.
 Le entit\`a predefinite nel linguaggio XML sono le seguenti:
\[
 \begin{array}{|c|c|}
  \hline
    Entity Reference & Carattere \\
    \hline
    \mathtt{\&lt;} & \mathtt{<} \\
    \mathtt{\&gt;} & \mathtt{>} \\
    \mathtt{\&amp;} & \mathtt{\&}\\ 
    \mathtt{\&apos;} & \mathtt{'}\\
    \mathtt{\&quot;} & \mathtt{"}\\
  \hline
 \end{array}
\]
 \vspace{\baselineskip}

 Inoltre, \`e possibile fare riferimento ad uno specifico carattere
 mediante il codice attraverso il quale \`e identificato il carattere.
 In questo caso, si usa la sintassi \texttt{\&\#code;}, sostituendo 
 a \texttt{<code>} il codice unicode del carattere che si vuole inserire
 nel documento
 ad esempio, tutte le occorrenze di \texttt{\&\#8364;} verranno 
 sostituite con il carattere \texttt{\euro} durante il parsing di un 
 documento.
\end{frame}

\begin{frame}[fragile]
 \frametitle{XML - Entit\`a - Esempio}
 Riportiamo un esempio di documento XML che contiene delle entit\`a:
 \vspace{\baselineskip}
 
 \emph{Documento Originale}
\begin{small}
\begin{verbatim}
<?xml version="1.1" encoding="UTF-16"?>
<message>
  Non usare il tag &lt;euro /&gt; &#33;
</message>
\end{verbatim}
\end{small}

 \emph{Documento Interpretato} (33 \`e il codice unicode per \texttt{!})
\begin{small}
\begin{verbatim}
<?xml version="1.1" encoding="UTF-16"?>
<message>
  Non usare il tag <euro /> !
</message>
\end{verbatim}
\end{small} 
\end{frame}

\begin{frame}[fragile]
 \frametitle{XML - Document Type Declaration (1/3)}
Nel prologo di un documento XML pu\`o essere specificata una
\emph{Document Type Declaration}. 
\vspace{\baselineskip}

La document type declaration
specifica il \emph{tipo} di documento in termini di struttura
(tipi elementi, possibili annidamenti, attributi degli elementi).
\vspace{\baselineskip}

Riportiamo un esempio di documento xml che segue la specifica
(leggi \emph{\'e di tipo}) HTML5:\footnote{\url{http://www.w3.org/TR/html5/}}

\begin{small}
\begin{verbatim}
<?xml version="1.0" encoding="UTF-8" standalone="yes"?>
<!DOCTYPE html> <!-- the doctype declaration -->

<html>
  <head>
    <meta charset="UTF-8">
    <title>Title of the document</title>
   </head>
   <body>
     Content of the document......
   </body>
</html> 
\end{verbatim}
\end{small}
\vspace{\baselineskip}

NB: nei documenti HTML5 la XML Declaration pu\`o essere omessa.
\end{frame}

\begin{frame}[fragile]
 \frametitle{XML - Document Type Declaration (2/3)}
 
 La struttura pu\`o essere specificata:
 \vspace{\baselineskip}
 
 - usando i formati DTD e XML Schema;
 \vspace{\baselineskip}

 - internamente alla Document Type Declaration, ad esempio\footnote{Esempio da \url{http://www.w3schools.com/xml/xml_dtd_intro.asp} .}
\begin{small}
\begin{verbatim}
<!DOCTYPE note [
  <!ELEMENT note (to,from,heading,body)>
  <!ELEMENT to (#PCDATA)>
  <!ELEMENT from (#PCDATA)>
  <!ELEMENT heading (#PCDATA)>
  <!ELEMENT body (#PCDATA)>
]> 
\end{verbatim}
\end{small}

 - o come risorsa esterna al documento, ad esempio
\begin{small}
\begin{verbatim}
<!DOCTYPE note SYSTEM "note.dtd">
\end{verbatim}
\end{small}
\end{frame}

\begin{frame}[fragile]
 \frametitle{XML - Document Type Declaration (3/3)}
 Inoltre, all'interno della document type declaration possono essere
 specificate nuove entity refence.

\begin{small}
\begin{verbatim}
<!DOCTYPE rdf:RDF [
    <!ENTITY org "http://www.w3.org/ns/org#" >
    <!ENTITY dcterms "http://purl.org/dc/terms/" >
    <!ENTITY locn "http://www.w3.org/ns/locn#" >
    <!ENTITY foaf "http://xmlns.com/foaf/0.1/" >
    <!ENTITY owl "http://www.w3.org/2002/07/owl#" >
    <!ENTITY xsd "http://www.w3.org/2001/XMLSchema#" >
    <!ENTITY rdfs "http://www.w3.org/2000/01/rdf-schema#" >
    <!ENTITY geo "http://www.w3.org/2003/01/geo/wgs84_pos#" >
    <!ENTITY rdf "http://www.w3.org/1999/02/22-rdf-syntax-ns#" >
    <!ENTITY odt "http://www.dmi.unict.it/~longo/opendatatour/" >
    <!ENTITY event "http://purl.org/NET/c4dm/event.owl#" >
    <!ENTITY time "http://www.w3.org/2006/time#" >
    <!ENTITY cct "http://www.comune.catania.it/comunect.owl/" >
]>
\end{verbatim}
\end{small}
\end{frame}

\begin{frame}
 \frametitle{XML - Contratti Pubblici}
 
 Tra gli adempimenti previsti dalla legge 190/2012
 vi \`e quello per le pubbliche amministrazioni di pubblicare
 un riepilogo delle gare d'appalto, degli affidamenti e dello 
 stato contratti stipulati mediante queste.
 \vspace{\baselineskip}
 
 Tale elenco deve essere pubblicato nella sezione \emph{Amministrazione
 Trasparente} del sito di ogni pubblica amministrazione in formato 
 XML che rispetti gli schemi XML creati all'uopo.\footnote{\url{http://dati.avcp.it/schema/datasetAppaltiL190.xsd}}
 
 Le modalit\`a di pubblicazione sono indicate alla seguente pagina
 \begin{center}
  \begin{small}
   \url{http://www.anticorruzione.it/portal/public/classic/Servizi/ServiziOnline/DichiarazioneAdempLegge190} .
  \end{small}
 \end{center}
 ed \`e anche disponibile l'elenco delle amministrazioni il cui file sia stato 
 correttamente recepito dall'Autorit\`a Nazionale AntiCorruzione (ANAC).
 \begin{center}
  \begin{small}
   \url{http://dati.anticorruzione.it/L190.html} .
  \end{small}
 \end{center} 
 
 Le \emph{specifiche tecniche}\footnote{\url{http://www.anticorruzione.it/portal/rest/jcr/repository/collaboration/Digital\%20Assets/anacdocs/Servizi/ServiziOnline/AdempimentoLegge190/SpecificheTecnicheL190v1.1.pdf}}
 contengono una descrizione esplicativa del formato XML da usare e degli esempi.
\end{frame}

\begin{frame}[fragile]
 \frametitle{XML - Contratti Pubblici - Metadati}
 Un file per la comunicazione di gare \`e contratti \`e costituito 
 da una sezione contenente i \emph{metadati} del file (data pubblicazione,
 anno di riferimento, ...) ed una riguardante i \emph{lotti}.
 Riportiamo qui un estratto dell'esempio contenuto nelle
 specifiche tecniche. 
\begin{small}
\begin{verbatim}
<?xml version="1.0" encoding="UTF-8"?> 
<legge190:pubblicazione xsi:schemaLocation="legge190_1_0 datasetAppaltiL190.xsd" 
xmlns:xsi="http://www.w3.org/2001/XMLSchema-instance" xmlns:legge190="legge190_1_0"> 
 <metadata> 
   <titolo> Pubblicazione 1 legge 190</titolo> 
   <abstract> Pubblicazione 1 legge 190 anno 1 rif. 2010</abstract> 
   <dataPubbicazioneDataset>2012-08-13</dataPubbicazioneDataset> 
   <entePubblicatore>ANAC</entePubblicatore> 
   <dataUltimoAggiornamentoDataset>2012-09-15</dataUltimoAggiornamentoDataset> 
   <annoRiferimento>2012</annoRiferimento> 
   <urlFile>http://www.pubblicazione.it/dataset1.xml </urlFile> 
   <licenza>IODL</licenza> 
 </metadata> 
 <data> 
   <lotto> 
     <cig>4939483E4E</cig> 
     <strutturaProponente> 
        <codiceFiscaleProp> 97584460584</codiceFiscaleProp>  
     ...
\end{verbatim}
\end{small}
\end{frame}

\begin{frame}[fragile]
 \frametitle{XML - Contratti Pubblici - Lotti}
 Per ogni lotto si indica innanzitutto il \emph{codice identificativo gara}.
 Tale codice \`e lo stesso che identifica la gara nel sistema \emph{Sistema 
 Informativo Monitoraggio Gare} (in breve SIMOG) dell'ANAC.
 
 \begin{small}
\begin{verbatim}
<?xml version="1.0" encoding="UTF-8"?> 
<legge190:pubblicazione xsi:schemaLocation="legge190_1_0 datasetAppaltiL190.xsd" 
xmlns:xsi="http://www.w3.org/2001/XMLSchema-instance" xmlns:legge190="legge190_1_0"> 
 <metadata> 
 ...
 </metadata>
 <data>
   <lotto>
     <cig>4939483E4E</cig>
     ...
\end{verbatim}
\end{small}

\end{frame}

\begin{frame}[fragile]
 \frametitle{XML - Contratti Pubblici - Gara}
 Successivamente, troviamo le indicazioni sulla stazione appaltante e gli estremi della gara. 
 
\begin{small}
\begin{verbatim}
<?xml version="1.0" encoding="UTF-8"?> 
<legge190:pubblicazione xsi:schemaLocation="legge190_1_0 datasetAppaltiL190.xsd"
xmlns:xsi="http://www.w3.org/2001/XMLSchema-instance" xmlns:legge190="legge190_1_0">
...
 <data>
   <lotto>
     <cig>4939483E4E</cig>
     <strutturaProponente>
        <codiceFiscaleProp>97584460584</codiceFiscaleProp>
        <denominazione>Autorità Nazionale Anticorruzione </denominazione> 
     </strutturaProponente> 
     <oggetto>Gara a procedura aperta per l’affidamento della Fornitura di infrastrutture 
     informatiche per il programma AVCPass
     </oggetto>
     <sceltaContraente>17-AFFIDAMENTO DIRETTO EX ART. 5 DELLA LEGGE N.381/91</sceltaContraente>
     ...
 \end{verbatim}
\end{small}
  Si noti che il contenuto dell'elemento \texttt{sceltaContraente} varia in un insieme finito
  di valori, come indicato nello schema XML.
\end{frame}

\begin{frame}[fragile]
 \frametitle{XML - Contratti Pubblici - Partecipanti alla Gara}
 Per ogni gara vengono indicati i partecipanti, singoli o raggruppamenti.
 Successivamente, troviamo le indicazioni sulla stazione appaltante e gli estremi della gara. 
 
\begin{small}
\begin{verbatim}
<?xml version="1.0" encoding="UTF-8"?> 
<legge190:pubblicazione xsi:schemaLocation="legge190_1_0 datasetAppaltiL190.xsd"
xmlns:xsi="http://www.w3.org/2001/XMLSchema-instance" xmlns:legge190="legge190_1_0">
...
 <data>
   <lotto>
     ...
     <partecipanti>
       <raggruppamento>
         <membro>
            <codiceFiscale>00000000001</codiceFiscale>
            <ragioneSociale>Azienda 1</ragioneSociale>
            <ruolo>04-CAPOGRUPPO</ruolo>
         </membro>
         <membro>
            <codiceFiscale>00000000002</codiceFiscale>
            <ragioneSociale>Azienda 2</ragioneSociale>
            <ruolo>03-ASSOCIATA</ruolo>
         </membro>
       </raggruppamento>
      <partecipante>
         <codiceFiscale>00000000003</codiceFiscale>
         <ragioneSociale>Azienda Individuale 1</ragioneSociale>
      </partecipante>
    </partecipanti>
     ...
 \end{verbatim}
\end{small}
\end{frame}

\begin{frame}[fragile]
 \frametitle{XML - Contratti Pubblici - Aggiudicatari del Contratto}
 Nel caso in cui la gara abbia esito positivo, viene specificato l'aggiudicatario
 e l'importo del contratto.
 
\begin{small}
\begin{verbatim}
<?xml version="1.0" encoding="UTF-8"?> 
<legge190:pubblicazione xsi:schemaLocation="legge190_1_0 datasetAppaltiL190.xsd"
xmlns:xsi="http://www.w3.org/2001/XMLSchema-instance" xmlns:legge190="legge190_1_0">
...
 <data>
   <lotto>
     ...
     <aggiudicatari>
       <aggiudicatarioRaggruppamento>
         <membro>
           <codiceFiscale>00000000001</codiceFiscale>
           <ragioneSociale>Azienda 1</ragioneSociale>
           <ruolo>04-CAPOGRUPPO</ruolo>
         </membro>
         <membro>
           <codiceFiscale>00000000002</codiceFiscale>
           <ragioneSociale>Azienda 2</ragioneSociale>
           <ruolo>03-ASSOCIATA</ruolo>
         </membro>
       </aggiudicatarioRaggruppamento>
   </aggiudicatari>
   <importoAggiudicazione>1000.00</importoAggiudicazione>
   ...
 \end{verbatim}
\end{small}
\end{frame}

\begin{frame}[fragile]
 \frametitle{XML - Contratti Pubblici - Svolgimento dei Lavori}
 Per i contratti in essere e completati, si indica la data di inizio dei lavori.
 Per quelli completati si indica anche la data di chiusura e l'importo complessivo
 liquidato, al netto dell'IVA.
\begin{small}
\begin{verbatim}
<?xml version="1.0" encoding="UTF-8"?> 
<legge190:pubblicazione xsi:schemaLocation="legge190_1_0 datasetAppaltiL190.xsd"
xmlns:xsi="http://www.w3.org/2001/XMLSchema-instance" xmlns:legge190="legge190_1_0">
...
 <data>
   <lotto>
     ...
     <tempiCompletamento>
       <dataInizio>2012-08-13</dataInizio>
       <dataUltimazione>2012-08-13</dataUltimazione>
     </tempiCompletamento>
     <importoSommeLiquidate>1000.00</importoSommeLiquidate>
   </lotto>
   ...
 \end{verbatim}
\end{small}
\end{frame}

\begin{frame}
 \frametitle{XML - Contratti Pubblici - Esempi}
 Il dataset su bandi e contratti (anni 2013 e 2014) del comune di Catania \`e
 disponibile al seguente indirizzo:
\begin{center}
 \begin{small}
  \url{http://www.comune.catania.it/pubblicitaappalti/opendata.aspx} .
 \end{small}
\end{center}

 Un tool realizzato a partire dal dataset ANAC e dalle comunicazioni XML 
 su bandi e contratti \'e \emph{Public Contracts}\footnote{\url{http://public-contracts.nexacenter.org/}} 
 del centro NEXA del Politecnico di Torino. 
\end{frame}

\section{Javascript}

\begin{frame}[fragile]
 \frametitle{Il Linguaggio Javascript}
 Il linguaggio \emph{Javascript}\footnote{\url{http://www.ecma-international.org/publications/standards/Ecma-262.htm}}
 \`e un linguaggio di programmazione imperativo che viene solitamente inserito
 all'interno di pagine HTML per essere \emph{interpretato} ed eseguito dai browser.
 \vspace{\baselineskip} 
 
 Uno \emph{script} viene specificato all'interno 
 del documento HTML usando un elemento di tipo \texttt{script}.
 Nel caso di codice javascript, l'attributo type,
 che rappresenta il tipo MIME del contenuto,
 dell'elemento \texttt{script} deve avere come valore la 
 stringa \texttt{text/javascript}. 
 \vspace{\baselineskip} 
 
 Riportiamo nel seguito un esempio di pagina html
 che incorpora uno script che visualizza una finestra
 con un testo usando la funzione \texttt{alert}.
 \begin{small}
 \begin{verbatim}
<!DOCTYPE html>
<html>
  <head>
    <script type="text/javascript">
      alert("This is an alert window created via JS");
    </script>
  </head>
<body>
  an empty body
</body>
</html>  
 \end{verbatim}
 \end{small}
\end{frame}

\begin{frame}[fragile]
 \frametitle{Javascript - Script esterni}
 Oltre ad essere specificato all'interno della pagina, uno script
 pu\`o essere caricato da una fonte esterna speficandone la URI
 con l'attributo \texttt{src}.
 \begin{small}
 \begin{verbatim}
<!DOCTYPE html>
<html>
  <head>
    <script src="externalScript.js" type="text/javascript"></script>
  </head>
<body>
  an empty body
</body>
</html>  
 \end{verbatim}
 \end{small}
 
\end{frame}

\begin{frame}[fragile]
 \frametitle{Javascript - Esecuzione}
 Gli script vengono solitamente eseguiti durante il rendering della pagina 
 non appena il motore di rendering incontra il tag script. Nel caso
 in cui sia specificato l'attributo \texttt{defer} nel tag script,
 lo script sar\`a eseguito invece al termine del caricamento della pagina.
 Nota bene che il deferimento funziona solo per script esterni.
 \begin{small}
 \begin{verbatim}
<!DOCTYPE html>
<html>
  <head>
    <title>Javascript Defer Example</title>
  </head>
  <body>
    The first part of the page, now the rendering will stop waiting the script execution.
    <script src="ext1.js" type="text/javascript"></script>
    <script src="ext2.js" type="text/javascript" defer></script>
    You can see this just after the former script completes. 
  </body>
</html> 
 \end{verbatim}
 \end{small} 
\end{frame}

\begin{frame}[fragile]
 \frametitle{Javascript - Eventi}
 Inoltre l'esecuzione di codice javascript pu\`o essere collegata al verificarsi 
 di un evento. Nell'esempio seguente il codice specificato come valore dell'attributo 
 \texttt{onclick} viene eseguito quando si preme sull'elemento \texttt{a} corrispondente.
 
 \begin{small}
 \begin{verbatim}
<!DOCTYPE html>
<html>
  <head>
    <title>Event Driven Javascript</title>
  </head>
  <body>
    <p><a href="#" onclick="alert('Click!')">Click here to open a alert window</a></p>
  </body>
</html> 
 \end{verbatim}
 \end{small}
\end{frame}

\begin{frame}[fragile]
 \frametitle{Javascript - Funzioni}
 \`E anche possibile definire delle funzioni, che potranno essere richiamate
 all'interno di script o al verificarsi di determinati eventi, e delle variabili.
 
 \begin{small}
 \begin{verbatim}
<!DOCTYPE html>
<html>
  <head>
    <title>Event Driven Javascript</title>
    <script type="text/javascript">
      var alertStr="Click!";
      function myAlert(){
        alert(alertStr);
      }
    </script>
  </head>
  <body>
    <p><a href="#" onclick="myAlert()">Click here to open a alert window</a></p>
  </body>
</html> 
 \end{verbatim}
 \end{small}
\end{frame}

\begin{frame}[fragile]
 \frametitle{Javascript - Oggetti (1/2)}
 Il linguaggio Javascript appartiene alla famiglia dei linguaggi 
 \emph{prototype-based}.\footnote{\url{https://developer.mozilla.org/it/docs/Web/JavaScript/Introduzione_al_carattere_Object-Oriented_di_JavaScript}}
 In questo paradigma non esistono le classi, ma \`e possibile comunque
 creare degli oggetti ai quali collegare metodi e attributi.
 \vspace{\baselineskip}

 Javascript fornisce alcuni oggetti direttamente nel core.\footnote{\url{http://www.w3schools.com/jsref/default.asp}}
 \vspace{\baselineskip}

 Per definire nuovi oggetti vengono utilizzate le funzioni,
 che fungono da \emph{costruttore} quando l'oggetto viene istanziato.
 con la parola chiave \texttt{new}. Nell'esempio 
 seguente vengono creati due oggetti di \emph{classe} \texttt{TheClass}.
 \begin{small}
 \begin{verbatim}
function MyClass(){ //empty }
var o1 = new MyClass();
var o2 = new MyClass();
 \end{verbatim}
 \end{small}
 
\end{frame}

\begin{frame}[fragile]
 \frametitle{Javascript - Oggetti (2/2)}
 
 \`E possibile poi definire metodi e attributi dell'oggetto
 con una istruzione di assegnazione. La sintassi per accedere ai 
 metodi e agli attributi di un oggetto \`e quella consueta. Nell'esempio
 seguente viene aggiunto un metodo \texttt{showAlert} all'oggetto \texttt{o}.
 Nota che il nuovo metodo viene specificato attraverso una \emph{funzione 
 anonima}.

 \begin{small}
 \begin{verbatim}
 function MyClass(){ } //empty constructor
 var o = new MyClass();
 o.showAlert = function(){
   alert("Showing Alert!");
 } 
 \end{verbatim}
 \end{small}
 
 All'interno dei costruttori e dei metodi 
 di un oggetto \`e possibile fare riferimento 
 allo stesso mediante la parola chiave \texttt{this}.
 \begin{small}
 \begin{verbatim}
 function MyClass(name){ 
   this.name=name; //set the attribute name of the object
   this.printName = function(){ //set a method which access the name attribute
      alert(this.name);
   }
 }
 var o = new MyClass("my name");
 o.printName(); //show the name
 \end{verbatim}
 \end{small}
\end{frame}

\begin{frame}
 \frametitle{Javascript - Document Object Model}

 La specifica \emph{Document Object Model} (in breve DOM)\footnote{\url{http://www.w3.org/TR/dom/}}
 definisce un modello (oggetti, metodi, ...) indipendente dal linguaggio
 per accedere agli elementi di un documento specificato in un linguaggio di
 markup (solitamente HTML o XML).
 \vspace{\baselineskip}
 
 Uno script javascript all'interno di un documento HTML pu\`o
 accedere al documento stesso attraverso la variabile \texttt{document},\footnote{\url{http://www.w3schools.com/jsref/dom_obj_document.asp}}
 e all'elemento radice del documento attraverso \texttt{document.documentElement} (elemento
 di tipo \texttt{html} nel caso di documenti HTML).
 \vspace{\baselineskip}
 
 Dalla radice \`e possibile navigare e modificare tutto 
 il contenuto del documento attraverso i metodi di cui sono
 equipaggiati i suoi elementi.\footnote{\url{http://www.w3schools.com/jsref/dom_obj_all.asp}}
 Ad esempio, dato un elemento \texttt{e}
 \begin{itemize}
  \item \texttt{e.children} restituisce tutti gli elementi figli;
  \item \texttt{e.getElementsByTagName("<tag>")} restituisce tutti gli elementi figli di 
  tipo \texttt{<tag>};
  \item \texttt{e.getElementsByByID("<id>")} restituisce l'elemento figlio con identificativo
  \texttt{<id>}.  
 \end{itemize}

 Si noti che i metodi \texttt{getElementsByTagName} e \texttt{getElementsByByID} 
 sono disponibili anche per l'oggetto \texttt{document}.
\end{frame}

\begin{frame}[fragile]
 \frametitle{Javascript - Esempio}
 Il seguente esempio mostra come aggiungere elementi ad
 una lista non ordinata (\texttt{ul}) con identificativo
 \texttt{theList}.
 
\begin{small}
\begin{verbatim}
<!DOCTYPE html>
<html>
  <head>
    <title>DOM Example</title>
    <script type="text/javascript">
var counter=0;

function addItem(listId){
  var list = document.getElementById(listId);
  var e = document.createElement("li");
  var t = document.createTextNode("Item "+counter);
  e.appendChild(t);
  list.appendChild(e);
  counter++;
}
  </script>
  </head>
  <body>
    <p>
      <a href="#" onclick="addItem('theList')">
       Click here to add a list item
      </a>
    </p>
    <ul id="theList" />
  </body>
</html> 
 \end{verbatim} 
\end{small}
\end{frame}

\begin{frame}[fragile]
 \frametitle{XML e Javascript}
 
 Solitamente XML viene utilizzato come formato di ritorno da molti servizi
 web. Per effettuare una chiamata HTTP il core Javascript mette a disposizione
 l'oggetto \texttt{XMLHttpRequest}. 
 \vspace{\baselineskip}
 
 Per effettuare il parsing di un documento XML \`e possibile
 istanziare un oggetto \texttt{DOMParser}, che fornisce il metodo 
 \texttt{parseFromString}. I metodi per ispezionare e modificare 
 il documento XML sono quelli previsti nella specifica DOM vista 
 prima.
 \vspace{\baselineskip}

 Il seguente esempio effettua il parsing di una stringa in 
 un documento XML e ne stampa il tipo di elemento della radice.
\begin{small}
\begin{verbatim}
    var xmlAsText ="\<?xml version=\"1.1\" encoding=\"UTF-16\" ?\>\n";
    xmlAsText+="<anEmptyElement />"
    var p = new DOMParser();
    var doc=p.parseFromString(xmlAsText,"text/xml");
    alert(doc.documentElement.tagName); 
\end{verbatim}
\end{small}
\end{frame}

\begin{frame}[fragile]
 \frametitle{Javascript e Servizi Web}
 
 Javascript pu\`o essere utilizzato per interrogare direttamente i servizi web. 
 Per effettuare una chiamata HTTP il core Javascript mette a disposizione
 l'oggetto \texttt{XMLHttpRequest}. 
 \vspace{\baselineskip}

\begin{small}
\begin{verbatim}
    var url = "ComunediCatania-2014-DatasetLegge190.xml";
    var xmlhttp = new XMLHttpRequest();
    var p = new DOMParser();
    xmlhttp.onreadystatechange = function() {
        if (xmlhttp.readyState == 4 && xmlhttp.status == 200) {
            var doc = p.parseFromString(xmlhttp.responseText, "text/xml");
            alert(doc.getElementsByTagName("lotto").length);
        }
    }
    xmlhttp.open("GET", url, true);
    xmlhttp.send();
\end{verbatim}
\end{small}

Nel caso in cui la richiesta riguardi un servizio posto in un dominio differente
dalla pagina web cui si fa riferimento \`e necessario che il server remoto supporti
la specifica \emph{Cross-Origin Resource Sharing} (in breve, CORS).\footnote{\url{http://www.w3.org/TR/cors/}} 
In caso contrario il browser segnaler\`a un errore.
\vspace{\baselineskip}

Per aggirare questo problema \`e sufficiente creare un proxy sullo stesso dominio
della pagina web che si occupi di scaricare il file.
\end{frame}

%TODO strumenti del browser
%TODO ID in xml

\section{JSON}

\begin{frame}
 \frametitle{Il linguaggio JSON}
 
 Unaltro tra i pi\`u diffusi formati \`e \emph{JSON},\footnote{\url{http://json.org/}}
 generalmente pi\`u \emph{leggero} di XML.
 \vspace{\baselineskip}

 Un documento JSON \`e un file di testo (con encoding UTF-8) che rispetta la seguente grammatica:

\begin{small}
\[
  \begin{array}{ccl}
object & \rightarrow & \{\}\\
&&\{members\}\\
members & \rightarrow & pair \\
&&pair , members \\
pair & \rightarrow & string : value \\
array & \rightarrow & [] \\
&&[ elements ] \\
elements & \rightarrow & value \\ 
&&value , elements \\
value & \rightarrow & string \\
&&number\\
&&object\\
&&array\\
&&\mathtt{true}\\
&&\mathtt{false}\\
&&\mathtt{null} 
\end{array}
\]
\end{small}
 \vspace{\baselineskip}

 Un esempio di file JSON \`e disponibile al seguente indirizzo:
\begin{center}
 \begin{small}
  \url{http://json.org/example}
 \end{small}
\end{center}
\end{frame}

\begin{frame}[fragile]
 \frametitle{JSON e Javascript (1/3)}
 Il formato JSON \`e supportato nella maggior parte dei liguaggi di programmazione,
 il particolare in quelli orientati al web. Il linguaggio \emph{Javascript} offre 
 un supporto nativo a JSON.\footnote{Vedi \url{http://www.json.org/js.html} .}
 \vspace{\baselineskip}
 
%\uncover<2->{
\`E possibile dichiarare un oggetto JSON all'interno di uno script javascript:
\begin{verbatim}
var obj = { "name" : "Cristiano",
  "familyname" : "Longo",
  "email" : [ "mailto://cristianolongo@gmail.com", 
  "mailto://longo@dmi.unict.it"] }
\end{verbatim}

\`E possibile accedere ai campi di un oggetto con la sintassi $<obj>.<field>$.
Ad esempio il valore del campo familyname dell'oggetto obj si ottiene 
come segue:
\begin{verbatim}
var fn=obj.familyname;
\end{verbatim}

Per gli elementi di un array si usa invece la sintassi $arr[index]$. La
prima mail definita in $obj$ si pu\`o ottenere come segue:

\begin{verbatim}
var firstmail=obj.email[0];
\end{verbatim}
\end{frame}

\begin{frame}[fragile]
 \frametitle{JSON e Javascript (2/3)}
 
 Il metodo $parse$ dell'oggetto $JSON$ si utilizza per ottenere
 il corrispondente oggetto JSON da una stringa.
\begin{verbatim}
var obj = JSON.parse("{\"name\" : \"Cristiano\",  \"familyname\" : \"Longo\"});
\end{verbatim}
\end{frame}

\begin{frame}[fragile]
 \frametitle{JSON e Javascript (2/3)}
 
Il seguente esempio mostra come utilizzare la replica JSON ottenuta da 
una chiamata HTTP GET.\footnote{Questo esempio \`e da \url{http://www.w3schools.com/json/json_http.asp} .}
\begin{verbatim}
var xmlhttp = new XMLHttpRequest();
var url = "http://example.org";

xmlhttp.onreadystatechange = function() {
    if (xmlhttp.readyState == 4 && xmlhttp.status == 200) {
        var obj = JSON.parse(xmlhttp.responseText);
        alert(obj.name);
    }
}
xmlhttp.open("GET", url, true);
xmlhttp.send();
\end{verbatim}
\end{frame}

\begin{frame}[fragile]
 \frametitle{JSON e Javascript - Esempio: linee Bus}
 Un esempio di open data in formato JSON sono le informazioni su linee, 
 orari e percorsi dei BUS fornite da AMT Catania.\footnote{\url{http://www.amt.ct.it/?page_id=4623}}
 Questo servizio non supporta CORS.
 
\begin{verbatim}
{"Lines":
  {"Line":[
    {"id":1,
     "name":"101",
     "description":["Ognina","Barriera","San G. Galermo"],
     "timetables":{
       "weekdays":["06:40","07:45","08:50","09:50","10:40",
         "11:30","12:30","13:40","14:35"],
       "holidays":["Circolare Non Esercita"]},
       "routes":["Via Mon. D. Orlando","V.le Ulisse",...],
       "note":[],
       "polyline":[[37.532129128297,15.108727463789],...]
    },
    {"id":2,
     "name":"1-4",
     ...
    }
  }
}
\end{verbatim}

\begin{small}
NB: il frammento viene presentato abbreviato. I tre punti rappresentano le parti omesse. 
\end{small}
\end{frame}

\section{GeoJSON}

\begin{frame}
 \frametitle{GeoJSON}
 Il formato \emph{GeoJSON}\footnote{\url{http://geojson.org/}} \`e basato su JSON
 e permette la rappresentazione di dati geospaziali (punti, linee, aree).
 \vspace{\baselineskip}

\uncover<2->{
 Un documento GeoJSON contiene sempre un unico oggetto, con un attributo \texttt{type}
 che pu\`o assumere uno dei seguenti valori:
 \begin{itemize}
  \item \texttt{Point} - un singolo punto nello spazio;
  \item \texttt{MultiPoint} - un insieme di punti;
  \item \texttt{LineString} - un segmento o un insieme di segmenti contigui (spezzata);
  \item \texttt{MultiLineString} - un insieme di \texttt{LineString};
  \item \texttt{Polygon} - un poligono;
  \item \texttt{MultiPolygon} - un insieme di poligoni;
  \item \texttt{GeometryCollection} - un array di elementi dei tipi precedenti;
  \item \texttt{Feature} - permette di associare ad un elemento dei tipi precedenti alcuni metadati 
  (ad esempio il nome) e un identificativo;
  \item \texttt{FeatureCollection} - un insieme di \texttt{Feature}.
 \end{itemize}
 
 Esempi esplicativi sono disponibili su \url{https://it.wikipedia.org/wiki/GeoJSON}.
}

\uncover<3->{
  Un oggetto GeoJSON pu\`o avere un attributo \texttt{crs} per indicare il
  sistema di coordinate utilizzate. Nel seguito indicheremo col termine \emph{position}
  un array di coordinate, la cui lunghezza dipende dal sistema di coordinate in uso.
}
\end{frame}

\begin{frame}[fragile]
 \frametitle{GeoJSON - Point}
 Gli elementi di tipo \texttt{Point} hanno un attributo \texttt{coordinates} cui \`e associata una 
 posizione.
 
\begin{verbatim}
{ "type": "Point", "coordinates": [100.0, 0.0] } 
\end{verbatim}
\end{frame}

\begin{frame}[fragile]
 \frametitle{GeoJSON - MultiPoint}
 Gli elementi di tipo \texttt{MultiPoint} hanno un attributo \texttt{coordinates} 
 il cui valore \`e un array di almeno due coordinate. Si utilizza per rappresentare
 insiemi di punti.
  
\begin{verbatim}
{ "type": "MultiPoint",
    "coordinates": [ [100.0, 0.0], [101.0, 1.0] ]
    }
\end{verbatim}
\end{frame}

\begin{frame}[fragile]
 \frametitle{GeoJSON - LineString}
 Gli elementi di tipo \texttt{Point} hanno un attributo \texttt{coordinates} 
 il cui valore \`e un array di almeno due coordinate. Esse indicano i nodi
 della spezzata.
  
\begin{verbatim}
{ "type": "LineString",
    "coordinates": [ [100.0, 0.0], [101.0, 1.0] ]
    }
\end{verbatim}
\end{frame}

\begin{frame}[fragile]
 \frametitle{GeoJSON - MultiLineString}
 Gli elementi di tipo \texttt{Point} hanno un attributo \texttt{coordinates} 
 il cui valore \`e un array di array di coordinate. Ogni elemento dell'array
 principale rappresenta una \texttt{LineString}.
  
\begin{verbatim}
{ "type": "MultiLineString",
    "coordinates": [
        [ [100.0, 0.0], [101.0, 1.0] ],
        [ [102.0, 2.0], [103.0, 3.0] ]
      ]
    }
\end{verbatim}
\end{frame}

\begin{frame}[fragile]
 \frametitle{GeoJSON - Polygon}
 Con \texttt{LinearRing} indichiamo gli array di coordinate nei quali
 il primo elemento coincide con l'ultimo.
 \vspace{\baselineskip}
 
 Gli elementi di tipo \texttt{Polygon} hanno un attributo \texttt{coordinates} 
 il cui valore \`e un array non vuoto di \texttt{LinearRing}. Ogni elemento dell'array
 rappresenta gli \emph{anelli} del poligono, dal pi\`u esterno al pi\`u interno.
 
\begin{verbatim}
{ "type": "Polygon",
    "coordinates": [
      [ [100.0, 0.0], [101.0, 0.0], [101.0, 1.0], [100.0, 1.0], [100.0, 0.0] ],
      [ [100.2, 0.2], [100.8, 0.2], [100.8, 0.8], [100.2, 0.8], [100.2, 0.2] ]
      ]
   }
\end{verbatim}
\end{frame}

\begin{frame}[fragile]
 \frametitle{GeoJSON - MultiPolygon}
 Gli elementi di tipo \texttt{MultiPolygon} hanno un attributo \texttt{coordinates} 
 il cui valore \`e un array non vuoto di elementi che rappresentano poligoni (si veda 
 come sono rappresentate le coordinate per i poligoni).
\begin{verbatim}
{ "type": "MultiPolygon",
    "coordinates": [
      [[[102.0, 2.0], [103.0, 2.0], [103.0, 3.0], [102.0, 3.0], [102.0, 2.0]]],
      [[[100.0, 0.0], [101.0, 0.0], [101.0, 1.0], [100.0, 1.0], [100.0, 0.0]],
       [[100.2, 0.2], [100.8, 0.2], [100.8, 0.8], [100.2, 0.8], [100.2, 0.2]]]
      ]
    }
\end{verbatim}
\end{frame}

\begin{frame}[fragile]
 \frametitle{GeoJSON - GeometryCollection}
 Gli elementi di tipo \texttt{GeometryCollection} hanno un attributo \texttt{geometries} 
 il cui valore \`e un array non vuoto di oggetti dei tipi visti in precedenza.
\begin{verbatim}
{ "type": "GeometryCollection",
    "geometries": [
      { "type": "Point",
        "coordinates": [100.0, 0.0]
        },
      { "type": "LineString",
        "coordinates": [ [101.0, 0.0], [102.0, 1.0] ]
        }
    ]
  }
\end{verbatim}
\end{frame}

\begin{frame}
 \frametitle{GeoJSON - Feature}
 Un oggetto di tipo \texttt{Feature} permette di associare ad un elemento
 geometrico (un oggetto GeoJSON di uno dei tipi visti prima), specificato
 mediante l'attributo \texttt{geometry} un insiemi di meta-dati, indicati 
 da un generico oggetto JSON associato alla feature mediantre l'attributo 
 \texttt{properties}.
 \vspace{\baselineskip}

\uncover<2->{
 Una \texttt{FeatureCollection} \`e un insieme di oggetti di tipo \texttt{Feature}
 specificati come array associato alla \texttt{FeatureCollection} mediante la
 propriet\`a \texttt{features}.
}
\end{frame}

\begin{frame}[fragile]
 \frametitle{GeoJSON - Esempio: Farmacie Comune di Catania}
 Nel portale open data del comune di catania \`e disponibile l'elenco delle farmacie
 in formato GeoJSON.

\begin{center}
 \begin{small}
  \url{http://opendata.comune.catania.gov.it/dataset/test-geo}
\end{small} 
\end{center}
 
Ne riportiamo un frammento. 
\begin{small}
\begin{verbatim}
{
    "type": "FeatureCollection",
    "features": [
        {
            "type": "Feature",
            "properties": {
                "FID": "Farmacie.47",
                "OBJECTID": "91",
                "RECAPITO": "VIA MEDAGLIE D'ORO DELLE 13",
                "NOME": "BELLINAZZI DANILO - FRANCAVIGLIA",
                "NUMERO": "74",
                "CODICE": "19426",
                "MUNI": "8",
                "PROPRIETA": "BELLINAZZI DANILO"
            },
            "geometry": {
                "type": "Point",
                "coordinates": [
                    15.069901081722616,
                    37.50463603412214
                ]
            }
        },
     ...]
}     
\end{verbatim}
\end{small}
\end{frame}
\end{document}
